\documentclass{article}
\usepackage[T1]{fontenc}
\usepackage{garamondx}
\usepackage[garamondx,cmbraces]{newtxmath}
\usepackage{fullpage}
\usepackage[backend=biber, sorting=none]{biblatex}
\usepackage{setspace}

\newcommand{\sw}[1]{\textit{#1}}
\newcommand{\etal}{\textit{et al.}}

\title{Phylogenetic estimation of contact network parameters with kernel-ABC}
\author{Rosemary McCloskey \and Art FY Poon}
\addbibresource{papers.bib}

\begin{document}

\maketitle

\onehalfspacing

\section{Abstract}

Models of the spread of HIV in a population often make the simplifying
assumption that the population is homogeneously mixed, or is divided into
homogeneously mixed compartments. However, human populations are likely to have
complex structures formed by social contacts; this structure can have a
significant influence on the rate of epidemic spread. Contact network models
capture this structure by explicitly representing each possible transmission. 

We developed a tree kernel-based approximate Bayesian computation (kernel-ABC)
method for estimating structural parameters of the contact network underlying
an observed HIV transmission tree, and used it to investigate the preferential
attachment (PA) power parameter of the Barab\'asi-Albert network model. To
validate the method, we constructed networks of 5000 nodes with mean degree 4
and PA power values \{0, 0.25, \ldots, 2\}. Epidemics were simulated over these
networks until 1000 nodes were infected, then transmission trees were obtained
by sub-sampling 500 tips each. We also applied the method to a viral phylogeny
relating a cluster of 399 HIV infections in British Columbia.

On simulated data, we were able to estimate the PA power: the median [IQR]
absolute deviation of our estimate from the true value was 0.08 [0.03-0.18],
with 95\% confidence intervals of width 0.57 [0.38-0.73]. However, our estimates
of the number of infected nodes in these networks, using a uniform prior on
[500, 2000], were mostly uninformative (1130, 95\% CI [645-1761], true value
1000). The estimated PA power of the contact network for the HIV phylogeny in
British Columbia was 1.01 (0.94-1.05).

Using a novel kernel-ABC method to estimate contact network parameters from
virus phylogenies, we found that structural parameters such as PA power can be
recovered. Our method offers the potential to quantitatively investigate social
network dynamics underlying the spread of viral epidemics.


\section{Background}

The domain of this work is the emerging field of
phylodynamics~\autocite{grenfell2004unifying}. Briefly, phylodynamics is the
study of the interaction between epidemiological and evolutionary processes, in
organisms such as RNA viruses where these occur on the same time
scale~\autocite{drummond2003measurably}. In particular, phylodynamic methods
provide a way to use evolutionary data, like phylogenies, to estimate
epidemiological quantities, like transmission rates.

One type of epidemiological data we might want to investigate with
phylodynamics is the structure of the host population. This includes questions
such as how many connections a person has on average, or whether there are
``superspreaders'' responsible for a disproportionately high number of
transmissions. A contact network is a graphical representation of a population
of individuals (nodes) and potential transmissions between them (edges).
Contact networks models remove the assumption of homogeneously mixed
populations used in many epidemiological investigations, and can potentially
offer much more detailed information about the structure of a population
experiencing an epidemic. However, contact network models can be challenging to
work with. Unlike standard compartmental models, one cannot generally write
down a set of equations describing the epidemic behaviour over time, nor
calculate the likelihood of observed data for a given contact network.

Approximate Bayesian computation (ABC) is a statistical method for estimating
parameters of models for which explicit likelihoods cannot be calculated. The
idea of ABC is to simulate data according to a broad range of possible model
parameters, and prefer parameters which produce simulated data similar to the
observed data. Contact network models are a good candidate for the use of ABC,
since simulating networks, and transmission trees over those networks, is
straightforward.

\section{Objective}

The goal of this work is to develop and validate a method to quantitatively
infer structural parameters of contact networks from observed transmission
trees. This idea fits in between three established bodies of work. First, there
are phylodynamic studies which estimate epidemiological parameters, such as the
transmission rate or basic reproductive number
(e.g.~\autocite{stadler2011estimating, volz2012simple}). These studies have
provided valuable insight into emerging epidemics, however they typically
either ignored any population structure or assumed the population was
panmictic. Second, there have been several \emph{qualitative} investigations of
the impact of contact network structure on phylogeny shape
(e.g.~\autocite{leventhal2012inferring, colijn2014phylogenetic, o2010contact}).
Third, there have been a couple of attempts to discern contact network
parameters from other epidemiological data, like infection and recovery
times~\autocite{britton2002bayesian, groendyke2011bayesian}, but these have
been limited to the edge density parameter $p$ in networks with randomly chosen
edges. So far as we are aware, there have been no attempts to
\emph{quantitatively} estimate contact network parameters other than $p$, and
no attempts at all to estimate these parameters directly from viral sequence
data.

\section{Methods}

\subsection{Programs}

Three software programs were written for this work: \sw{nettree},
\sw{treekernel}, and \sw{netabc}. \sw{Nettree} simulates a transmission tree
over a supplied contact network using Gillespie
simulation~\autocite{gillespie1976general}. \sw{Treekernel} is a fast
implementation of the phylogenetic kernel of Poon
\etal~\autocite{poon2013mapping} using the algorithm of
Moschitti~\autocite{moschitti2006making}. \sw{Netabc} is an implementation of
the adaptive sequential Monte-Carlo algorithm developed by Del Moral
\etal~\autocite{del2012adaptive}, tailored for approximate Bayesian computation
using the phylogenetic kernel (kernel-ABC).

\subsection{Simulation experiments}

I performed three sets of simulation experiments to validate the kernel-ABC
method. The first two sets have either already been performed, or will be in
the near future, for all the parameters of three contact network models:
Erd\H{o}s-R\`enyi or random networks~\autocite{erdos1960evolution},
Watts-Strogatz or small-world networks~\autocite{watts1998collective}, and
Barab\'asi-Albert or preferential attachment
networks~\autocite{barabasi1999emergence}. However, due to computational
constraints, I have so far only performed the third set of experiments for the
preferential attachment power parameter (henceforth $\alpha$) of the
Barab\'asi-Albert network model. Therefore, I will focus only on this parameter
for this meeting. $\alpha$ indicates the strength of attraction to popular
nodes: as $\alpha$ increases, so does the tendency of other nodes to attach
themselves to nodes of high degree.

The first experiment was similar to the validation performed previously by
Poon~\autocite{poon2015phylodynamic}. It is designed to ensure that the chosen
parameters are identifiable, in that they produce transmission trees which are
different enough from each other that the parameter can be recovered. This
analysis also serves the purpose of checking that the tree kernel is the most
effective distance measure to use for ABC, and of identifying the optimal
meta-parameters for tuning the kernel. I simulated 100 trees each under three
different values of $\alpha$ (0.5, 1.0, and 1.5), constructed a pairwise
distance matrix relating the 300 trees, and estimated the accuracy of a
kernel-SVM classifier trained with this matrix by cross-validation. This was
repeated for several different sets of simulation and tree kernel parameters.

The second experiment was a marginal grid search, designed to investigate our
ability to infer a particular model parameter when all others are fixed. For
each of nine different $\alpha$ values (0.0, 0.25, \ldots, 2.0), ten testing
trees were simulated. These were compared using the tree kernel to 6030
training trees, which had been simulated on a finely spaced grid of $\alpha$
values between 0 and 2. The $\alpha$ value with the highest median tree kernel
score was taken as the point estimate of $\alpha$ for the testing tree. These
point estimates were compared to the true values.

The third and final experiment tested the ability of the full ABC algorithm to
infer the $\alpha$ parameter. Ten transmission trees were simulated under each
of the same nine $\alpha$ values as in the previous experiment. The
simulations were performed on networks of 5000 nodes, 1000 of which eventually
became infected. We assumed a uniform prior on $[1000, 10000]$ for the number
of nodes in the network, on $[500, 2000]$ for the number of infected nodes, and
$[0, 2]$ for $\alpha$. The mean degree of the network was fixed at either 4,
10, or 16. In each case, we evaluated the accuracy and precision of the
posterior distributions on $\alpha$, as well as those on the number of nodes
and number of infected.

\subsection{Real world application}

I applied the ABC algorithm to a phylogeny relating 399 HIV infections in
British Columbia, which had previously been identified as being part of a
phylogenetic cluster. The priors used were the same as in the previous section.
Because this application involved only one replicate, I ran the ABC algorithm
with a ten-fold higher number of particles, to increase the resolution and
hopefully obtain a narrower confidence interval.

\section{Results}

\subsection{Simulation experiments}

The first validation experiment showed that trees simulated under different
values of the $\alpha$ parameter were sufficiently distinct to be identifiable
with the tree kernel. With suitable kernel meta-parameters, the $R^2$ of the
kernel-SVM classifier was above 0.95 for 1000- and 500- tip trees, dropping to
0.85 for 100-tip trees. Surprisingly, multiplication of the tree kernel by the
normalized lineages-through-time statistic
(see~\autocite{janzen2015approximate}) actually reduced accuracy somewhat. This
prompted us to perform a sanity check by doing the same type of experiment for
the number of infected nodes (henceforth $I$) under a full network model, where
every node is connected to every other node. The full graph satisfies the
assumption of panmixia required for Kingman's coalescent, so the
lineages-through-time statistic should be informative. We found that under the
full model, the $R^2$ did improve slightly with the addition of the
lineages-through-time statistic.

In the second experiment using grid search, we found that the point estimates
for $\alpha$ were quite accurate. The least accurate estimates were for the
true value of $\alpha = 0$, which had a mean of 0.17 and a standard error of
0.05 across replicates. The most accurate were for the true value of $\alpha =
1.25$, which had a mean of 1.25 with standard error 0.01. 

Finally, in the third experiment using the full ABC algorithm, we were able to
recover the $\alpha$ parameter well in most cases, although the accuracy was
lower than with grid search. This was expected, since we were not fixing the
other model parameters. The least accurate point estimates were again for
$\alpha = 0$, with mean 0.31 and standard error 0.03. The most accurate were
for $\alpha = 1.25$, with mean 1.2 and standard error 0.01. The 95\% confidence
intervals on the parameters were fairly wide, averaging 0.64 across all
experiments. The estimates of the number of nodes and number of infected nodes
were mostly uninformative, with 95\% confidence intervals spanning almost the
entire prior region (average width 7313 and 1033 respectively, for prior
regions [1000, 10000] and [500, 2000]).

\subsection{Real world application}

We estimated the preferential attachment power of the real HIV network to be
1.01, with a 95\% confidence interval of [0.94, 1.05]. The reduced size of the
confidence interval compared to our simulations was likely due to the increased
number of particles in the ABC algorithm. As before, the estimates of numbers
of total nodes and infected nodes were uninformative, however the estimate of
$\alpha$ was robust to changing the mean degree of nodes in the network.

\section{Future work}

Two major tasks remain. First, I plan to perform the ABC validation experiments
for other parameters and network model types, namely random
networks~\autocite{erdos1960evolution} and small-world
networks~\autocite{watts1998collective}. This is a straightforward modification
of the existing experimental setup for $\alpha$, although the actual experiment
will take significant time. Second, I would like to find additional real-world
datasets to which to apply the method. Currently, I am running the algorithm on
a dataset from the CRF07 subtype of HIV in China, and am curating additional
datasets from Genbank.

\singlespacing
\printbibliography

\end{document}
