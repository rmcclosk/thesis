\documentclass{beamer}
\usepackage{tikz}
\usepackage{tikz-qtree}
\usepackage{gillius}
\usepackage{abraces}
\usepackage{etex}
\usepackage{array}
\usepackage{multirow}
\usepackage{qrcode}
\usepackage[backend=biber]{biblatex}

% http://tex.stackexchange.com/questions/12703/how-to-create-fixed-width-table-columns-with-text-raggedright-centered-raggedlef
\newcolumntype{L}[1]{>{\raggedright\let\newline\\\arraybackslash\hspace{0pt}}m{#1}}
\newcolumntype{C}[1]{>{\centering\let\newline\\\arraybackslash\hspace{0pt}}m{#1}}
\newcolumntype{R}[1]{>{\raggedleft\let\newline\\\arraybackslash\hspace{0pt}}m{#1}}

\usetikzlibrary{external}
\usetikzlibrary{shapes}
\usetikzlibrary{arrows}
\usetikzlibrary{positioning}
\usetikzlibrary{decorations.pathreplacing}

\usetheme[everytitleformat=regular]{m}
\setbeamertemplate{navigation symbols}{}

\graphicspath{{../figures/}}
\newcommand{\tablepath}{../tables}

\title{Phylogenetic inference of contact network parameters with kernel approximate Bayesian computation}
\author[RMM, RHL \& AFYP]{Rosemary M McCloskey$^1$ \and Richard H Liang$^1$ \and Art FY Poon$^{1,2}$}
\institute[UBC \& BCCfE]{$^1$BC Centre for Excellence in HIV/AIDS, Vancouver, Canada \\ $^2$Department of Medicine, University of British Columbia, Vancouver, Canada}
\date{HIV Dynamics \& Evolution, Woods Hole, USA, April 25, 2016}

\newcommand{\dd}[2]{\frac{\text{d}\,#1}{\text{d}\,#2}}

\addbibresource{papers.bib}

\begin{document}
\setbeamercolor{background canvas}{bg=white}
\renewcommand{\footnotesize}{\tiny}
\definecolor{red}{RGB}{228,26,28}
\definecolor{blue}{RGB}{55,126,184}
\definecolor{green}{RGB}{77,175,74}
\definecolor{purple}{RGB}{152,78,163}

\maketitle

\begin{frame}{Most epidemiological models assume homogeneous mixing}
  \begin{columns}
    \begin{column}{0.6\textwidth}
      \centering
      \includegraphics{compartments}
      \vspace{1cm}

      \uncover<2->{\includegraphics[width=\textwidth]{sir-trajectories}}
    \end{column}
    \begin{column}{0.4\textwidth}
      \vspace{-2cm}
      \begin{itemize}
        \setlength{\itemsep}{24pt}
        \item Often provides reasonable approximation in practice.
        \uncover<2->{
        \item Can be inaccurate when substantial contact heterogeneity exists.
        }
      \end{itemize}
    \end{column}
  \end{columns}
\end{frame}

\begin{frame}{Network models capture contact heterogeneity}
  \begin{columns}
    \begin{column}{0.5\textwidth}
      \includegraphics[width=\textwidth]{contactnet-empty}
    \end{column}
    \begin{column}{0.5\textwidth}
      \begin{itemize}
        \setlength{\itemsep}{12pt}
          \pause
        \item May offer more accurate predictions for highly structured
          populations.
          \pause
        \item Network parameters may be of interest for their own sake,
          \textit{e.g.} are there superspreaders?
          \pause
        \item Extremely difficult to estimate in practice.
      \end{itemize}
    \end{column}
  \end{columns}

\end{frame}

\begin{frame}{Contact networks shape transmission trees}
  \includegraphics[width=\textwidth]{contactnet}

  \begin{itemize}
      \pause
    \item Transmission trees in turn shape viral phylogenies.
      \pause
    \item \textbf{Aim:} estimate contact network parameters from viral
      phylogenies.
  \end{itemize}
\end{frame}

\begin{frame}{Scale-free networks and preferential attachment}
  \includegraphics[width=\textwidth]{pa-example}
  \begin{itemize}
    \item Sexual networks are often scale-free: degree distributions appear
      follow a power law ($\Pr(\text{degree} = d) \propto d^{\gamma}$).
      \pause
    \item Preferential attachment models generate scale-free networks.
  \end{itemize}
\end{frame}

\begin{frame}{Barab\'asi-Albert (BA) network model}
  \begin{columns}
    \begin{column}{0.5\textwidth}
      \begin{minipage}[p][\textheight][t]{\textwidth}
        \centering
        \only<1>{\includegraphics[page=1, width=\textwidth]{pa.pdf}}
        \only<2>{\includegraphics[page=2, width=\textwidth]{pa.pdf}}
        \only<3>{\includegraphics[page=3, width=\textwidth]{pa.pdf}}
        \only<4>{\includegraphics[page=4, width=\textwidth]{pa.pdf}}
        \only<5>{\includegraphics[page=5, width=\textwidth]{pa.pdf}}
        \only<6>{\includegraphics[page=6, width=\textwidth]{pa.pdf}}
      \end{minipage}
    \end{column}
    \begin{column}{0.5\textwidth}
      \begin{minipage}[t][\textheight][t]{\textwidth}
      \begin{itemize}
        \uncover<1->{ \item Start with a small number of connected nodes. }
        \uncover<2->{ \item Attach new nodes with $m$ edges. }
        \uncover<3->{ \item Other endpoints of degree $d$ are chosen with probability $\propto d^{\alpha} + 1$. }
        \uncover<5->{ \item Continue until network has $N$ nodes. }
        \uncover<6->{ \item Also consider the prevalence $I$ when the transmission
          tree is sampled.}
      \end{itemize}
      \end{minipage}
    \end{column}
  \end{columns}
\end{frame}

\begin{frame}{Do network parameters measurably affect tree shape?}
  \only<1>{
    \includegraphics[height=1in, trim=0 2in 4in 0, clip]{kernel-idea}
  }
  \only<2>{
    \includegraphics[height=1in, trim=0 2in 2in 0, clip]{kernel-idea}
  }
  \only<3->{
    \includegraphics[height=1in, trim=0 2in 0 0, clip]{kernel-idea}
  }
  \begin{itemize}
    \item Generate networks under different parameter values ($N$, $m$, $\alpha$).
    \uncover<2->{
    \item Simulate epidemic over each network with Gillespie algorithm ($I$).
    }
    \uncover<3->{
    \item Randomly subsample to form transmission trees.
    }
    \uncover<4->{
    \item Compare pairwise using tree kernel\footfullcite{poon2013mapping}.
    }
  \end{itemize}
\end{frame}

\begin{frame}{$\alpha$ and $I$ separate well in feature space}
  \begin{tikzpicture}
    \node {\includegraphics[height=0.8\textheight]{kernel-kpca.pdf}};
    \fill<1>[white] (-4.6,-3.2) rectangle (5,0.2);
  \end{tikzpicture}
  \pause
\end{frame}

\begin{frame}{Classifiers for network parameters}
  \vspace{-1cm}
  \begin{itemize}
    \item kSVM using tree kernel
    \item SVM using normalized
      lineages-through-time\footfullcite{janzen2015approximate}
    \item linear regression using Sackin's index~\footfullcite{shao1990tree}
  \end{itemize}

  \pause
  \includegraphics[width=\textwidth]{kernel-rsquared.pdf}
\end{frame}

\begin{frame}{Approximate Bayesian computation (ABC) for networks}
  \begin{minipage}[p][\textheight][t]{\textwidth}
    \only<1>{\includegraphics[width=\textwidth, trim=0 3in 0 0, clip]{abc-idea.pdf}}
    \only<2>{\includegraphics[width=\textwidth, trim=0 1.8in 0 0, clip]{abc-idea.pdf}}
    \only<3>{\includegraphics[width=\textwidth, trim=0 1in 0 0, clip]{abc-idea.pdf}}
    \only<4>{\includegraphics[width=\textwidth, trim=0 0in 0 0, clip]{abc-idea.pdf}}
  \end{minipage}
\end{frame}

\begin{frame}{Sequential Monte Carlo for ABC}
  \begin{tikzpicture}
    \only<1-2>{\node at (0, 0) [anchor=north west] {\includegraphics[height=0.7\textheight, trim=0 0in 4in 0, clip]{abc-smc.pdf}}};
    \only<1>{\fill[white] (0.2,-6.5) rectangle (4,-3.5)};
    \only<3>{\node at (0, 0) [anchor=north west] {\includegraphics[height=0.7\textheight, trim=0 0in 2in 0, clip]{abc-smc.pdf}}};
    \only<4>{\node at (0, 0) [anchor=north west] {\includegraphics[height=0.7\textheight, trim=0 0in 0 0, clip]{abc-smc.pdf}}};
  \end{tikzpicture}
\end{frame}

\begin{frame}{Marginal posterior distributions for one simulated tree}
  \begin{minipage}[p][\textheight][t]{\textwidth}
    \only<1>{\centerline{\includegraphics[width=0.9\textwidth, trim=0 2.5in 0 0, clip]{abc-posterior-example.pdf}}}
    \only<2>{\centerline{\includegraphics[width=0.9\textwidth]{abc-posterior-example.pdf}}}
  \end{minipage}
\end{frame}

\begin{frame}{Accuracy of maximum \textit{a posteriori} estimates for $m = 2$}
  \begin{minipage}[p][\textheight][t]{\textwidth}
    \vspace{-0.5cm}
    \only<1>{\includegraphics[height=0.9\textheight, trim=0 0 3.4in 0, clip]{abc-point-estimate-m2.pdf}}
    \only<2>{\includegraphics[height=0.9\textheight]{abc-point-estimate-m2.pdf}}
  \end{minipage}
\end{frame}

\begin{frame}{Application to published HIV datasets}
  \uncover<2->{
  \includegraphics[width=\textwidth]{realdata-hpd-bc-horiz}
  }
  \vspace{0.5cm}

  \scriptsize
  \begin{tabular}{ccccc}
  Reference & Sequences ($n$) & Location & Risk group & Gene \\
  \hline
  \autocite{wang2015targeting} & 173 & Beijing, China & MSM & \textit{pol} \\
  \autocite{cuevas2009hiv} & 287 & Basque Country, Spain & mixed & \textit{pol} \\
  \autocite{novitsky2013phylogenetic} & \multirow{2}{*}{180} &
  \multirow{2}{*}{Mochudi, Botswana} & \multirow{2}{*}{mixed} &
  \multirow{2}{*}{\textit{env}} \\ \autocite{novitsky2014impact} \\
  \autocite{li2015hiv} & 280 & Shanghai, China & MSM & \textit{pol} \\
  \autocite{niculescu2015recent} & 136 & Romaina & IDU & \textit{pol} \\
  N/A & 399 & British Columbia, Canada & IDU & \textit{pol} \\
  \hline
\end{tabular}

  \normalsize
\end{frame}

\begin{frame}{Conclusions}
  \begin{itemize}
    \item Kernel-ABC is a viable method for fitting contact network models to
      phylogenetic data.
      \pause
    \item Some parameters of the Barab\'asi-Albert preferential attachment data
      have a measurable impact on tree shape and can be re-estimated using
      kernel-ABC.
      \pause
    \item The networks underlying real epidemics may be heterogeneous,
      underscoring the importance of considering network structure in
      phylodynamic analyses.
      \pause
  \end{itemize}
  \centerline{\qrcode{github.com/rmcclosk/netabc} github.com/rmcclosk/netabc}
\end{frame}

\begin{frame}{Acknowledgements}
  \begin{columns}
    \begin{column}{0.6\textwidth}

      \textbf{BC Centre for Excellence in HIV/AIDS}

      Art Poon

      Jeff Joy

      Richard Liang

      Thuy Nguyen

      P. Richard Harrigan

      \hfill\\
      \textbf{University of British Columbia}

      Sarah Otto

      Alexandre Bouchard-C\^ot\'e

      \vfill
    \end{column}
    \begin{column}{0.4\textwidth}
      \centering 

      \includegraphics[width=3cm]{logos/cihr}
      \vspace{0.5cm}

      \includegraphics[width=3cm]{logos/bmgf}
      \vspace{0.5cm}

      \includegraphics[width=1cm]{logos/ubc}
      \vspace{0.5cm}

      \includegraphics[width=3cm]{logos/btp}
    \end{column}
  \end{columns}
\end{frame}
\end{document}
