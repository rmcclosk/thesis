In rapidly evolving organisms such as RNA viruses, evolutionary and
epidemiological dynamics occur on the same time scale. Phylodynamics attempt to
detect and quantify the signatures of epidemiological processes in viral
phylogenies. Due to the rapid advancement of nucleotide sequencing technology,
viral sequence data data have become increasingly feasible to collect on a
population level. Through phylodynamic methods, these data offer a window into
epidemiological processes which would otherwise be virtually impossible to
study on a realistic scale. However, the vast majority of phylodynamic methods
assume a homogeneously mixed host population, where every pair of individuals
is equally likely to come into contact. This assumption is clearly unrealistic
for most human communities.

We developed a method based on kernel approximate Bayesian computation to fit
contact network models to viral phylogenies without calculating intractible
likelihoods. Our method can be used to fit any network model from which
simulated networks can be generated. We applied our method to investigate a
preferential attachment model using simulated and real-world HIV sequence
datasets. On simulated data, we found that the preferential attachment power
and the number of infected nodes could often be accurately estimated, while the
edge density and total number of nodes were more challenging to determine. We
found significant heterogeneity in the networks underlying several real-world
datasets, with estimated preferential attachment power ranging from 0.35 to
1.16. These results underscore the importance of taking contact structure into
account when investigating viral epidemics, and suggest kernel-ABC as an
effective tool for such investigations.

