\documentclass[12pt]{letter}
\usepackage[T1]{fontenc}
\usepackage{mathptmx}
\usepackage[garamondx,cmbraces]{newtxmath}

\signature{Rosemary M. McCloskey}
\address{Centre for Excellence in HIV/AIDS \\ 608-1081 Burrard Street \\ Vancouver, BC, Canada V6Z 1Y6}
\begin{document}
\begin{letter}{ }

\opening{Dear editors of \textit{Virus Evolution},}

Thank you for the opportunity to revise our manuscript ``Reconstructing contact
network parameters from viral phylogenies''. While our submission was pending,
we made some additions and improvements to the work that we would like you and
the reviewers to consider. 

\begin{itemize}

  \item It was pointed out to us by a colleague that the posterior mean is a
    more conventionally used Bayesian estimator than the maximum \textit{a
    posteriori} (MAP) value. Therefore, we have replaced the MAP point
    estimates by posterior means. Qualitatively, the results are not much
    different, except that it is more obvious that $m$ and $N$ are not
    identifiable with this method. 
  \item We have replaced some ad hoc statistical tests in the analysis of the
    simulation results with a $p$-value corrected generalized linear model, and
    rewritten the relevant section of the results accordingly.
  \item We have found an analysis of the Barab\'asi-Albert model in the physics
    literature (Krapivsky et al. 2000) which demonstrates that the degree
    distribution is only a power law when $\alpha = 1$. We acknowledge this but
    do not believe it detracts from the validity of our results, especially in
    light of a paper by de Blasio et al. (2007) which found sub-linear
    preferential attachment through a population-level survey. We have added to
    the supplemental materials a series of power law fits to the degree
    distributions of BA networks with $\alpha \neq 1$ (Figure TODO), which
    shows that, although the power law is cleary an imperfect fit, the slope
    is still reasonably representative of the true distribution.
  \item We did some additional literature searching and found five more
    published HIV datasets that seemed to be appropriate for our method. For
    clarity, the real HIV datasets are now referred to by risk group and
    geographic location, rather than by author (eg. MSM/Beijing instead of Wang
    et al.). The highest $\alpha$ values were associated with IDU networks,
    coroborating the assertion of Dombrowski et al. (2013) that IDU networks
    exhibit the scale-free property more strongly than sexual networks do. 
  \item Three of these new datasets had sequenced two HIV genes, which we
    analysed separately. We found that estimates of $\alpha$ were very robust
    to the choice of gene; $I$ was less robust, with \textit{env} genes tending
    to lead to higher estimated values.
  \item Figure 5 (formerly Figure 4) now shows 50\% HPD intervals (similar to
    interquartile ranges) in addition to 95\% HPD intervals.
  \item We discovered that our use of the term ``kernel-ABC'' to refer to ABC 
    with a kernel for the distance function is in conflict with another usage,
    namely an expression of Bayes' rule in terms of kernel means (Fukumizu et
    al. ``Kernel Bayes' rule.'' 2011). We therefore simply use the term ``ABC''
    to refer to our method.
  \item The discussion has been divided into subsections to make it easier to
    follow.
\end{itemize}

Our responses to each of the reviewers' comments are in line below. We were
asked by the Editor in Chief to consider how to disambiguate ABC approximation
error from a truly noisy posterior, which is addressed under the first point
brought up by Reviewer 1.

\textbf{Reviewer 1}

\begin{quote}
  \itshape

  I think the question of how we might learn about contact network structure
  from trees -- transmission or phylogenetic -- is interesting, and it is
  refreshing to see a non-BEAST approach to a phylodynamic question (nice
  though BEAST is, one of several limitations is its unsuitability to large
  datasets, for example). \\

  Overall, though, it seems that the conclusion is that in this simulation
  study, transmission trees from the BA network in these conditions don't carry
  sufficient information about the network parameters to be very informative.
  Alpha is most robustly identified, perhaps because of the dramatic effect of
  alpha on gamma (fig S6) so that alpha < ~1.2 would have a very different
  gamma than alpha > ~1.2. The HIV datasets yielded wide and somewhat varying
  estimates. The authors concluded something about contact heterogeneity, but
  is it robust inference of heterogeneous contact, or just poor posterior
  resolution?
\end{quote}

We are grateful that the reviewer finds our work to be of some interest, and we
acknowledge that our ability to precisely resolve numerical values of the BA
parameters is limited. We believe that our simulation results show that some
network parameters (specifically $\alpha$ in our case) can be at least roughly
estimated from the information contained in tree shapes.

We have included several new analyses that attempt to disambiguate true
features of the posterior distribution from artefacts of a poor ABC
approximation. First, we used ABC to estimate marginal posterior distributions
when some combinations of parameters were held fixed to known values.
Estimation error which is removed during marginal estimation is likely to be
due to confounding between model parameters; error which remains may be due to
either poor resolution or truely diffuse posteriors. The accuracy of the
estimates was improved about 50\% in marginal estimation.

Second, we examined the possible Monte Carlo error by increasing the resolution
of the ABC approximation (more particles, more simulated datasets per particle,
and more iterations). The results were similar; surprisingly, increased
settings made the estimation about 10\% worse. We suspect this may be simply
due to stochastic effects. In any event, this result demonstrates that Monte
Carlo error from the SMC algorithm is likely not responsible for the observed
estimation errors.

Finally, three of the newly included HIV datasets included both \textit{gag}
and \textit{env} sequences which we analysed separately. The estimates of
$\alpha$ for the two genes were nearly identical for each of the datasets,
which seems unlikely if the ABC approximations were very poor.

\begin{quote}
  \itshape

  I appreciate the detail and clarity in the Methods section. One concern I had
  was the use of $R^2$. As many quantities can be called $R^2$, and you have
  categorical data, how have you defined it, and why is it reflective of
  goodness of fit? Usually an $R^2$ (linear model?) is not considered the best
  way to test categorical relationships. How much does the linear assumption
  behind $R^2$ impact the quality of the reported results?
\end{quote}

We apologise for this misunderstanding due to our poor choice of words. An SVR
is in fact a regressor, not a classifier, so the reported $R^2$ were of true
values against predictions. However, given that the point of this experiment
was to demonstrate identifiability by categorizing tree shapes into
predetermined values, we redid the analysis using SVMs (which are true
classifiers) instead of SVRs and linear regression. We now report the accuracy
of the categorical classification, and figures 1 and S3-S6 have been revised
accordingly. Some additional variation among meta-parameters and epidemic
scenarios is observed, but the overall results are similar and the best choice
of meta-parameters is the same.

\begin{quote}
  \itshape

  Could you describe how the cross-validation was done (numbers training vs
  test etc)?
\end{quote}

We have updated the text as follows (addition in italics): ``The accuracy of
each classifier was evaluated with 1000 two-fold cross validations \emph{with
equally-sized folds}.''

\begin{quote}
  \itshape

  Please describe how you extract the transmission trees -- are they binary
  trees, with individual transmission events resolved in real time? Or are they
  not binary trees, eg case 1 infecting cases 2, 3, 4, 5 corresponding to an
  internal node with degree 5 (in-degree 1, out-degree 4)? If the latter, it's
  not really comparable to using HIV phylogenies as these will be binary.
\end{quote}

The method we used to estimate transmission trees simply reroots and adjusts
the branch lengths of an estimated viral phylogeny, which is always binary as
long as there are no duplicated sequences. We have added the following
clarifications in the methods section: (1) ``Duplicated sequences were removed
with BioPython'' (2) ``Due to the removal of duplicated sequences, all
estimated transmission trees were fully binary.''

\begin{quote}
  \itshape

  Is inferring m any easier for fixed alpha, N, I than varying? You note in the
  Discussion that inferring N is optimistic (and this seems particularly true
  if you stop the simulation when either I or the target number of tips is
  reached), and you briefly comment about the relationships between the
  variables. You might be able to get stronger inference if some parameters
  were fixed. Anyway it would be nice to see some joint posteriors with two
  parameters, to see if some of the posterior uncertainty is due to
  correlations.
\end{quote}

We thank the reviewer for this helpful suggestion, which we addressed by
marginal estimation as outlined under the first comment. The inference of $I$
was dramatically improved when $N$ was fixed. Fixing each of the parameters
contributed slightly to the estimate of $\alpha$, but none as dramatically.
However, the posterior mean estimate of $m$ was totally unaffected by fixing
the other parameters. This indicates that our inability to estimate $m$ is
simply due to lack of identifiability, rather than to parameter correlations. 

As suggested, we have also included a set of two-dimensional posterior
distributions in the text (Figure 4). The correlation between $N$ and $I$
demonstrated in the marginal analysis is clearly observable, but there are no
obvious correlations among the other parameters.

\begin{quote}
  \itshape

  You found that m is hard to infer. Could this be because with pre-determined
  I, even if m increases, only so many edges per host will actually be active
  in transmitting infection? Overall I would have been interested in more
  intuitive-level discussion of why the results were what they were.
\end{quote}

We have substantially expanded the discussion of the synthetic data results,
including this point about $m$. The weaker identifiability in lower $\alpha$
values is related to the degree distributions of the networks, and the upward
biases in $I$ and $N$ are discussed in relation to the epidemic trajectory of
the SI model. Our intuition about $m$ is weaker, but we include this point by
the reviewer and hypothesize that $m$ may be more readily estimable from the
speed of epidemic expansion rather than the shape of the transmission tree.

\begin{quote}
  \itshape

  You qualitatively describe errors, and show lots of posteriors, but a direct
  visual quantification of how well the mean / mode /etc of the posterior
  reflects the true value would be helpful - for example as a box plot, where
  you could then illustrate all the tests in the same figure, stratified as you
  like.
\end{quote}

We agree with this assessment of our visual presentation of the results. We
have reduced the number of posterior plots shown to one in the text and two in
the supplemental materials. Figure 2, which formerly represented each
simulation by an individual point, has been replaced by a box plot showing the
$\alpha$ and $I$ estimates stratified by each of the other parameters. A
similar boxplot for $m$ and $N$, which were not identifiable with ABC, has been
added to the supplemental materials.

\begin{quote}
  \itshape

  The Discussion focusses around model mis-specification, which is very
  relevant (and to which the inference isn't that robust, suggesting that it
  might also not be robust to other things like dynamics, clustering,
  age-related mixing etc). But another question is how much of the signal you
  would expect to find in this simulation strategy, and how you would expect to
  see it using these kinds of tree comparisons. I don't have much intuition
  about the kernel but the Sackin imbalance will presumably be lower where
  there are very high-degree nodes, and higher where there are long
  transmission chains. The nLTT will depend on the epidemic curves, presumably.
\end{quote}

The identifiability of the BA parameters is of course a critical prerequisite
to the kind of inference we are trying to perform. Indeed, the objective of the 
analysis with classifiers was to investigate identifiability, and we apologise
for not making this clearer. The reviewer's intuition is correct -- we do
expect imbalance to be sensitive to the degree distribution (related to
$\alpha$) and the nLTT to be sensitive to the epidemic curves (related to $I$
and $N$). This is reflected in the performance of the respective classifiers:
Sackin's index does better than the nLTT at classifying $\alpha$, but the nLTT
is better at classifying $I$ and $N$.

We have included some expository text near the start of the methods section to
emphasize these points. In particular, ``Sackin's index (Shao 1990)
is a measure of tree imbalance which not take branch lengths into account,
considering only the topology. The normalized lineages-through-time (nLTT,
Janzen et al. 2015) compares two trees based on normalized distributions of
their branching times, and does not explicitly consider the topology. Since the
tree kernel incorporates both of these sources of information, we expected it
to outperform the other two statistics.''

\begin{quote}
  \itshape

  ABC is computationally intensive, as you note, but does it perform better
  than classification? Enough better to be worth it?
\end{quote}

Our classification experiments were marginal -- they only estimated one
parameter at a time, when all others were fixed and known. In this situation,
something like k-nearest-neighbours classification or grid search might be just
as effective as ABC, and will certainly be faster. The justification for our
method comes in overcoming the curse of dimensionality when jointly estimating
multiple parameters of a network model. The text added to address the previous
point makes it clearer that the classifier experiments were marginal in nature.
(TODO: We have emphasized this point in the discussion as well: ).

\textbf{Reviewer 2}

\begin{quote}
  \itshape
  The paper describes a study using ABC to estimate the parameters of contact
  networks from transmission trees, using phylogenies as a proxy for the
  latter. The paper is well-written and the method and results are clearly
  presented. While it must be said that the method as presented appears to lack
  accuracy in the reconstruction of many features of an epidemic, the authors
  do not pretend otherwise. I regard this as an important class of models whose
  development should be encouraged, and this is one of the first few such
  examples. As a result, I am happy to recommend publication with some
  revisions. \\

  General comment: This paper was clearly written with results coming before
  methods and wasn't adequately reformatted; various mathematical symbols are
  mentioned in the methods but only defined later. As a result, both methods
  and results sections need another look.
\end{quote}

We are quite embarrassed by this oversight and have revised the text
accordingly. In particular, the first three paragraphs of the methods section
define the BA parameters ($\alpha$, $I$, $m$, and $N$) and introduce the
meta-parameters to the tree kernel.

\begin{quote}
  \itshape

  P2 \\
  L19: This is the effective size of the viral population, so best to specify
  that explicitly given work on the ``effective number of infections''.
\end{quote}

We have changed the phrase to ``effective viral population size.''

\begin{quote}
  \itshape

  L25-26: ``Straightforward to collect'' is a bit of a strong statement. While
  sequencing is getting easier, representative sample collection may still be
  challenging.
\end{quote}

We have revised the statement as follows: ``Viral sequence data, on the other
hand, has become easier to collect as the cost of sequencing has declined. In
the case of HIV, genotyping has become part of routine clinical care in several
health regions.'' We have also added the following qualification to the end of
the paragraph: ``In the context of networks, sequence data have the advantage
of being objective, in that they are not affected by misreporting. However,
just as with survey data, it is important to collect a representative sample
from the population to perform accurate inference (Novitsky et al. 2014).''

\begin{quote}
  \itshape

  P4\\
  L53-54: The y-axis does not match closely with the Leventhal figure; is this
  an error or genuine? If the latter, please comment.
\end{quote}

We thank the reviewer for their sharp eye. This was indeed an error on our
part. Leventhal et al. constructed Watts-Strogatz (``small world'') networks by
rewiring each edge with probability $p = 0.01$. However, in the implementation
we used, $p$ is the probability of rewiring each \emph{endpoint}, not each
edge. Therefore, when we took $p = 0.01$, the net effect was that each edge had
probability $2p = 0.02$ to be rewired (the sum of the probabilities of each
endpoint). We remade the figure with $p = 0.005$, which should match up with $p
= 0.01$ from Leventhal et al., and the scales are now the same. The lines
corresponding to the Erd\H{o}s-R\'enyi and Barab\'asi-Albert networks are
unchanged.

\begin{quote}
  \itshape

  P5 \\
  L45 How were the tips selected? I'm assuming randomly, but make this explicit.
\end{quote}

The reviewer's assumption is correct. We have clarified this in the section
describing \textit{netabc}, since it is an aspect of the method itself and
applies to all analyses: ``Tips of the simulated transmission tree are randomly
removed until the simulated tree has the same number of tips as the input
tree.''

\begin{quote}
  \itshape

  L45-46: “…and in epidemics of differing size (500, 1000 and 2000)”. Unless
  I’m missing something this is just a restatement of what’s already in this
  paragraph about I, hence repetitive. 
\end{quote}

We apologise for the misunderstanding. Because $I$ is not a structural
parameter of the network, we attempted to classify each of $\alpha$, $N$, and
$m$ under several different fixed values of $I$. We have rephrased this as
follows: ``For the structural parameters $\alpha$, $m$, and $N$, the
experiments were repeated with three different fixed values of $I$ (500, 1000,
and 2000).''

\begin{quote}
  \itshape

  P7 \\
  L20-24: It’s slightly surprising to me that the time trees were constructed
  using “fast” methods when the datasets are not especially large. Were there
  problems in obtaining trees with other algorithms? While I do not insist that
  every analysis be redone using alternative methods, it would be reassuring to
  know that the results were consistent, in at least one example, if
  reconstruction were done with, for example, RAxML+LSD, or by taking the BEAST
  MCC tree. Similarly, are results similar if analysis is performed on several
  bootstraps or posterior samples?
\end{quote}

Although they are very approximate methods, our experience has been that using
approximate maximum likelihood phylogeny reconstruction with root-to-tip
regression does not result in large errors in phylodynamic inference (Poon
2015). We reanalysed the datasets with the highest and lowest estimated
$\alpha$ (Zetterberg et al. 2004 and Cuevas et al. 2009) using a BEAST MCC tree
from a run with simple models of evolution and population size. The results
were TODO

With respect to the suggestion to analyse bootstrap replicates, we believe that
the three new datasets containing both \textit{gag} and \textit{env} data fill
the need for within-dataset replication.

\begin{quote}
  \itshape

  P8 \\
  L15: Higher values of sigma do seem to consistently lead to worse estimates
  of alpha, and estimates are better on smaller datasets. Is there an
  explanation for this?
\end{quote}

We think the reviewer is referring to Figure S3, which shows the
cross-validation accuracy of the SVM classifier for $\alpha$ under several
values of the tree kernel meta-parameters $\lambda$ and $\sigma$. However,
higher $\sigma$ values ($x$-axis) lead to better estimates, not worse, and for
most meta-parameter combinations the estimates get worse, not better, on
smaller datasets. 

In any case, we can offer a few comments on this figure. When $\sigma = 0$,
subset trees only match when their branch lengths are identical. As $\sigma \to
\infty$, branch lengths are not taken into account at all, and matches are
based on topology alone. Thus, it seems that topology is more important for
classifying $\alpha$ than branch lengths are, which is consistent with our
observation that $\alpha$ strongly affects tree balance. The decay factor
$\lambda$ is intended to penalize very large subset tree matches which can
dominate the kernel score (see Poon et al. 2013). This ``large match'' problem
becomes worse as the trees become larger. For the largest trees we considered, 
it seems that $\lambda = 0.4$ was not a strict enough penalty, leading to a
decrease in classification accuracy. 

Part of the new text introducing the classification experiments now contains a
short description of the two meta-parameters. In addition, a brief comment
on the effects of $\lambda$ and $\sigma$ on the classifiers has been added to
the first paragraph of the results section.

\begin{quote}
  \itshape

  P9 \\
  Figure 1: Could this be in colour rather than shades of grey?
\end{quote}

The figure is now in color.

\begin{quote}
  \itshape

  L32: “Estimated correctly” – i.e. the posterior mode was correct?
\end{quote}

The reviewer's interpretation is correct, but this statement has been removed
in our updated version. When we calculated posterior means rather than modes,
as is more conventional, we found that nearly all of the estimates were very
near to the middle of the range. This is consistent with our observed lack of
signal for $m$ in phylogenetic data.

\begin{quote}
  \itshape

  P10-11 \\
  I think I might prefer that a firm decision on whether it is preferable to
  allow m=1 be made and justified and the results of that analysis be presented
  first and foremost, with the alternative results also presented, but in a
  separate paragraph.
\end{quote}

Since several of the datasets had the majority of their posterior mass on $m =
1$, it seemed that this choice of prior was more reasonable, and we therefore
focused our discussion on those results. The alternative results are presented
in the last paragraph of the discussion.

\begin{quote}
  \itshape

  P12 \\
  Figure 2: Since there were three simulations done of each type, presumably
  the variable number of points in the top right figure is due to the existence
  of duplicate values. But this needs to be clearer. Also presumably the points
  from the same sort of simulation were jittered so they appear in discrete
  columns, but this makes the regression line perhaps a little misleading.
  Perhaps this figure could also be made a little larger?
\end{quote}

As per Reviewer 1's suggestion we have replaced Figure 2 with a boxplot showing
posterior means stratified by the true parameter values. We agree that the
sloped lines were misleading and have replaced them with horizontal line
segments. 

\begin{quote}
  \itshape

  P13 \\
  Figure 3, legend: I’m guessing that the shaded polygon is the HPD region,
  rather than the “true value”?
\end{quote}

This is correct and the caption has been updated as follows: ``shaded areas
show 95\% highest posterior density intervals.''

\begin{quote}
  \itshape

  P15 \\
  L38-40: While I understand the point, it’s surely quite unlikely that the
  network will be identical. Better to say that the number of infected nodes
  with new children is small, and hence there is little difference in the
  outcomes of the simulations.
\end{quote}

We struggled with the phrasing of this particular point and are happy to have
recieved some guidance. The statement now reads as follows: ``If $I$ is small
relative to $N$, very few of the infected nodes will gain any new neighbours.
Thus, the outcome of a second simulation on the same network will likely be
very similar.''

\begin{quote}
  \itshape

  L41-42: “We note also that our accurate estimates of I may have been
  influenced by this prior”. Couldn’t you check?
\end{quote}

This issue was also addressed by performing marginal estimation. If the
accurate estimates of $I$ were solely due to the prior, they would have gotten
worse, not better, when $N$ was fixed and the prior on $I$ was truly uniform.
We make this point in the discussion: ``when $I$ was estimated marginally with
fixed $N$, the accuarcy of the estimate improved even though there was no
longer any extra prior mass on low $I$ values.''

\begin{quote}
  \itshape

  P16 \\
  L9-27: As stated above, I'd prefer your preference for whether to include m=1
  to be stated and justified.
\end{quote}

We have focused our discussion on $m = 1$ due to the several datasets for which
$m = 1$ contained most of the posterior mass.

\begin{quote}
  \itshape

  P17
  L1: Presumably by estimated prevalence you mean just I/N?
\end{quote}

We had been using the term ``prevalence'' interchangably with $I$, but since
this is ambiguous we have replaced all instances of ``prevalence'' with $I$
except for those which do not refer to $I$ explicitly.

\begin{quote}
  \itshape

  A point I’d like to see at least discussed – presumably if some iterations
  add nodes with one connection and others add them with two, you’d get cycles,
  but not as many as you’d expect with m=2. Is this worth considering in the
  question of why the binary choice to include m=1 is so problematic? Would it
  be better if m was drawn from a discrete distribution at every step?
\end{quote}

This is a valid point, and in fact drawing $m$ from a distribution is an
approach we considered when we were performing this work. However, since this
is our first attempt to fit network models, we thought it best to stick with
a simple, well-known model rather than defining our own parameters. We did
perform a small experiment allowing heterogeneity among nodes, but in $\alpha$
rather than in $m$, and recovered the average of the two $\alpha$ values.

(TODO: We
have added to the discussion the point that this binary choice is sub-optimal
and suggested this enhancement for future work.)

\begin{quote}
  Supplement:
  Figure S1 legend: What is the shaded area?
\end{quote}

The original figure in Leventhal et al. has two shaded regions, one
representing the interquartile range, and the other representing the 95\%
quantile range. We have edited the figure to contain both regions and
identified them in the figure legend.

\closing{Thank you for your consideration,}
\end{letter}
\end{document}
