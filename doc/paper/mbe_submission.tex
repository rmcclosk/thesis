\documentclass[12pt]{article}
\usepackage{fullpage}
\usepackage{garamondx}
\usepackage{parskip}
\usepackage{setspace}

% http://tex.stackexchange.com/questions/44618/dynamically-count-and-return-number-of-words-in-a-section
\newcommand\wordcount{\immediate\write18{texcount -sub=section \jobname.tex  | grep "Section" | sed -e 's/+.*//' | sed -n \thesection p > '/tmp/count.txt'}\ignorespaces\input{/tmp/count.txt}}

\title{Phylodynamic inference of contact network parameters through approximate Bayesian computation}
\author{Rosemary M. McCloskey, Richard H. Liang, and Art F.Y. Poon}

\doublespacing
\pagestyle{empty}
\begin{document}
\maketitle

\section{Abstract}

(\wordcount / 250 words)

Models of the spread of disease in a population often make the simplifying
assumption that the population is homogeneously mixed, or is divided into
homogeneously mixed compartments. However, human populations have complex
structures formed by social contacts, which can have a significant influence on
the rate of epidemic spread. Contact network models capture this structure by
explicitly representing each contact which could possibly lead to a
transmission. We developed a method based on kernel approximate Bayesian
computation (kernel-ABC) for estimating structural parameters of the contact
network underlying an observed viral phylogeny. The method combines adaptive
sequential Monte Carlo for ABC, Gillespie simulation for propagating epidemics
though networks, and a kernel-based tree similarity score. We used the method
to fit the Barab\'{a}si-Albert network model to simulated transmission trees,
and also applied it to viral phylogenies estimated from five published HIV
sequence datasets. On simulated data, we found that the preferential attachment
power and the number of infected nodes in the network can often be accurately
estimated. On the other hand, the mean degree of the network, as well as the
total number of nodes, were not estimable with kernel-ABC. We observed
substantial heterogeneity in the parameter estimates on real datasets, with
point estimates for the preferential attachment power ranging from 0.2 to 1.1.
These results underscore the importance of considering contact structures when
performing phylodynamic inference. Our method offers the potential to
quantitatively investigate the network structure underlying viral epidemics,
using models appropriate to the context being studied.

\section{Keywords}

Phylogenetics, phylodynamics, approximate Bayesian computation, contact network,
transmission tree, human immunodeficiency virus.

\section{Cover Letter}

Dear editors of \textit{Molecular Biology and Evolution},

On behalf of my coauthors and myself, I am submitting a manuscript entitled
``\textit{Phylodynamic inference of contact network parameters through
approximate Bayesian computation}'' for your consideration. 

In this work, we have developed a method to fit contact network models to viral
phylogenetic data. These models capture the contact heterogeneity and
population structure often observed in real host communities. It has previously
been demonstrated analytically and through simulation studies that network
structure can affect the speed and pattern of epidemic spread. Consequently,
the often-made assumption of a homogeneously mixed population may introduce
errors into epidemiological predictions. Furthermore, the network itself may be
of interest from a public health perspective, such as for investigating the
presence of highly-connected ``superspreader'' nodes.

Despite their potential usefulness, most network parameters are extraordinarily
difficult to estimate with existing methods. Direct epidemiological
investigation of contact networks requires extensive contact tracing and may be
hampered by misreporting. Viral genetic data are much easier to collect, but as
far as we are aware, no other phylodynamic methods have been developed to make
use of this data to fit network models. Therefore, our work opens new avenues
for phylodynamic investigation, and we believe it will be of broad interest to
the molecular evolution and epidemiology community.

All authors fulfill the authorship criteria, and have agreed to this
submission. \textit{Molecular Biology and Evolution} is our first choice for
this manuscript, which has not been submitted to any other journal.

Thank you for your time,

Rosemary M. McCloskey

\section{Reviewers}

\begin{enumerate}
  \item David Welch
  \item Gabriel Leventhal
  \item Erik Volz
  \item Tanja Stadler
  \item Oliver Ratmann
  \item Andrew Rambaut
\end{enumerate}

\section{Editors}

\begin{enumerate}
  \item Thomas Leitner
  \item Beth Shapiro
  \item Kieth Crandall
  \item John Novembre
\end{enumerate}

\section{Funding}

\begin{enumerate}
  \item Bill \& Melinda Gates Foundation (OPP1110049)
  \item Canadian Institutes for Health Research (HOP-111406)
\end{enumerate}

\section{Significant discoveries}

(\wordcount / 150 words)

Through a simulation study, we have found that certain parameters of the
contact network underlying a viral epidemic can be quantitatively estimated
from phylogenetic data. In particular, we investigated a preferential
attachment network model and were able to accurately infer the attachment power
and the number of infected nodes in the network, in most simulation scenarios
we considered. On the other hand, the average degree and total number of nodes
in the network had an insufficient effect on tree shape and could not be
quantitatively estimated. When applied five real-world HIV epidemics, we found
substantial heterogeneity of these parameters, which are known to have a
significant impact on the speed and pattern of epidemic spread.

\section{Significant methodological or theoretical advances}

(\wordcount / 150 words)

We have developed a phylodynamic method to quantitatively estimate the
parameters of contact network models from phylogenetic data. Our method is
based on approximate Bayesian computation, and combines adaptive sequential
Monte Carlo, Gillespie simulation, and a kernel-based tree similarity score.
Due to the use of the tree kernel, the method falls under the heading of
"kernel-ABC", which has been previously demonstrated to be a robust and
effective means of estimating epidemiological quantities from observed
phylogenies. To our knowledge, no other methods have been developed to estimate
network parameters from this type of data. Therefore, our method expands the
domain of phylodynamic inference to include network structure, offering
molecular epidemiologists a new perspective into ongoing epidemics. 

\section{Significant new tools or resources}

(\wordcount / 150 words)

We provide a computer program "netabc" implementing our kernel-ABC method. The
C source code for the tool is freely available under a permissive license at
github.com/rmcclosk/netabc. This allows users to both apply the method as-is to
their data, and to incorporate new network models tailored to particular
research questions. The program is multithreaded, facilitating faster analysis
when more computing power is available. In addition, we provide two
supplemental programs to assist in phylodynamic analyses involving contact
networks: "nettree", which simulates a transmission tree over a contact network
by Gillespie simulation, and "treekernel", which quickly computes the kernel
similarity score between two trees. These tools will enable researchers to
quickly test hypotheses about the effect of network structure on transmission
tree shape.

\section{Broad impact}

(\wordcount / 150 words)

Phylodynamics and molecular epidemiology are relatively new areas of study, yet
they have undergone rapid development and have already seen public health
applications for real epidemics. Recent advances in sequencing technology have
allowed the collection of viral genetic data on a population scale in diverse
geographic and demographic settings. These data, together with phylodynamic
methods, provide an opportunity to gain insight into the real-world dynamics of
disease spread. Our work expands the domain of phylodynamic inference to
include structural parameters of the contact network underlying an epidemic.
Because our method is broadly applicable to any model from which networks can
be sampled, it will enable researchers to quantitatively investigate population
structure and contact heterogeneity in a variety of contexts.

\end{document}
