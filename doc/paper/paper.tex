\documentclass[12pt]{article}\usepackage[]{graphicx}\usepackage[]{color}
%% maxwidth is the original width if it is less than linewidth
%% otherwise use linewidth (to make sure the graphics do not exceed the margin)
\makeatletter
\def\maxwidth{ %
  \ifdim\Gin@nat@width>\linewidth
    \linewidth
  \else
    \Gin@nat@width
  \fi
}
\makeatother

\definecolor{fgcolor}{rgb}{0.345, 0.345, 0.345}
\newcommand{\hlnum}[1]{\textcolor[rgb]{0.686,0.059,0.569}{#1}}%
\newcommand{\hlstr}[1]{\textcolor[rgb]{0.192,0.494,0.8}{#1}}%
\newcommand{\hlcom}[1]{\textcolor[rgb]{0.678,0.584,0.686}{\textit{#1}}}%
\newcommand{\hlopt}[1]{\textcolor[rgb]{0,0,0}{#1}}%
\newcommand{\hlstd}[1]{\textcolor[rgb]{0.345,0.345,0.345}{#1}}%
\newcommand{\hlkwa}[1]{\textcolor[rgb]{0.161,0.373,0.58}{\textbf{#1}}}%
\newcommand{\hlkwb}[1]{\textcolor[rgb]{0.69,0.353,0.396}{#1}}%
\newcommand{\hlkwc}[1]{\textcolor[rgb]{0.333,0.667,0.333}{#1}}%
\newcommand{\hlkwd}[1]{\textcolor[rgb]{0.737,0.353,0.396}{\textbf{#1}}}%

\usepackage{framed}
\makeatletter
\newenvironment{kframe}{%
 \def\at@end@of@kframe{}%
 \ifinner\ifhmode%
  \def\at@end@of@kframe{\end{minipage}}%
  \begin{minipage}{\columnwidth}%
 \fi\fi%
 \def\FrameCommand##1{\hskip\@totalleftmargin \hskip-\fboxsep
 \colorbox{shadecolor}{##1}\hskip-\fboxsep
     % There is no \\@totalrightmargin, so:
     \hskip-\linewidth \hskip-\@totalleftmargin \hskip\columnwidth}%
 \MakeFramed {\advance\hsize-\width
   \@totalleftmargin\z@ \linewidth\hsize
   \@setminipage}}%
 {\par\unskip\endMakeFramed%
 \at@end@of@kframe}
\makeatother

\definecolor{shadecolor}{rgb}{.97, .97, .97}
\definecolor{messagecolor}{rgb}{0, 0, 0}
\definecolor{warningcolor}{rgb}{1, 0, 1}
\definecolor{errorcolor}{rgb}{1, 0, 0}
\newenvironment{knitrout}{}{} % an empty environment to be redefined in TeX

\usepackage{alltt}
\usepackage[T1]{fontenc}
\usepackage{mathptmx}
\usepackage[backend=biber, style=numeric-comp, sorting=none]{biblatex}
\usepackage{fullpage}
\usepackage{array}
\usepackage{multirow}
\usepackage{setspace}
\usepackage{cleveref}
\usepackage[normalem]{ulem}
\usepackage{color}
\usepackage{amsmath}

\addbibresource{papers.bib}

% beginning of diff stuff
\usepackage{twoopt}

\let\mref\cref
\let\Mref\Cref
\let\mtextcite\textcite
\let\mautocite\autocite
\let\mciteauthor\citeauthor
\renewcommand{\cref}[1]{\mbox{\mref{#1}}}
\renewcommand{\Cref}[1]{\mbox{\Mref{#1}}}
\renewcommandtwoopt{\textcite}[3][][]{\mbox{\mtextcite[#1][#2]{#3}}}
\renewcommandtwoopt{\autocite}[3][][]{\mbox{\mautocite[#1][#2]{#3}}}
\renewcommand{\citeauthor}[1]{\mbox{\mciteauthor{#1}}}

\newcommand{\add}[1]{\color{blue} \uline{#1} \color{black}}
\newcommand{\del}[1]{\color{red} \sout{#1} \color{black}}
% end of diff stuff

\newcommand{\tablepath}{../tables}
\graphicspath{{../figures/}}
\IfFileExists{upquote.sty}{\usepackage{upquote}}{}
\begin{document}
\onehalfspacing

\title{Reconstructing contact network parameters from viral phylogenies}
\author{Rosemary M. McCloskey$^1$, Richard H. Liang$^1$, and Art F.Y. Poon$^{1,2}$ \\
\small $^1$BC Centre for Excellence in HIV/AIDS, Vancouver, Canada \\ \small $^2$ Department of Medicine, University of British Columbia, Vancouver, Canada}



\maketitle
\abstract{Models of the spread of disease in a population often make the simplifying
assumption that the population is homogeneously mixed, or is divided into
homogeneously mixed compartments. However, human populations have complex
structures formed by social contacts, which can have a significant influence on
the rate of epidemic spread. Contact network models capture this structure by
explicitly representing each contact that could possibly lead to a
transmission. We developed a method based on approximate Bayesian computation
(ABC) for estimating structural parameters of the contact network underlying an
observed viral phylogeny. The method combines adaptive sequential Monte Carlo
for ABC, Gillespie simulation for propagating epidemics though networks, and a
kernel-based tree similarity score. We used the method to fit the
Barab\'{a}si-Albert network model to simulated transmission trees and applied
it to viral phylogenies estimated from six real-world HIV sequence datasets. On
simulated data, we found that the preferential attachment power and the number
of infected nodes in the network can often be accurately estimated. On the
other hand, the mean degree of the network, as well as the total number of
nodes, appeared to be weakly or non-identifiable with ABC. We observed
substantial heterogeneity in the parameter estimates on real datasets, with
point estimates for the preferential attachment power ranging from 0.06 to
1.05. These results underscore the importance of considering contact structures
when performing phylodynamic inference. Our method offers the potential to
quantitatively investigate the contact network structure underlying viral
epidemics.
}

\section*{Introduction}

When an infectious disease spreads through a population, transmissions are
generally more likely to occur between certain pairs of individuals. Such pairs
must have a particular mode of contact with one another, which varies with the
mode of transmission of the disease. For airborne pathogens, physical proximity
may be sufficient, while for sexually transmitted diseases, sexual or in some
cases blood-to-blood contact is required. The population together with the set
of links between individuals along which transmission can occur is called the
contact network~\autocite{klovdahl1985social, morris1993epidemiology}. The
structure of the contact network underlying an epidemic can profoundly impact
the speed and pattern of the epidemic's expansion. Network structure can
influence the prevalence trajectory~\autocite{o2011contact, ma2013importance}
and epidemic threshold~\autocite{barthelemy2005dynamical}, in turn affecting
the estimates of quantities such as effective viral population
size~\autocite{goodreau2006assessing}.  From a public health perspective,
contact networks have been explored as tools for curtailing epidemic spread, by
way of interventions targeted to well-connected
nodes~\autocite{wang2015targeting}. True contact networks are a challenging
type of data to collect, requiring extensive epidemiological
investigation~\autocite{welch2011statistical, eames2015six}.

Viral sequence data, on the other hand, has become \del{relatively inexpensive
and straightforward to collect on a population level.} \add{easier to collect
as the cost of sequencing has declined. In the case of HIV, genotyping has
become part of routine clinical care in several health regions.} Due to the
high mutation rate of RNA viruses, epidemiological processes impact the course
of viral evolution, thereby shaping the inter-host viral
phylogeny~\autocite{drummond2003measurably}. The term ``phylodynamics'' was
coined to describe this interaction, as well as the growing family of inference
methods to estimate epidemiological parameters from viral
phylogenies~\autocite{grenfell2004unifying}. These methods have revealed
diverse properties of local viral outbreaks, from basic reproductive
number~\autocite{stadler2012estimating}, to the degree of
clustering~\autocite{hughes2009molecular}, to the elevated transmission risk
during acute infection~\autocite{volz2012simple}. On the other hand, although
sophisticated methods have been developed for fitting complex population
genetic models to phylogenies~\autocite{volz2012complex,
rasmussen2014phylodynamic}, inference of structural network parameters has to
date been limited. However, it has been shown that network structure has a
tangible impact on phylogeny shape~\autocite{leventhal2012inferring,
colijn2014phylogenetic, goodreau2006assessing, robinson2013dynamics,
villandre2016assessment}, suggesting that such statistical inference might be
possible~\autocite{welch2011statistical}. \add{In the context of networks,
sequence data have the advantage of being objective, in that they are not
affected by misreporting. However, just as with survey data, it is important to
collect a representative sample from the population to perform accurate
inference~\autocite{novitsky2014impact}.}

Survey-based studies of sexual networks~\autocite{colgate1989risk,
liljeros2001web, schneeberger2004scale, latora2006network, rothenberg2007large,
clemenccon2015statistical} have found that these networks tend to have a degree
distribution which follows a power law \autocite[although there has been some
disagreement, see][]{handcock2004likelihood}. That is, the number of nodes of
degree $k$ is proportional to $k^{-\gamma}$ for some constant $\gamma$. These
networks are also referred to as
``scale-free''~\autocite{barabasi1999emergence}. One process by which
scale-free networks can be generated is preferential attachment, where nodes
with a high number of contacts attract new connections at an elevated rate. The
first contact network model incorporating preferential attachment was
introduced by \textcite{barabasi1999emergence}, and is now referred to as the
Barab\'asi-Albert (BA) model. Under this model, networks are formed by
iteratively adding nodes with $m$ new edges each. In the most commonly studied
formulation, these new edges are joined to existing nodes of degree $k$ with
probability proportional to $k$, so that nodes of high degree tend to attract
more connections. \citeauthor{barabasi1999emergence} suggested an extension
where the probability of attaching to a node of degree $k$ is $k^\alpha$ for
some non-negative constant $\alpha$, and we use this extension in this work.
\add{When $\alpha \neq 1$, the degree distribution is no longer a power law:
for $\alpha < 1$, the distribution is a stretched exponential, while for
$\alpha > 1$, it is a ``gelation'' type distribution where one or a few hub
nodes are connected to nearly every other node in the
graph~\autocite{krapivsky2000connectivity}.}

Previous work offers precedent for the possibility of statistical inference of
structural network parameters. \textcite{britton2002bayesian} develop a Bayesian
approach to estimate the edge density in an Erd\H{o}s-R\'enyi
network~\autocite{erdos1960evolution} given observed infection dates, and
optionally recovery dates. Their approach was later extended by
\textcite{groendyke2011bayesian} and applied to a much larger data set of 188
individuals. \textcite{volz2007susceptible, volz2008sir} developed differential
equations describing the spread of a susceptible-infected (SI) epidemic on
static and dynamic contact networks with several degree distributions, which
could in principle be used for inference if observed incidence trajectories
were available. \textcite{brown2011transmission} analyzed the degree distribution
of an approximate transmission network, estimated based on genetic similarity
and estimated times of infection, relating 60\% of HIV-infected men who have
sex with men (MSM) in the United Kingdom. The transmission network is a
subgraph of the contact network which includes only those edges which have
already led to a new infection. The authors found that a Waring distribution,
which is produced by a more sophisticated preferential attachment model, was a
good fit to their estimated network. 

Standard methods of model fitting involve calculation of the likelihood of
observed data under the model. In maximum likelihood estimation, a quantity
proportional to the likelihood is optimized, often through a standard
multi-dimensional numerical optimization procedure. Bayesian methods integrate
prior information by optimizing the posterior probability instead. To avoid
calculation of a normalizing constant, Bayesian inference is often performed
using Markov chain Monte Carlo (MCMC), which uses likelihood \emph{ratios} in
which the normalizing constants cancel out. Unfortunately, it is generally
difficult to explicitly calculate the likelihood of an observed transmission
tree under a contact network model, even up to a normalizing constant. To do
so, it would be necessary to integrate over all possible networks, and also
over all possible labellings of the internal nodes of the transmission tree.
While it is not known (to us) whether such integration is tractable, a simpler
alternative is offered by likelihood-free methods, namely approximate Bayesian
computation (ABC)~\autocite{tavare1997inferring, beaumont2002approximate}. ABC
leverages the fact that, although calculating the likelihood may be
impractical, generating simulated datasets according to a model is often
straightforward. If our model fits the data well, the simulated data it
produces should be similar to the observed data. More formally, if $D$ is the
observed data, the posterior distribution $f(\theta \mid D)$ on model
parameters $\theta$ is replaced as the target of statistical inference by
$f(\theta \mid \rho(\hat{D}, D) < \varepsilon)$, where $\rho$ is a distance
function, $\hat{D}$ is a simulated dataset according to $\theta$, and
$\varepsilon$ is a small tolerance~\autocite{sunnaaker2013approximate}. \del{In
the specific case when $\rho$ is a kernel function, the approach is known as
ABC~\autocite{nakagome2013kernel, poon2015phylodynamic}.} \add{Our
group~\autocite{poon2015phylodynamic} and others~\autocite{park2015k2} have
demonstrated that taking $\rho$ to be a well-chosen kernel function can produce
a more accurate ABC approximation than the typical choice of a difference of
summary statistics.}

Here, we develop a method using ABC to estimate the parameters of contact
network models from observed phylogenetic data. The distance function we use is
the tree kernel developed by \textcite{poon2013mapping}, which computes a
weighted dot product of the trees' representations in the space of all possible
subset trees. We apply the method to investigate the parameters of the BA
network model on a variety of simulated and real datasets. Our results show
that some network parameters can be inferred with reasonable accuracy, while
others \del{have a minimal detectable impact on tree shape and therefore cannot
be estimated accurately}\add{are weakly- or non-identifiable with ABC}. We also
find that these parameters can vary considerably between real epidemics from
different settings.

\section*{Methods}

\subsection*{\textit{Netabc}: phylogenetic inference of contact network
parameters with ABC}

We have developed an ABC-based method to perform statistical inference of
contact network parameters from a transmission tree estimated from an observed
viral phylogeny. We implemented the adaptive sequential Monte Carlo (SMC)
algorithm for ABC developed by \textcite{del2012adaptive}. The SMC algorithm
keeps track of a population of parameter ``particles'', which are initially
sampled from the parameters' joint prior distribution. Several datasets are
simulated under the model of interest for each of the particles. In this case,
the datasets are transmission trees, which are generated by a two-step process.
First, a contact network is simulated according to the network model being fit.
Second, a transmission tree is simulated over that network with a Gillespie
simulation algorithm~\autocite{gillespie1976general}, in the same fashion as
several previous studies \autocite[\textit{e.g.}][]{robinson2013dynamics,
leventhal2012inferring}. \add{Tips of the simulated transmission tree are
randomly removed until the simulated tree has the same number of tips as the
input tree.} The particles are weighted according to the similarity between
their associated simulated trees and the observed tree. To quantify this
similarity, we used the tree kernel developed by \textcite{poon2013mapping}.
Particles are iteratively perturbed by applying a Metropolis-Hastings kernel
and, if the move is accepted, simulating new datasets under the new parameters.
When a particle's weight drops to zero,
because its simulated trees are too dissimilar to the observed tree, the
particle is dropped from the population, and eventually replaced by a resampled
particle with a higher weight. As the algorithm progresses, the population
converges to a Monte Carlo approximation of the ABC target distribution, which
is assumed to approximate the desired posterior~\autocite{del2012adaptive,
sunnaaker2013approximate}. 

\add{In the original formulation of ABC-SMC~\autocite{sisson2007sequential,
beaumont2009adaptive}, the user is required to specify a decreasing sequence of
tolerances $\lbrace\varepsilon_i\rbrace$. At iteration $i$, particles with no
associated simulated datasets within distance $\varepsilon_i$ of the observed
data are removed from the population. In the adaptive version of
\textcite{del2012adaptive}, the sequence of tolerances is determined
automatically by fixing the decay rate of the population's expected sample size
(ESS) to a user-defined value. \citeauthor{del2012adaptive} call this value
$\alpha$, but we will refer to it here as $\alpha_{\text{ESS}}$ to avoid
confusion with the preferential attachment power parameter of the BA model.} 

To check that our implementation of Gillespie simulation was correct, we
reproduced Figure 1A of \textcite{leventhal2012inferring} (our
\cref{fig:leventhal}), which plots the imbalance of transmission trees
simulated over four network models at various levels of pathogen
transmissibility. Our implementation of adaptive ABC-SMC was tested by applying
it to the same mixture of Gaussians used by \citeauthor{del2012adaptive} to
demonstrate their method (originally used by~\textcite{sisson2007sequential}).
We were able to obtain a close approximation to the function (see
\cref{fig:smctest}), and attained the stopping condition used by the authors in
a comparable number of steps. 

Nodes in our networks followed simple SI dynamics, meaning that they became
infected at a rate proportional to their number of infected neighbours, and
never recovered. For all analyses, the transmission trees' branch lengths were
scaled by dividing by their mean. We used the \textit{igraph} library's
implementation of the BA model~\autocite{csardi2006igraph} to generate the
graphs. The analyses were run on Westgrid (\url{https://www.westgrid.ca/}) and
a local computer cluster. \add{A computer program implementing our method is
freely available at \url{https://github.com/rmcclosk/netabc} (last accessed
\today).}

\subsection*{\del{Kernel classifiers}\add{Classifiers for BA model parameters from tree shapes}}

\add{We considered four parameters related to the BA model, denoted $N$, $m$,
$alpha$, and $I$. The first three of these parameterize the network structure,
while $I$ is related to the simulation of transmission trees over the network.
However, we will refer to all four as BA parameters. $N$ denotes the total
number of nodes in the network, or equivalently, susceptible individuals in the
population. $m$ is the number of new undirected edges added for each new
vertex, or equivalently one-half of the average degree. $\alpha$ is the power
of preferential attachment -- new nodes are attached to existing nodes of
degree $d$ with probability proportional to $d^{\alpha} + 1$. Finally, $I$ is
the number of infected individuals at the time when sampling occurs. The
$\alpha$ parameter is unitless, while $m$ has units of edges or connections per
vertex, and $N$ and $I$ both have units of nodes or individuals.}

\add{Before proceeding with a full validation of \textit{netabc} on simulated
data, we undertook an experiment designed to assess the identifiability of the
BA parameters. One parameter of the BA model was investigated at a time while
holding all others fixed, a strategy commonly used when performing sensitivity
analyses of mathematical models. This allowed us to perform a fast preliminary
analysis without dealing with the ``curse of dimensionality'' of the full
parameter space. We simulated trees under three different values of each
parameter, and asked how well we could tell the different trees apart. The
better we are able to distinguish the trees, the more identifiability we might
expect for the corresponding parameter when we attempt to estimate it with
ABC.}

\add{This experiment also had the secondary purpose of validating our choice of
the tree kernel as a distance measure in ABC. To tell the trees apart, we used
a classifier based on the tree kernel, but we also tested two other tree shape
statistics. Sackin's index~\autocite{shao1990tree} is a measure of tree
imbalance which not take branch lengths into account, considering only the
topology. The normalized lineages-through-time
\autocite[nLTT,][]{janzen2015approximate} compares two trees based on normalized
distributions of their branching times, and does not explicitly consider the
topology. Since the tree kernel incorporates both of these sources of
information, we expected it to outperform the other two statistics. Finally,
the tree kernel can be tuned by adjusting the values of the meta-parameters
$\lambda$ and $\sigma$ (the ``decay factor'' and ``radial basis function
variance'', see \textcite{poon2013mapping}). $\lambda$ is used to penalize
ween large subset trees which tend to dominate the kernel score. When $\lambda
= 0$, all but the root branches of each subset tree are ignored, while when
$\lambda = 1$, no penalty is applied. $\sigma$ controls how strictly the notion
of similarity is applied to branch lengths. When $\sigma = 0$, branch lengths
must match exactly, while as $\sigma \to \infty$, branch lengths are not
considered at all.}
  
\del{We used the phylogenetic kernel developed by \textcite{poon2013mapping}}
To test whether the parameters of the BA model had a \add{measurable} effect on
tree shape, 100 networks were simulated under each of three different values of
$\alpha$: 0.5, 1.0, and 1.5 (300 networks total). The other parameters were
fixed to the following values: $N = 5000$, $I = 1000$, and $m = 2$. A
transmission tree with 500 tips was simulated over each network (300
transmission trees total). The 300 trees were compared pairwise with the tree
kernel to form a $300 \times 300$ kernel matrix. The kernel meta-parameters
$\lambda$ \del{(the ``decay factor''),} and $\sigma$ \del{(the ``radial basis
function variance'')~\autocite[see][]{poon2013mapping},} were set to 0.3 and 4
respectively. \add{We also computed a $300 \times 300$ matrix of pairwise nLTT
values, and a $1 \times 300$ vector of Sackin's index values.} We constructed 
\add{three classifiers for $\alpha$:} a kernel support vector regression (kSVM)
\add{from the kernel matrix} with the \textit{kernlab}
package~\autocite{zeileis2004kernlab}, and two ordinary SVMs \add{from the nLTT
matrix and Sackin's index vector} with the e1071
package~\autocite{meyer2015e1071}.\del{, and a linear regression from the
Sackin's index values.} The accuracy of each classifier was evaluated with 1000
two-fold cross validations \add{with equally-sized folds}.
  
Three similar experiments were performed for the other BA model parameters (one
experiment per parameter). $m$ was varied between 2, 3, and 4; $I$ between 500,
1000, and 2000; and $N$ between 3000, 5000, and 8000. The parameters not being
tested were fixed at the values $N$ = 5000, $I$ = 1000, $m$ = 2, and $\alpha$ =
1. Thus, we performed a total of four cross-validations \add{for each
classifier}, one for each of the BA model parameters $\alpha$, $I$, $m$, and
$N$. We repeated these four cross-validations with different values of
$\lambda$ (0.2, 0.3, and 0.4) and $\sigma$ ($2^{-3}$, $2^{-2}$, \ldots, $2^3$),
as well as on trees with differing numbers of tips (100, 500, and 1000).
\add{For the structural parameters $\alpha$, $m$, and $N$, the experiments were
repeated with three different fixed values of $I$ (500, 1000, and 2000).}
\del{and in epidemics of differing size (500, 1000, and 2000).} The combination
of the number of sampled individuals (\textit{i.e.} the number of tips) and the
epidemic size (\textit{i.e.} $I$) will be referred to as an ``epidemic
scenario''. When evaluating the classifier for $I$, we did not consider trees
with 1000 tips, because one of the tested $I$ values was 500, and the number of
tips cannot be larger than $I$.

\del{For each of the four parameters, we also tested a linear regression against
Sackin's index~\autocite{shao1990tree} and an ordinary SVM based on the normalized
lineages-through-time (nLTT) statistic~\autocite{janzen2015approximate}.}

\subsection*{ABC simulations}
  
\add{We tested \textit{netabc} by jointly estimating the four parameters of the
BA model. We used the standard validation approach of simulating transmission
trees under the model with known parameter values and attempting to recover
those values with \textit{netabc}. The algorithm was not informed of any of
the true parameter values for the main set of simulations.} We simulated three
transmission trees, each with 500 tips, under every element of the Cartesian
product of these parameter values: $N$ = 5000, $I$ = \{1000, 2000\}, $m$ = \{2,
3, 4\}, and $\alpha$ = \{0.0, 0.5, 1, 1.5\}. This produced a total of 24
parameter combinations $\times$ three trees per combination = 72 trees total.
The adaptive ABC algorithm was applied to each tree with these priors: $m \sim$
DiscreteUniform(1, 5), $\alpha \sim$ Uniform(0, 2), and $(N, I)$ jointly
uniform on the region \{$500 \leq N \leq 15000$, $500 \leq I \leq 5000$, $I
\leq N$\}. Proposals for $\alpha$, $N$, and $I$ were Gaussian, while proposals
for $m$ were Poisson. Following \textcite{del2012adaptive} and
\textcite{beaumont2009adaptive}, the variance of all proposals was equal to the
empirical variance of the particles.

The SMC settings used were 1000 particles, 5 simulated datasets per particle,
and \del{the ``quality'' parameter controlling the decay rate of the tolerance
$\varepsilon$ set to}\add{$\alpha_{\text{ESS}} =  0.95$}. We used the same
stopping criterion as \citeauthor{del2012adaptive}, namely when the MCMC
acceptance rate dropped below 1.5\%. \del{Point estimates for the parameters
were obtained by taking the highest point of an estimated kernel density on the
final set of particles, calculated using the \textit{density} function with the
default parameters in \textit{R}.} \add{Approximate posterior means for the
parameters were obtained by taking the weighted average of the final set of
particles.} Highest posterior density (HPD) intervals were calculated with the
\textit{HPDinterval} function from the \textit{R} package
\textit{coda}~\autocite{plummer2006coda}.

\add{To evaluate the effects of the true parameter values on the accuracy of
the posterior mean estimates, we analyzed the $\alpha$ and $I$ parameters
individually using generalized linear models (GLMs) The response variable was
the error of the point estimate, and the predictor variables were the true
values of $\alpha$, $I$, and $m$. We did not test for differences across true
values of $N$, because $N$ was not varied in these simulations. The
distribution family and link function for the GLMs were Gaussian and inverse,
respectively, chosen by examination of residual plots and Akaike information
criteria (AIC). The $p$-values of the estimated GLM coefficients were corrected
using Holm-Bonferroni correction~\autocite{holm1979simple} with $n =
6$ (two GLMs with three predictors each). Because there was clearly little to
no identifiability of $N$ and $m$ with ABC (see results in next section), we
did not construct GLMs for those parameters.}

Two further simulations were performed to address \del{potential sources of
error} \add{the possible impact of two types of model misspecification}. To
evaluate the effect of model misspecification in the case of heterogeneity
among nodes, we generated a network where half the nodes were attached with
power $\alpha$ = 0.5, and the other half with power $\alpha$ = 1.5. The other
parameters for this network were $N$ = 5000, $I$ = 1000, and $m$ = 2. To
investigate the effects of potential sampling bias, we simulated a transmission
tree where the tips were sampled in a peer-driven fashion, rather than at
random. That is, the probability to sample a node was twice as high if any of
that node's network peers had already been sampled. The parameters of this
network were $N= 5000$, $I = 2000$, $m = 2$, and $\alpha = 0.5$.

\add{Despite the fact that the parameter values used to generate the simulated
transmission trees were known, the true posterior distributions of the BA
parameters were unknown. Therefore, any apparent errors or biases in the
estimates could be due to either poor performance of our method, or to real
features of the posterior distribution. Two retrospective experiments were
performed to disambiguate some of the observed errors. To assess the impact of
the SMC settings on \textit{netabc}'s accuracy, we ran \textit{netabc}
twice on the same simulated transmission tree. For the first run, the SMC
settings were the same as in the other simulations: 1000 particles, 5 simulated
transmission trees per particle, and $\alpha_{\text{ESS}} = 0.95$. The second
run was performed with 2000 particles, 10 simulated transmission trees per
particle, and $\alpha_{\text{ESS}} = 0.97$. To investigate the extent to which
errors in the estimated BA parameters were due to true features of the
posterior, rather than an inaccurate ABC approximation, we performed marginal
estimation for one set of parameter values. Each combination of 1, 2, or 3
model parameters (14 combinations total) was fixed to their known values, and
the remaining parameters were estimated with \textit{netabc}. The parameter
values were $\alpha = 0.0$, $m = 2$, $I = 2000$, and $N = 5000$.}
  
\subsection*{Investigation of published data}
  
We applied our ABC method to ten published HIV datasets. Because the BA model
generates networks with a single connected component, we specifically searched
for datasets which originated from existing clusters, either phylogenetically
or geographically defined. Characteristics of the datasets we investigated are
given in \cref{tab:data}. \add{For clarity, we will refer to each dataset by
its risk group and location of origin in the text. For example, the
\textcite{zetterberg2004two} data will be referred to as IDU/Estonia.}

\begin{table*}[!t]
  \centering
  \begin{tabular}{ccccc}
  Reference & Sequences ($n$) & Location & Risk group & Gene \\
  \hline
  \textcite{wang2015targeting} & 173 & Beijing, China & MSM & \textit{pol} \\
  \textcite{cuevas2009hiv} & 287 & Basque Country, Spain & mixed & \textit{pol} \\
  \textcite{novitsky2013phylogenetic} & \multirow{2}{*}{180} &
  \multirow{2}{*}{Mochudi, Botswana} & \multirow{2}{*}{HET} &
  \multirow{2}{*}{\textit{env}} \\ \textcite{novitsky2014impact} \\
  \textcite{li2015hiv} & 280 & Shanghai, China & MSM & \textit{pol} \\
  \textcite{niculescu2015recent} & 136 & Romania & IDU & \textit{pol} \\
  \hline
\end{tabular}

  \caption{
    Characteristics of published datasets investigated with ABC.
    Acronyms: MSM, men who have sex with men; IDU, injection drug users; HET,
    heterosexual. The HET data were sampled from a primarily heterosexual risk
    environment, but did not explicitly exclude other risk factors. The number
    of sequences column indicates how many sequences were included in our
    analysis; there may have been additional sequences linked to the study
    which we excluded for various reasons (see methods).
  }
  \label{tab:data}
\end{table*}

We downloaded all sequences associated with each published study from GenBank.
\add{For the IDU/Romania data, only sequences from injection drug users (IDU,
whose sequence identifiers included the letters ``DU'') were included in the
analysis. \textcite{kao2011surveillance} (MSM/Taiwan) found a strong
association in their study population between subtype and risk group - subtype
B was most often associated with men who have sex with men (MSM), whereas IDU
were usually infected with a circulating recombinant form. Since there were
many more subtype B sequences in their data than sequences of other subtypes,
we restricted our analysis to the subtype B sequences and labelled this dataset
as MSM. Two datasets (HET/Uganda and HET/Malawi) included both \textit{env}
and \textit{gag} sequences. Each gene was analyzed separately to assess the
robustness of \textit{netabc} to the particular HIV gene sequence used to
estimate a transmission tree. The IDU/Estonia data also sequenced both genes, 
but the highly variable coverage and high homology of the \textit{gag}
sequences made it impossible to obtain a sufficiently large block of
non-identical sequences to analyze. Therefore, we analyzed only \textit{env}
for this dataset.}

\del{For the \textcite{novitsky2014impact} data,} Each \textit{env} sequence
was aligned pairwise to the HXB2 reference sequence (GenBank accession number
K03455), and the hypervariable regions were clipped out with
\textit{BioPython} version 1.66+~\autocite{cock2009biopython}. Sequences were
multiply aligned using \textit{MUSCLE} version 3.8.31
\autocite{edgar2004muscle}, and alignments were manually inspected with
\textit{Seaview} version 4.4.2 \autocite{gouy2010seaview}. \add{Duplicated
sequences were removed with \textit{BioPython}.} Phylogenies were constructed
from the nucleotide alignments by approximate maximum likelihood using
\textit{FastTree2} version 2.1.7 \autocite{price2010fasttree} with the
generalized time-reversible (GTR) model~\autocite{tavare1986some}. Transmission
trees were estimated by rooting and time-scaling the phylogenies by root-to-tip
regression, using a modified version of \textit{Path-O-Gen} (distributed as
part of \textit{BEAST}~\autocite{drummond2007beast}) as described
previously~\autocite{poon2015phylodynamic}. \add{Due to the removal of
duplicated sequences, all estimated transmission trees were fully binary.}

\add{To check if our results were robust to the choice of phylogenetic
reconstruction method, we built and reanalyzed phylogenies for the datasets
with the lowest and highest estimated $\alpha$ values (mixed/Spain and
IDU/Estonia) with \textit{RAxML}~\autocite{stamatakis2014raxml} with the
GTR+$\Gamma$ model of sequence evolution and rate heterogeneity. The trees were
rooted and time-scaled with \textit{Least Square
Dating}~\autocite[\textit{LSD},][]{to2016fast}. For expediency, the analysis
was run with the prior $m \sim \text{DiscreteUniform}(2, 5)$, which defines a
smaller total search space than the prior allowing $m = 1$. For both of these
datasets, we also analyzed five bootstrap replicate alignments generated by
resampling alignment columns with replacement.}

\del{Two} \add{Four} of the datasets \del{\autocite{li2015hiv,novitsky2014impact}}
(MSM/Shanghai, HET/Botswana, HET/Uganda, and MSM/USA)
were initially much larger than the others, containing 1265, 1299,
\add{1026/915 (\textit{env}/\textit{gag}), and 648} sequences respectively. To
ensure that the analyses were comparable, we reduced these to a number of
sequences similar to the smaller datasets. For the MSM/Shanghai data, we
detected a cluster of size 280 using a patristic distance cutoff of 0.02 as
described previously~\autocite{poon2015impact}. Only sequences within this
cluster were carried forward. For the HET/Uganda, HET/Botswana, and MSM/USA
data, no large clusters were detected using the same cutoff, so we analyzed
\del{a subtree} subsets of sizes 255, 180, and 180 respectively. \add{The
subset of the HET/Uganda data was chosen by eye such that the individuals were
monophyletic in both the \textit{gag} and \textit{env} trees. The other
subsets were arbitrarily chosen subtrees from phylogenies of the complete
datasets.}

For all datasets, we used the priors $\alpha$ $\sim$ Uniform(0, 2) and $N$ and
$I$ jointly uniform on the region \{$n \leq N \leq 10000$, $n \leq I \leq
10000$, $I \leq N$\}, where $n$ is the number of tips in the tree. Since the
value $m = 1$ produces networks with no cycles, which we considered fairly
implausible, we ran one analysis with the prior $m \sim$ DiscreteUniform(1, 5),
and one with the prior $m \sim$ DiscreteUniform(2, 5). The other parameters to
the SMC algorithm were the same as used for the simulation experiments,
\add{except that we used 10000 particles instead of 1000 to increase the
accuracy of the estimated posterior for all analyses except the bootstrap
replicates. This was computationally feasible due to the small number of runs
required for this analysis.}

\section*{Results}

\subsection*{\del{Kernel classifiers}\add{Classifiers for BA model parameters from tree shapes}}



We investigated the \add{identifiability of four} parameters of the BA network
model~\autocite{barabasi1999emergence}: the number of nodes $N$, the
preferential attachment power $\alpha$, the number of edges added per vertex
$m$, and the number of infected nodes $I$. \del{In addition to $m$ and $\alpha$
(see Introduction), we considered $N$, which denotes the total number of nodes
in the network, and $I$, which is the number of infected nodes at which to stop
the simulation and sample the transmission tree.} To examine the effect of
these parameters on tree shape, we simulated transmission trees under different
parameter values, calculated pairwise tree kernel scores between them, and
attempted to classify the trees using a kernel support vector machine (kSVM).
We also tested classifiers based on Sackin's index~\autocite{shao1990tree} and
the normalized lineages-through-time (nLTT)
statistic~\autocite{janzen2015approximate}. \add{We report the accuracy of the
classifiers, which is simply the proportion of trees which were assigned the
correct parameter value. Since there were three possible values, random
guessing would produce an accuracy of 0.33.} \del{The accuracy of each
classifier varied based on the parameter being tested} \add{The results are
shown in} \cref{fig:rsquared}. Classifiers based on \del{two other tree
statistics,} the nLTT and Sackin's index generally exhibited worse performance
than the tree kernel, although the magnitude of the disparity varied between
the parameters (\cref{fig:rsquared}, centre and right). \del{The results were
largely robust to variations in the tree kernel meta-parameters $\lambda$ and
$\sigma$, although accuracy varied between different epidemic and sampling
scenarios} \add{Larger datasets were generally classified more accurately
(\cref{fig:alphacrossv,fig:mcrossv,fig:Ncrossv,fig:Icrossv}), although large
values of $\lambda$ produced worse estimates on large datasets. Extremely low
$\sigma$ values, which require nearly-exact matches between branch lengths,
resulted in low accuracy in some cases (\textit{e.g.} \cref{fig:alphacrossv},
center row).}

\begin{figure}[ht]
    \centering
    \includegraphics[width=\textwidth]{kernel-rsquared.pdf}
    \caption[
        Cross-validation accuracy of kernel-SVM, nLTT-based SVM, and Sackin's
        index SVM classifiers for BA model parameters.
    ]{
        Cross-validation accuracy of kernel-SVM classifier (left), SVM
        classifiers using nLTT (centre) and Sackin's index (right) for BA model
        parameters. Kernel meta-parameters were set to $\lambda = 0.3$ and
        $\sigma = 4$. Each point was calculated based on 300 simulated
        transmission trees over networks with three different values of the
        parameter being tested, assuming perfect knowledge of the other
        parameters. Vertical lines are empirical 95\% confidence intervals
        based on 1000 two-fold cross-validations. \add{The classifiers for $I$
        were not evaluated with 1000-tip trees, because one of the tested $I$
        values was 500, and it is not possible to sample a tree of size 1000
        from 500 infected individuals.}
    }
    \label{fig:rsquared}
\end{figure}

The kSVM classifier for $\alpha$ had an average accuracy of 
    0.92,
compared to 
    0.6
for the nLTT, and
    0.77
for Sackin's index. \del{There was little variation about the mean for
different tree and epidemic sizes.} No classifier could accurately identify $m$
in any epidemic scenario, with average accuracy values of 
  0.35 for kSVM,
  0.32 for the nLTT, and
  0.38
for Sackin's index. There was little variation in accuracy between epidemic
scenarios, although the accuracy of the kSVM was slightly higher on 1000-tip
trees (\cref{fig:rsquared}, left).

The accuracy of classifiers for $I$ varied significantly with the number of
tips in the tree. For 100-tip trees, the average accuracy was
  0.59,
  0.58, and
  0.34
for the tree kernel, nLTT, and Sackin's index respectively.
For 500-tip
trees, the values increased to
  0.99,
  0.76, and
  0.37.
Finally, the performance of classifiers for $N$ depended heavily on the
epidemic scenario. The accuracy of the kSVM classifier ranged from
  0.36
for the smallest epidemic and smallest sample size, to
  0.81
for the largest. Accuracy for the nLTT ranged from 
  0.33
to
  0.63.
Sackin's index did not accurately classify $N$ in any scenario, with an average
accuracy of
  0.35
and little variation between scenarios.

\subsection*{ABC simulations}



\Cref{fig:abcpt} shows \del{maximum \textit{a posteriori} (MAP)}
\add{stratified posterior mean} point estimates of the BA model parameters
\add{$\alpha$ and $I$}, obtained with ABC on simulated data. \add{The
parameters $m$ and $N$ were not identifiable with ABC for any parameter
combinations (\cref{fig:abcpt2}).} \del{The estimates shown correspond only to
the simulations where the $m$ parameter was set to 2, however the results for
$m = 3$ and $m = 4$ were similar.} Average boundaries of 95\% HPD intervals
\add{for all parameters} are given in \cref{tab:abchpd}.

\begin{figure}[ht]
    \centering
    \includegraphics[width=\textwidth]{abc-boxplot}
    \caption[
        Posterior mean point estimates for BA model parameters $\alpha$ and $I$
        obtained by running \textit{netabc} on simulated data, stratified by
        true parameter values.
    ]{
        Posterior mean point estimates for BA model parameters $\alpha$ and $I$
        obtained by running \textit{netabc} on simulated data, stratified by
        true parameter values. First row of plots contains true versus
        estimated values of $\alpha$; second row contains true versus estimated
        values of $I$. Columns are stratified by $\alpha$, $I$, and $m$
        respectively. Dashed lines indicate true values. 
    }
    \label{fig:abcpt}
\end{figure}

\begin{table}[ht]
    \centering
    % latex table generated in R 3.2.3 by xtable 1.8-2 package
% Fri Jun 17 12:51:45 2016
\begin{tabular}{lr>{\raggedleft\arraybackslash}p{2.5cm}>{\raggedleft\arraybackslash}p{2.5cm}>{\raggedleft\arraybackslash}p{2.5cm}}
  \hline
Parameter & True value & Mean point estimate & Mean HPD lower bound & Mean HPD upper bound \\ 
  \hline
$\alpha$ & 0.0 & 0.36 & 0.01 & 0.81 \\ 
   & 0.5 & 0.43 & 0.04 & 0.83 \\ 
   & 1.0 & 0.90 & 0.51 & 1.09 \\ 
   & 1.5 & 1.52 & 1.26 & 1.81 \\ 
  $I$ & 1000 & 1450 & 651 & 2592 \\ 
   & 2000 & 2622 & 1114 & 4080 \\ 
  $m$ & 2 & 2.96 & 2.00 & 5.00 \\ 
   & 3 & 3.04 & 2.04 & 4.96 \\ 
   & 4 & 3.17 & 1.88 & 5.00 \\ 
  $N$ & 5000 & 9041 & 2613 & 14659 \\ 
   \hline
\end{tabular}

    \caption[
        Average posterior mean point estimates and 95\% highest posterior
        density interval widths for BA model parameter estimates obtained
        with \textit{netabc} on simulated data.
    ]{
        Average posterior mean point estimates and 95\% highest posterior
        density interval widths for BA model parameter estimates obtained with
        \textit{netabc} on simulated data. Three transmission trees were
        simulated under each combination of the listed parameter values, and
        the parameters were estimated with ABC without training.
    }
    \label{tab:abchpd}
\end{table}

\del{The accuracy of the parameter estimates obtained with ABC
paralleled the results from the kSVM classifier. Of the four parameters,
$\alpha$ was the most accurately estimated, with point estimates having a
median [IQR] absolute error of 
    0.11 
    [0.03 - 
    0.25].
The errors when the true value of $\alpha$ was zero were significantly greater
than those for the other values 
(Wilcoxon rank-sum test, $p$ = $0$).
Errors in estimating $\alpha$ also varied with the true value of $m$ just at
the threshold of statistical significance
    $p =0.5$),
but did not vary across the true values of $N$ or $I$ (both one-way ANOVA).
Estimates for $I$ were relatively accurate, with point estimate errors of
    492 
    [294 - 
    782] individuals.
These errors were significantly higher when the true value of $\alpha$ was
at least 1
    (Wilcoxon rank-sum test, $p$ = $0$)
and when the true value of $I$ was 2000 ($p < 10^{-5}$). The true value of $m$
did not affect the estimates of $I$ (one-way ANOVA).}

\del{The $m$ parameter was estimated correctly in only
    37 \%
of simulations, barely better than random guessing. The true values of the
other parameters did not significantly affect the estimates of $m$ (both
one-way ANOVA). Finally, the total number of nodes $N$ was consistently
over-estimated by about a factor of two
    (error 4153 
    [3660 - 
     4489] individuals).
No parameters influenced the accuracy of the $N$ estimates (all one-way ANOVA).}

\add{Across all simulations, the median [IQR] absolute errors of the parameter 
estimates obtained with \textit{netabc} were
    0.11 
    [0.03 - 
    0.25]
for $\alpha$,
    492 
    [294 - 
    782]
for $I$,
    1 
    [0 - 
    1]
for $m$, and
    4153 
    [3660 - 
    4489]
for $N$. These errors comprised, respectively,
    6\%,
    11\%,
    17\%, and
    29\%
of the regions of nonzero prior density. For $I$ and $N$, relative errors were
    38\%
    [20 - 
    50\%]
and
    83\%
    [73 - 
    90\%].
Average 95\% HPD interval widths were
    0.68,
    2454,
    3.01, and
    12046,
representing 
    34\%,
    55\%,
    50\%, and
    83\%
of the nonzero prior density regions. Point estimates of $I$ were upwardly
biased: $I$ was overestimated in
    69
out of 
    72
simulations
    (96\%).
The estimates for $m$ and $N$ were similar across all simulations (median [IQR]
point estimates
    3
    [3 - 
    3]
and 
    9153
    [8660 - 
    9489])
regardless of the true values of any of the BA parameters (\cref{fig:abcpt2}).}



% latex table generated in R 3.2.3 by xtable 1.8-2 package
% Thu Jul 14 15:34:47 2016
\begin{table}[ht]
\centering
\begin{tabular}{lllll}
  \hline
Dependent variable & Independent variable & Estimate & Standard error & $p$-value \\ 
  \hline
$\alpha$ & (Intercept) & $2$ & $0.6$ & $0.01$ \\ 
   & $\alpha$ & $10$ & $2$ & ${<}10^{-5}$ \\ 
   & $I$ & $-3\!\times\!10^{-4}$ & $2\!\times\!10^{-4}$ & $0.7$ \\ 
   & $m$ & $0.5$ & $0.2$ & $0.01$ \\ 
  $I$ & (Intercept) & $0.004$ & $5\!\times\!10^{-4}$ & ${<}10^{-5}$ \\ 
   & $\alpha$ & $-0.001$ & $2\!\times\!10^{-4}$ & ${<}10^{-5}$ \\ 
   & $I$ & $-4\!\times\!10^{-7}$ & $2\!\times\!10^{-7}$ & $0.05$ \\ 
   & $m$ & $-7\!\times\!10^{-5}$ & $8\!\times\!10^{-5}$ & $1$ \\ 
   \hline
\end{tabular}
\caption[Parameters of fitted GLMs relating error in estimated $\alpha$ and $I$ to true values of BA parameters.]{Parameters of fitted GLMs relating error in estimated $\alpha$ and $I$ to true values of BA parameters. GLMs ere fitted with the Gaussian distribution and inverse link function. Coefficients are interpretable as additive effects on the inverse of the mean error.} 
\label{tab:glm}
\end{table}



\add{To analyze the effects of the true parameter values on the
accuracy our estimates of $\alpha$ and $I$, we fitted one GLM for each of these
two parameters, with error rate as the dependent variable and the true
parameter values as independent variables. Since the estimates of $m$ and
$N$ were roughly equal across all simulations (\cref{fig:abcpt2}), GLMs were
not fitted for these parameters. The estimated coefficients are shown in
\cref{tab:glm}. The GLM analysis indicated that the error in estimates of
$\alpha$ decreased with larger true values of $\alpha$
    ($p {<}10^{-5}$)
and $m$
    ($p =0.01$)
but was not significantly affected by $I$. Qualitatively, $\alpha$ seemed to be
only weakly identifiable between the values of 0 and 0.5 (\cref{fig:abcpt}).
The error in the estimated $I$ value was slightly lower for smaller values of
$\alpha$
    ($p {<}10^{-5}$)
and $I$
    ($p =0.05$),
but was not significantly affected by the true value of $m$.}



The dispersion of the ABC approximation to the posterior also varied between
the parameters\del{, with narrower HPD intervals for the parameters with more
accurate point estimates} (\cref{tab:abchpd}). HPD intervals around $\alpha$
and $I$ were often narrow relative to the region of nonzero prior
density, whereas the intervals for $m$ and $N$ were more widely dispersed.
\Cref{fig:abc1,fig:abc2} \del{shows the distributions for one
simulation.} \add{show one- and two-dimensional marginal distributions for a
simulation with $\alpha$ and $I$ errors close to their respective medians. 
%In particular, the simulation shown is one of
%replace_number(nrow(middle)) with errors in both $\alpha$ and $I$
%between the round(q*100)nd and round((1-q)*100)th percentiles.
The parameters for this simulation were
    $\alpha = 1$,
    $I = 1000$,
    $m = 3$, and
    $N = 5000$.
The two-dimensional marginals indicate some dependence between pairs of
parameters, particularly $I$ and $N$ which show a diagonally shaped region of
high posterior density.} 

\begin{figure}[ht]
    \centering
    \includegraphics[width=0.7\textwidth]{{abc-posterior/1.0_1000_3_5000_0}.pdf}
    \caption[
        One-dimensional marginal posterior distributions of BA model
        parameters estimated by \textit{netabc} from a simulated transmission
        tree.
    ]{
        One-dimensional marginal posterior distributions of BA model parameters
        estimated by \textit{netabc} from a simulated transmission tree. Dashed
        lines indicate true values, solid lines indicate posterior means, and
        shaded areas show 95\% highest posterior density intervals.
    }
    \label{fig:abc1}
\end{figure}

\begin{figure}[ht]
    \centering
    \includegraphics[width=\textwidth]{{abc-posterior-2d/1.0_1000_3_5000_0}.pdf}
    \caption[
        Two-dimensional marginal posterior distributions of BA model parameters
        estimated by \textit{netabc} from a simulated transmission tree.
    ]{
        Two-dimensional marginal posterior distributions of BA model parameters
        estimated by \textit{netabc} from a simulated transmission tree. White
        circles indicate true values, magenta diamonds
        indicate posterior means. 
    }
    \label{fig:abc2}
\end{figure}



To test the effect of model misspecification, we simulated one network where
the nodes exhibited heterogeneous preferential attachment power (half 0.5, the
other half 1.5), with $m$ = 2, $N$ = 5000, and $I$ = 1000. The posterior mean
[95\% HPD] estimates for each parameter were: 
$\alpha$, 
  1.03 
  [0.67 -
   1.18];
$I$,
  1474 
  [511 -
   2990];
$m$,
  3 
  [1 -
   5];
$N$,
  9861 
  [3710-
   14977].
\del{The approximate one-dimensional marginal posterior distributions for this
simulation are shown in \cref{fig:mixed}.} To test the effect of sampling bias,
we sampled one transmission tree in a peer-driven fashion, where the
probability to sample a node was twice as high if one of its peers had already
been sampled. The parameters for this experiment were $N$ = 5000, $m$ = 2,
$\alpha$ = 0.5, and $I$ = 2000. The estimated values were $\alpha$, 
  0.3 
  [0 -
   0.63];
$I$,
  2449 
  [1417 -
   3811];
$m$,
  3 
  [2 -
   5];
$N$,
  9132 
  [2852 -
   14780].
\del{The approximate one-dimensional marginal posterior distributions are shown
in \cref{fig:peerdriven}.} Both of these results were in line with estimates
obtained on other simulated datasets (\cref{tab:abchpd}), although the estimate
of $\alpha$ for peer-driven sampling was somewhat lower than typical.



\add{\Cref{fig:marginal} shows the effect of performing marginal ABC estimation
of each of the BA parameters on the same simulated transmission tree. The
estimates of $m$ were apparently unaffected by marginalizing out the other
parameters, corroborating the previous experiments' findings that $m$ is not an
identifiable parameter from scaled tree shapes. Compared to allowing all
parameters to vary, estimates of $\alpha$, $I$, and $N$ were improved by
    41\%,
    59\%, and
    46\%
when all other parameters were fixed. \Cref{fig:better} shows the impact of
increasing the number of particles, simulated datasets, and
$\alpha_{\text{ESS}}$ parameter on the accuracy of a single simulation. The
number of iterations until the stopping condition was reached was 81 with the
basic settings and 124 with the higher settings. The results of the two
simulations were similar, but surprisingly, the results with higher SMC
settings were slightly worse (by
    10\%,
    8\%, and
    11\%
for $\alpha$, $I$, and $N$ respectively). However, the 50\% HPD interval for
$I$ was closer to the true value of 2000 with the improved settings
    (2338 - 
     3423, vs.
     2810 - 
     3767 with basic settings).
The estimate of $m$, 3 in both cases, was unaffected by the settings.}

\subsection*{Published HIV data}



We applied ABC to five published HIV datasets (\cref{tab:data}),
and found substantial heterogeneity among the parameter estimates
(\cref{fig:abchpd,fig:abchpdm2}). \add{Posterior mean point estimates and 50\%
and 95\% HPD intervals for each parameter are shown in \cref{fig:abchpd}.}
\del{Plots of the marginal posterior distributions for each dataset are shown
in \cref{fig:cuevas,fig:li,fig:niculescu,fig:novitsky,fig:wang}.}
\add{\Cref{fig:abchpdm2} shows point estimates and HPD intervals obtained when
the value $m = 1$ was disallowed by the prior. Since the results indicated that
$m = 1$ was the most credible value for several datasets, all results discussed
henceforth apply to the prior $m \sim \text{DiscreteUniform}(1, 5)$ unless
otherwise stated.}

\begin{figure}[ht]
    \centering
    \includegraphics{realdata-hpd}
    \caption[
        Posterior means and 50\%/95\% HPD intervals for parameters of the
        BA network model, fitted to ten HIV datasets with \textit{netabc}.
    ]{
        Posterior means (points), 50\% HPD intervals (notches), and 95\%
        HPD intervals (lines) for parameters of the BA network model, fitted to
        ten HIV datasets with \textit{netabc}. Legend labels indicate risk
        group and country of origin. Abbreviations: IDU, injection drug users;
        MSM, men who have sex with men; HET, heterosexual. Note that posterior
        means can fall outside of the HPD interval if the distribution is
        diffuse.
    }
    \label{fig:abchpd}
\end{figure}

\del{Two of the datasets~\autocite{niculescu2015recent, wang2015targeting} had
estimated $\alpha$ values near unity for the prior allowing $m = 1$ (MAP
estimates [95\% HPD] 
  0.73 
  [0.05 - 
   1.18]
and
  0.55 
  [0.01 -
   0.99] respectively).
The MAP estimates did not change appreciably when $m = 1$ was disallowed by the
prior, although the credible interval of the \textcite{niculescu2015recent}
data was narrower
  (0.05 - 
   1.18).
When $m = 1$ was permitted, the \textcite{li2015hiv, cuevas2009hiv} both had
low estimated $\alpha$ values
  (0.33 
  [0 - 
  0.76]
and
  0.27 
  [0 -
   0.59]). 
However, the MAP estimates increased when $m = 1$ was not permitted, although
the HPD intervals remained roughly the same
  (0.58 
  [0.06 - 
  0.99]
and
  0.48 
  [0.02 -
   0.87]).
The \textcite{novitsky2014impact} data had a fairly low estimated $\alpha$
for both priors on $m$
  (0.55 for $m \geq 1$;
   0.53 for $m \geq 2$).
However, the confidence interval was much wider when $m = 1$ was allowed
  ([0 -
    1.75] for $m \geq 1$ vs.
    0 -
    1.75 for $m \geq 2$).}



\add{Posterior mean point estimates for the preferential attachment power $\alpha$
were all sub-linear, ranging from 
    0.27
(mixed/Spain) to
    0.83
(IDU/Estonia). When aggregated by risk group, the average estimates were
    0.78
for IDU,
    0.41
for primarily heterosexual risk, and
    0.37
for MSM. 95\% HPD intervals were very wide for most datasets, often
encompassing nearly the entire range from 0 to 1 (\cref{fig:abchpd}). As shown
in \cref{fig:genes}, the estimates of $\alpha$ were quite robust to the gene
analyzed.} 

\del{For all the datasets except \citeauthor{novitsky2014impact}, estimated values
of $I$ were below 2000 when $m = 1$ was allowed, with relatively narrow HPD
intervals compared to the nonzero prior density region
  (\citeauthor{cuevas2009hiv}, 701 
  [289 -
   1279];
   \citeauthor{niculescu2015recent}, 747
  [136 - 
   2378];
  \citeauthor{li2015hiv}, 1390 
  [310 -
   2821];
   \citeauthor{wang2015targeting}, 675
  [175 - 
   1400]).
The \citeauthor{novitsky2014impact} data was the outlier, with a very high
estimated $I$, and HPD interval spanning almost the entire prior region
  (5431 
  [183 -
   8739]).
The $I$ estimates and HPD intervals were generally robust to the choice of
prior on $m$, with slightly narrower HPD intervals (compare
\cref{fig:abchpd,fig:abchpdm2}).}



\add{For all but the HET/Botswana data, the posterior mean estimates for
$I$ were between
    373
(IDU/Estonia) and
    1391
(MSM/Shanghai). The HET/Botswana data had a much higher estimated $I$ value
    (5432)
than the other datasets, with a very wide 95\% HPD interval covering almost the
entire prior region (\cref{fig:abchpd}). There was no significant correlation
between the number of sequences in the tree and the estimated $I$ value
(Spearman correlation, 
    $p =0.9$),
indicating that the higher estimates were not simply due to increased sampling
density. When both \textit{gag} and \textit{env} sequences were analyzed, the
estimates from the \textit{env} data were higher (HET/Uganda,
    939 for \textit{gag} vs.
    1615 for \textit{env};
HET/Malawi,
    724 for \textit{gag} vs.
    845 for \textit{env}).}

\del{The MAP estimate of $m$ was equal to 1 for all but the
\citeauthor{novitsky2014impact} data, when this value was allowed. However, the
upper bound of the HPD interval was different for each dataset
  (\citeauthor{niculescu2015recent}, 4;
   \citeauthor{wang2015targeting}, 1;
   \citeauthor{li2015hiv}, 1;
   \citeauthor{cuevas2009hiv}, 1).
When $m = 1$ was disallowed, the MAP for all datasets was either 2 or 3, with
HPD intervals spanning the entire prior region. The estimates for the total
number of nodes $N$ were largely uninformative for all samples, with almost all
MAP estimates greater than 7500 and HPD intervals spanning almost the entire
nonzero prior density region. The only exception was the \citeauthor{li2015hiv}
data, for which the MAP estimate was lower 
  (5916)
when $m = 1$ was allowed.}

\add{The posterior means of $m$ were equal to one for 
  zero
of the datasets analyzed. The widths of the 95\% HPD intervals varied from 0
(all the mass on the estimated value) to 5 (the entire prior region). Estimates
of $N$ were  mostly uninformative, with very similar estimates for all datasets
(mean
    6202,
range
    5881 - 6882).
This was similar to the pattern observed for the synthetic data, where the
posterior mean always fell around the upper two-thirds mark of the range
(\cref{fig:abcpt2}).}

\add{When the value $m = 1$ was disallowed by the prior, the separation in
$\alpha$ between the IDU datasets and the others became more striking
(\cref{fig:abchpdm2}). Both IDU datasets had estimated $\alpha$ values at or
above
    1.
The estimate for the MSM/Beijing data was slightly lower
    (0.85) and
the estimates for the seven remaining non-IDU datasets were bounded above by
    0.58. 
The values of $I$ were fairly robust to the choice of prior (compare
\cref{fig:abchpd,fig:abchpdm2}), although the 95\% HPD intervals were slightly
narrower (average width
    2159
for $m \geq 1$ and
    1874
for $m \geq 2$). The posterior means of $m$ for all but the HET/Botswana data
took on the value 3 with this prior, with the HPD intervals spanning the entire
prior region. This is very similar to the results observed for $m$ on simulated
data (\cref{tab:abchpd}), and suggests that $m$ is not identifiable from these
data with this prior. The results for $N$ did not change appreciably between
the two choices of prior.}



\add{For the two datasets we reanalyzed using
RAxML~\autocite{stamatakis2014raxml} and LSD~\autocite{to2016fast}, $\alpha$
was relatively robust to the choice of method (\cref{fig:methods}, posterior
means 
    0.48
vs. 
    0.48
for mixed/Spain and 
    1.02
vs.
    1.12
for IDU/Estonia). However, the estimates of $I$ were about twice as high when
RAxML was used instead of FastTree to reconstruct the trees 
    (228 
     vs. 437 for IDU/Estonia, 
    816
    vs. 1949 for mixed/Spain). 
\Cref{fig:boot} shows estimates obtained for five bootstrap replicate
alignments for each of these two datasets. For the mixed/Spain data, the
estimated posterior mean [range of bootstrap posterior means] was
    0.48
    [0.54 -
     0.67]
for $\alpha$, 
    816
    [403 -
     886] for $I$,
    2.76
    [2.61 -
     3.24] for $m$, and
    6639
    [6652 -
     7245] for $N$. 
For the IDU/Estonia data, values were
    1.02
    [0.78 -
     1.07]
for $\alpha$, 
    228
    [313 -
     741] for $I$,
    3.11
    [3.41 -
     3.46] for $m$, and
    6803
    [5941 -
    6913] for $N$.}

\section*{Discussion}

Contact networks can have a strong influence on epidemic progression, and are
potentially useful as a public health tool~\autocite{wang2015targeting,
little2014using}. Despite this, few methods exist for investigating contact
network parameters in a phylodynamic framework \autocite[although see][for
related work]{groendyke2011bayesian, volz2008sir, brown2011transmission,
leventhal2012inferring}. ABC is a model-agnostic method which can be used to
investigate any quantity that affects tree
shape~\autocite{poon2015phylodynamic}. In this work, we developed a ABC-based
method to infer the parameters of a contact network model. The method is
general, and could be applied to any model from which contact networks can be
simulated. We demonstrated the method on the BA model, which is a simple
preferential attachment model. \add{For some parameter choices,} the BA model
gives rise to the power law degree distributions commonly observed in
real-world networks.

\subsection*{Analysis of BA model with synthetic data}

\del{By training a kernel-SVM classifier, we found that the $\alpha$ and $I$
parameters, representing preferential attachment power and number of infected
nodes, had a strong influence on tree shape. This was reflected in the relative
accuracy of the ABC estimates of these parameters. The total number of
nodes $N$ had a weak influence on tree shape, which was most prominent when the
epidemic size $I$ and number of sampled tips were both large. The $m$
parameter, representing the number of edges created in the network per vertex,
did not produce much variation in tree shape, resulting in in both poorly
performing classifiers and uninformative ABC estimates.}

\add{The preferential attachment power $\alpha$ had a strong influence on tree
shape in the range of values we considered. Although the tree kernel was the
most effective classifier for $\alpha$, a Sackin's index of tree imbalance
performed nearly as well (\cref{fig:rsquared}). High $\alpha$ values produce
networks with few well-connected ``superspreader'' nodes which are involved in
a large number of transmissions, resulting in a highly unbalanced ladder-like
tree structure. There appeared to be weaker identifiability for $\alpha < 1$
than for $\alpha \geq 1$ (\cref{fig:abcpt,tab:abchpd}), which may be partially
explained by the relationship between $\alpha$ and the power law exponent
$\gamma$ (\cref{fig:gamma}). Although the degree distributions do not truly
follow a power law for $\alpha \neq 1$, the fitted exponent still captures the
shape of the degree distribution reasonably well (\cref{fig:powerlaw}). The
$\gamma$ values fitted to $\alpha = 0$ and $\alpha = 0.5$ are nearly identical
(about 2.28 for $\alpha = 0$ and 2.33 for $\alpha = 0.5$ with $N = 5000$ and $m
= 2$). In other words, the degree distributions of networks with $\alpha < 1$
are similar to each other, which may result in similarity of corresponding
transmission trees as well.}

\add{$I$, representing the number of infected individuals at the time of
sampling, was also identifiable, albeit over-estimated with ABC for both values
we considered. Sackin's index was better able to discern $I$ from tree shape
than the nLTT (\cref{fig:rsquared,fig:Icrossv}), suggesting that this parameter
impacts the distribution of branching times in the tree more than the topology.
In a homogeneously-mixed population, branching times can be modelled by the
coalescent process~\autocite{kingman1982coalescent}, in our case under the SI
model~\autocite{volz2009phylodynamics}. Although networks are not homogeneously
mixed, the forces which affect the distribution of branching times still apply.
In our simulations, all discordant edges shared the same
transmission rate, so that the waiting time until the next transmission in the
entire network was always inversely proportional to the number of discordant
edges. In the initial phase of the epidemic, when $I$ is small, each new
transmission results in many new discordant edges. Hence, there is an early
exponential growth phase, producing many short branches near the root of the
tree. As the epidemic gets closer to saturating the network, the number of
discordant edges decays, causing longer waiting times.}

\add{The number of nodes in the network, $N$, exhibited the most variation in
terms of its effect on tree shape. There was almost no measurable difference
between trees simulated under different $N$ values when the number of infected
nodes $I$ was small (\cref{fig:Ncrossv}).} In retrospect, it is unreasonable to
expect good estimation of $N$, in many cases, because adding additional nodes
does not change the edge density or overall shape of a BA network. This can be
illustrated by imagining that we add a small number of nodes to a network after
the epidemic simulation has already been completed. \add{If $I$ is small
relative to $N$, very few of the infected nodes will gain any new neighbours.
Thus, the outcome of a second simulation on the same network will likely be
very similar.} \del{It is possible that none of these new nodes attains a
connection to any infected node. Thus, running the simulation again on the new,
larger network could produce the exact same transmission tree as before.}
On the other hand, when $I$ is large relative to $N$, the coalescent dynamics
discussed above also apply. The waiting times until the next infection
increase, resulting in longer coalescence times toward the tips. The relative
accuracy of the nLTT in these situations (\cref{fig:rsquared,fig:Ncrossv})
corroborates this hypothesis, as the nLTT uses only information about the
coalescence times. When all BA parameters were simultaneously estimated with
ABC, $N$ was nearly always over-estimated by approximately a factor of two
(\cref{fig:abcpt2,tab:abchpd}). One factor which may have contributed to this
bias was our choice of prior distribution. Since the prior for $I$ and $N$ was
jointly uniform on a region where $I \leq N$, more prior weight was assigned to
higher $N$ values. We note also that \del{our accurate estimates of $I$ may
have been influenced by} this prior\del{, which} places more mass on low $I$
values. However, the estimate of $I$ was very high for the HET/Botswana data,
suggesting that a strong enough signal in the data can overcome the prior.
\add{Furthermore, when $I$ was estimated marginally with fixed $N$, the
accuracy of the estimate improved even though there was no longer any extra
prior mass on low $I$ values.}

\add{Another possible contributing factor to the overestimation of $I$ and $N$
relates to the dynamics of the SI model and the coalescent process. The number
of infected individuals follows a logistic growth curve under the SI model.
This kind of growth curve has three qualitative phases: a slow ramp-up, an
exponential growth phase, and a slow final phase when the susceptible
population is almost depleted. The waiting times until the next transmission,
which determine the coalescence times in the tree, are dependent on the growth
phase of the epidemic. Therefore, we hypothesize that it is the growth phase at
the time of sampling which most affects tree shape, rather than the specific
values of $I$ or $N$. To investigate this hypothesis, we simulated transmission
trees over networks on a grid of $I$ and $I$ values and fit logistic growth
curves to the proportion of infected individuals over time. We and calculated
the first and second derivatives of these curves (with respect to time) at the
time of transmission tree sampling. As shown in \cref{fig:derivs}, there are
bands along which both derivatives are similar which contain the values we
tested. These bands span mostly higher values of $N$ and $I$ than the true
values. Thus, if $N$ and $I$ are free to vary (as is the case in ABC), both
parameters will tend to be overestimated due to being less identifiable within
their own band. However, when $N$ is fixed at 5000, the derivatives vary
substantially along the $I$-axis, which explains why a marginal estimate of $I$
was more accurate (\cref{fig:marginal}). We also note the resemblance of the
contour surface of \cref{fig:derivs} to the two-dimensional marginal posterior
distribution on $I$ and $N$ obtained with simulated data (\cref{fig:abc2}).}

\add{The $m$ parameter, which controls the number of connections added to the
network per vertex, did not have a measurable impact on tree shape and was not
identifiable with ABC. It was pointed out to us by an anonymous reviewer that
for a fixed $I$, an infected node may only end up transmitting along a fraction
of its outgoing edges, which could mask the presence of the extra edges
associated with higher $m$. If $m$ were a continuous variable, it is possible
that we would observe stronger identifiability for lower values (say between 0
and 2), where extra edges are more likely to be involved in the epidemic and
have an impact on the transmission tree. One way to achieve this would be to
draw $m$ separately for each node from a distribution parameterized by a
continuous variable.}

As noted by \textcite{lintusaari2016identifiability}, uniform priors on model
parameters may translate to highly informative priors on quantities of
interest. We observed a non-linear relationship between the preferential
attachment power $\alpha$ and the power law exponent $\gamma$
(\cref{fig:gamma}). Therefore, placing a uniform prior on $\alpha$ between 0
and 2 is equivalent to placing an informative prior that $\gamma$ is close to
2. Therefore, if we were primarily interested in $\gamma$ rather than
$\alpha$, a more sensible choice of prior might have a shape informed by
\cref{fig:gamma} and be bounded above by approximately $\alpha$ = 1.5. This
would uniformly bound $\gamma$ in the region $2 \leq \gamma \leq 4$ commonly
reported in the network literature~\autocite{liljeros2001web,
schneeberger2004scale, colgate1989risk, brown2011transmission}. We note however
that \textcite{jones2003assessment} estimated $\gamma$ values greater than
four for some datasets, in one case as high as 17, indicating that a wider
range of permitted $\gamma$ values may be warranted.

\subsection*{Analysis of real world HIV datasets}

Our investigation of published HIV datasets indicated heterogeneity in the
contact network structures underlying several distinct local epidemics. When
interpreting these results, we caution that the BA model is quite simple and
most likely misspecified for these data. In particular, the average degree of a
node in the network is equal to $2m$, and therefore is constrained to be a
multiple of 2. Furthermore, we considered the case $m = 1$, where the network
has no cycles, to be implausible and therefore assigned it zero prior
probability in one set of analyses. This forced the average degree to be at
least four, which may be unrealistically high for sexual networks. The fact
that the estimated values of $\alpha$ differed substantially for several datasets
depending on whether or not $m = 1$ was allowed by the prior is further evidence
of this potential misspecification. However, we note that \add{the ordering of
the datasets with respect to $\alpha$ was similar between the two priors,}
\del{for two of the datasets, the estimated values of $\alpha$ did not change
much between priors,} and the estimates of $I$ were robust to the choice of
prior for all datasets studied (compare \cref{fig:abchpd,fig:abchpdm2}). More
sophisticated models, for example models incorporating heterogeneity in node
behaviour, are likely to provide a better fit to these data.



\del{With respect to the preferential attachment power $\alpha$, the five
datasets analyzed fell into three categories (\cref{fig:abchpd}). First, we
estimated a preferential attachment power close to 1, indicating linear
preferential attachment, for the outbreaks studied by
\textcite{niculescu2015recent} and \textcite{wang2015targeting}. These values
were robust to specifying different priors for $m$. Both studies were of
populations in which we would expect a high degree of epidemiological
relatedness: \textcite{niculescu2015recent} studied a recent outbreak among
Romanian injection drug users (IDU), while \citeauthor{wang2015targeting}
sampled acutely infected MSM in Beijing, China. Both these are contexts in
which we would expect some of the assumptions of the BA model, such as a
connected network, relatively high mean degree, and preferential attachment
dynamics, to hold.}

\del{The remaining three datasets (\textcite{cuevas2009hiv, novitsky2014impact,
li2015hiv}) had estimated values of $\alpha$ below 0.5 when $m = 1$ was
included in the prior, but these were not robust to changing the prior to
exclude $m = 1$. For the \citeauthor{cuevas2009hiv} data, model
misspecification is likely partially responsible. While the authors found that
a large proportion of the samples were epidemiologically linked, these were
mainly in small local clusters rather than the single large component
postulated by the BA model. In addition, the mixed risk groups in the dataset
would be unlikely to significantly interact, further weakening any global
preferential attachment dynamics. The dataset studied by
\textcite{novitsky2014impact} originated from a densely sampled population
where the predominant risk factor was believed to be heterosexual exposure.
Although the MAP estimate of $\alpha$ was almost unchanged when the value $m =
1$ was excluded from the prior, the confidence interval shrank significantly.
For both priors, the estimated $I$ value was extremely high, in fact higher
than the estimated HIV prevalence in the sampled region. The authors indicated
that the source of the samples was a town in close proximity to the country's
capital city, and suggested that there may have been a high degree of migration
and partner interchange between the two locations. It is possible that the
contact network underlying the subtree we investigated includes a much larger
group based in the capital city, which would explain the high estimate of $I$.
There is no clear explanation for the discrepancy between the two priors for
the \textcite{li2015hiv} data, as the subset we analyzed formed a phylogenetic
cluster and therefore was a good candidate for the BA model. However, nearly
all the posterior density was assigned to $m = 1$ when this value was allowed,
indicating that the network was more likely to have an acyclic tree structure.}

\subsubsection*{Preferential attachment power is sub-linear and higher for IDU networks}

\add{For all datasets we examined, the posterior mean estimates for $\alpha$
were sub-linear, ranging from 
    0.27
to 
    0.83.
The sub-linearity is consistent with the results of
\textcite{de2007preferential}, who developed a statistical inference method to
estimate the parameters of a more sophisticated preferential attachment model
incorporating heterogeneous node behaviour. When used to analyze
population-level longitudinal partner count data, they found $\alpha$ values
ranging from 0.26 to 0.62 depending on the gender and time period considered.}

\add{Both \textcite{de2007preferential} and the HET/Botswana data studied
populations whose primary risk factor for HIV infection was heterosexual
contact. \citeauthor{de2007preferential} explicitly excluded reported
homosexual contacts; \citeauthor{novitsky2014impact} did not, but noted that
heterosexual contact is the primary mode of transmission in Botswana where the
study was done. In the first of the two papers describing the Botswana
study~\autocite{novitsky2013phylogenetic}, the authors noted that their sample
was gender-biased, being composed of approximately 75\% women. Our estimate of
$\alpha$ for these data was 
    0.55
        or 
    0.53,
depending on the prior on $m$. Similarly, \textcite{de2007preferential}
estimated 0.54, 0.57, and 0.29 for 3-year, 5-year, and lifetime partnership
networks respectively for the female portion of their sample.}

\add{The datasets derived from IDU populations had a higher estimated
preferential attachment power than the other datasets
(\cref{fig:abchpd,fig:abchpdm2}). This finding is in line with
\textcite{dombrowski2013topological}, who reanalyzed a network of IDUs in
Brooklyn, USA, collected between 1991 and 1993~\autocite{friedman2006social}.
They found that the IDU network resembled a BA network much more closely than
other social and sexual networks, and offered sociological explanations for the
apparent preferential attachment dynamics in this population. Importantly, from
a public health perspective, the authors asserted that the removal of
\emph{random} individuals from IDU networks may have the paradoxical effect of
increasing the network's epidemic susceptibility. When low-degree nodes are
removed, as would occur during a police crackdown, their network neighbours may
turn to well-known community members for advice or supplies, thus increasing
the connectivity of these high-degree nodes.}

\add{Unfortunately, the sub-linear region for $\alpha$ identified by both
\textcite{de2007preferential} and \textit{netabc} is also the region of
poorest identifiability (\cref{fig:abcpt}). This was reflected in the high
level of uncertainty in the estimates, with most 95\% HPD intervals covering
the majority of the range [0, 1]. The value $\alpha = 0.5$ was contained in the
95\% HPD interval for every dataset; consequently, it is not possible to say
with high confidence that any of the $\alpha$ values are different from each
other. In synthetic data, the confidence intervals around $\alpha$ narrowed
when other parameters were marginalized out (\cref{fig:marginal}). Thus, it is
possible that estimates of $\alpha$ could be made more precise by specifying
either exact values or informative priors on the other BA parameters when these
are known.}

\subsubsection*{Other BA parameters}



\add{The true HIV prevalence in a population can be difficult to estimate for
several reasons. HIV-infected individuals may be asymptomatic for months or
years, possibly delaying their awareness of their status. In many contexts, the
risk factors for acquisition of HIV are illegal or stigmatized, which may
represent a barrier to testing, treatment, and/or disclosure of status.
Our simulation study showed that $I$ is weakly identifiable from tree shapes,
however, the estimates of $I$ obtained with \textit{netabc} were upwardly
biased (\cref{fig:abcpt}). In addition, our initial exploratory analysis showed
that the identifiability of $I$ decreases with the number of sampled tips
(\cref{fig:Icrossv}); in real world studies, the proportion of infected
individuals sampled is usually low. The estimated $I$ values for the HIV
datasets ranged from
    373
    (IDU/Estonia)
to
    5432
(HET/Botswana).}

\add{We were not able to discern and trends toward over- or underestimation of
$I$ from available prevalence data. For example, the authors of the
HET/Botswana data~\autocite{novitsky2013phylogenetic, novitsky2014impact}
estimated that there were 1731 HIV-positive individuals in the study
area. HIV sequences were obtained from approximately 70\% of these individuals.
The estimated prevalence we obtained was much higher 
    (5432),
with a 95\% HPD interval spanning nearly the entire prior region. On the other
hand, the study which produced the MSM/USA data~\autocite{little2014using}
enrolled 648 HIV-positive MSM; thus, our estimate of $I$
    (482)
was clearly an underestimate. Post-hoc explanations can be imagined for both of
these results. The HET/Botswana data were collected from a town located
proximally to the country's capital city; the authors suggested that frequent
travel between the two locations may have facilitated linking of their sexual
networks. Thus, the high estimate of $I$ we obtained for these data may
include a larger network component based in the capital city. The MSM/USA
result may indicate that the individuals genotyped for the study are members of
a smaller subnetwork which does not include the entire local MSM population.
Unfortunately, none of these hypotheses can be easily tested.}

\add{Over half of the datasets were estimated to have $m = 1$, which produces
tree-like networks without cycles. Since the average degree of a
Barab\'asi-Albert network is $2m$, this value may simply reflect the fact that
most people have a small number of sexual partners, especially when only
recent partnerships are considered~\autocite{liljeros2001web}. In fact, in one
survey, the most common number of partnerships in the past twelve months was
one~\autocite{liljeros2001web}; the BA model does not allow any nodes with
degree 1 when $m \geq 2$. For this reason, the choice of whether to allow $m =
1$ in the prior is problematic, as we must choose between an unrealistic
topology (no cycles) and an unrealistic minimum degree. Extensions to the BA
model which relax this constraint can be imagined and may offer improved
parameter resolution. Estimates of $N$ were not informative for any of the
datasets under either choice of prior, consistent with our simulation results.}

\subsubsection*{Modelling assumptions}

In addition to the aforementioned possibility of misspecification, additional
modelling assumptions include the network being connected and static, all
transmission rates being equal, no removal after infection, identical behaviour
of all nodes, and random sampling. The last two were addressed with small-scale
experiments. We simulated a network where some nodes exhibited a higher
attachment power than others, and found that the estimated attachment power was
simply the average of the two values. This indicated that, although we could
characterize the network in aggregate, the estimated parameters could not be
said to apply to any individual node. The effect of biased sampling was
investigated by analyzing a transmission tree which had been sampled in a
peer-driven fashion. The results were roughly in line with those for random
sampling, however the estimated value of $\alpha$ was lower than the average
for randomly-sampled trees. Further experiments would be necessary to fully
explore the impact of these assumptions on the method's accuracy. However,
despite these issues, we felt it was best to demonstrate the method first on a
simple model. It is possible to use this framework to fit more complex models
which address some of these issues, such as one incorporating heterogeneous
node behaviour, which may prove a fruitful avenue
for future investigations.

Our method has a number of caveats, perhaps the most significant being that it
takes a transmission tree as input. In reality, true transmission trees are not
available and must be approximated, often by way of a viral phylogeny. Although
this has been demonstrated to be a fair
approximation~\autocite[e.g.][]{leitner1996accurate}, and is frequently used in
practice~\autocite[e.g.][]{stadler2013uncovering}, the topologies of a viral
phylogeny and transmission tree can differ
significantly~\autocite{ypma2013relating} due to within-host evolution and the
sampling process~\autocite{giardina2016inference}. The ABC-SMC algorithm is
computationally intensive, taking about a day when run on 20 cores in parallel
with the settings we described in the methods. Nevertheless, our method is
potentially useful to epidemiological researchers interested in the general
characteristics of the network structure underlying disease outbreaks. This
work, and previous work by our group~\autocite{poon2015phylodynamic}, has
demonstrated that ABC is a broadly applicable and effective framework in which
to perform phylodynamic inference.

\section*{Acknowledgements}

\add{We are grateful to Dr. Sally Otto, Dr. Alexandre Bouchard-C\^ot\'e, 
Dr. Richard Harrigan, and the two anonymous reviewers for many helpful
suggestions.} This work was supported by grants from the Canadian Institutes of
Health Research (CIHR, operating grant HOP-111406), and the Bill \& Melinda
Gates Foundation (award number OPP1110049). R.M.M. was supported by a
scholarship from the CIHR Strategic Training Program in Bioinformatics.
A.F.Y.P. was supported by a CIHR New Investigator Award (Canadian HIV Vaccine
Initiative, Vaccine Discovery and Social Research) and by a Career Investigator
Scholar Award from the Michael Smith Foundation for Health Research, in
partnership with the Providence Health Care Research Institute and St. Paul's
Hospital Foundation.

\clearpage
\printbibliography
\newpage

\section*{Supplementary Figures}

\renewcommand{\thefigure}{S\arabic{figure}}
\setcounter{figure}{0}
\setcounter{page}{1}

\begin{figure}[ht]
  \centering
  \includegraphics{leventhal2012fig1.pdf}
  \caption{
    Reproduction of Figure 1A from Leventhal \textit{et al.} (2012) used to
    check the accuracy of our implementation of Gillespie simulation.
    Transmission trees were simulated over three types of network, with
    pathogen transmissibility varying from 0 to 1 (1000 trees per
    transmissibility value). Sackin's index was calculated for each simulated
    transmission tree. Lines are means, light shaded areas are 95\% quantile
    range, and dark shaded areas are interquartile ranges. 
    }
  \label{fig:leventhal}
\end{figure}

\begin{figure}[ht]
  \centering
  \includegraphics{smc-test.pdf}
  \caption{Approximation of mixture of Gaussians used by
    Del Moral \textit{et al.} (2012) and Sisson \textit{et al.} (2009) to test
    SMC. Solid black line indicates true distribution. Grey shaded area shows
    SMC approximation obtained with our implementation, using 10000
    particles with one simulated data point per particle.
  }
  \label{fig:smctest}
\end{figure}

\begin{figure}[ht]
  \centering
  \includegraphics{kernel-alpha-crossv.pdf}
  \caption{
    Cross-validation accuracy of kernel-SVM classifiers for $\alpha$ parameter
    of BA network model, for various tree kernel meta-parameters and
    epidemic scenarios. Each point was calculated based on 300 simulated
    transmission trees over networks with $\alpha$ = 0.5, 1.0, or 1.5. Dotted
    and and dashed lines indicate, respectively, performance of SVMs using the
    nLTT statistic and Sackin's index. Facets are number of infected nodes
    before the simulation was stopped ($I$) and number of tips in the sampled
    transmission tree.
  }
  \label{fig:alphacrossv}
\end{figure}

\begin{figure}[ht]
  \centering
  \includegraphics{kernel-m-crossv.pdf}
  \caption{Cross-validation accuracy of kernel-SVM classifiers for $m$
      parameter of BA network model, for various tree kernel
      meta-parameters and epidemic scenarios. Each point was calculated based
      on 300 simulated transmission trees over networks with $m$ = 2, 3, or 4.
      Dotted and and dashed lines indicate, respectively, performance of SVMs
      using the nLTT statistic and Sackin's index. Facets are number of
      infected nodes before the simulation was stopped ($I$) and number of tips
      in the sampled transmission tree.
  }
  \label{fig:mcrossv}
\end{figure}

\begin{figure}[ht]
  \centering
  \includegraphics{kernel-I-crossv.pdf}
  \caption{Cross-validation accuracy of kernel-SVM classifiers for number of
      infected nodes ($I$) under BA network model, for various tree
      kernel meta-parameters and two tree sizes. Each point was calculated
      based on 300 simulated transmission trees over networks with $I$ = 500,
      1000, or 2000. Dotted and and dashed lines indicate, respectively,
      performance of SVMs using the nLTT statistic and Sackin's index. Facets
      are the number of tips in the sampled transmission tree.
  }
  \label{fig:Icrossv}
\end{figure}

\begin{figure}[ht]
  \centering
  \includegraphics{kernel-N-crossv.pdf}
  \caption{Cross-validation accuracy of kernel-SVM classifiers for total number
      of nodes ($N$) under BA network model, for various tree kernel
      meta-parameters and epidemic scenarios sizes. Each point was calculated
      based on 300 simulated transmission trees over networks with $N$ = 3000,
      5000, or 8000. Dotted and and dashed lines indicate, respectively,
      performance of SVMs using the nLTT statistic and Sackin's index. Facets
      are the number of tips in the sampled transmission tree.
  }
  \label{fig:Ncrossv}
\end{figure}

\begin{figure}
    \includegraphics[width=\textwidth]{abc-boxplot-mN}
    \caption[
        Posterior mean point estimates for BA model parameters $m$ and $N$
        obtained by running \textit{netabc} on simulated data, stratified by
        true parameter values.
    ]{
        Posterior mean point estimates for BA model parameters $m$ and $N$
        obtained by running \textit{netabc} on simulated data, stratified by
        true parameter values. First row of plots contains true versus
        estimated values of $m$; second row contains true versus estimated
        values of $N$. Columns are stratified by $\alpha$, $I$, and $m$
        respectively. Dashed lines indicate true values. 
    }
    \label{fig:abcpt2}
\end{figure}

\begin{figure}
    \includegraphics[width=\textwidth]{abc-marginal}
    \caption[
        Posterior means, 50\% HPD intervals, and 95\% HPD intervals for BA
        model parameters estimated marginally with ABC.
    ]{
        Posterior means (points), 50\% HPD intervals (notches), and 95\% HPD
        intervals (lines) for BA model parameters estimated marginally with
        ABC. $x$-axis labels indicate parameters which were fixed to their true
        values by specifying Dirac-delta priors.
    }
    \label{fig:marginal}
\end{figure}

\begin{figure}
    \centering
    \includegraphics[width=0.7\textwidth]{better-smc}
    \caption[
        Posterior means, 50\% HPD intervals, and 95\% HPD intervals for BA
        model parameters estimated with ABC using two sets of SMC settings.
    ]{
        Posterior means (points), 50\% HPD intervals (notches), and 95\% HPD
        intervals (lines) for BA model parameter estimated with ABC using two
        sets of SMC settings. ``Low'' settings are 1000 particles, 5 simulated
        datasets per particle, and $\alpha_{\text{ESS}} = 0.95$. ``High''
        settings are 2000 particles, 10 simulated datasets per particle, and
        $\alpha_{\text{ESS}} = 0.97.$
    }
    \label{fig:better}
\end{figure}

\begin{figure}[ht]
  \centering
  \includegraphics{alpha-gamma.pdf}
  \caption{
      Relationship between preferential attachment power parameter $\alpha$
      and power law exponent $\gamma$ for networks simulated under the BA
      network model with $N$ = 5000 and $m$ = 2.
  }
  \label{fig:gamma}
\end{figure}

\begin{figure}[ht]
    \centering
    \includegraphics{powerlaw.pdf}
    \caption{
        Best fit power law and stretched exponential curves for degree
        distributions of simulated BA networks for several values of $\alpha$
        and $m$.
    }
    \label{fig:powerlaw}
\end{figure}

\begin{figure}[ht]
  \includegraphics{realdata-hpd-m2}
  \vspace{8pt}
  \caption{
      Maximum \textit{a posteriori} point estimates and 95\% HPD intervals for
      parameters of the BA network model, fitted to five published HIV datasets
      with ABC. $x$-axes indicate regions of nonzero prior density. In
      particular, the prior on $m$ was DiscreteUniform(2, 5).
  }
  \label{fig:abchpdm2}
\end{figure}

\begin{figure}
    \centering
    \includegraphics[width=\textwidth]{realdata-genes}
    \caption[
        Posterior means and 50\%/95\% HPD intervals for parameters of the
        BA network model, fitted to two HIV datasets where both \textit{gag}
        and \textit{env} genes were sequenced.
    ]{
        Posterior means (points), 50\% HPD intervals (notches), and 95\%
        HPD intervals (lines) for parameters of the BA network model, fitted to
        two HIV datasets where both \textit{gag} and \textit{env} genes were
        sequenced.
    }
    \label{fig:genes}
\end{figure}

\begin{figure}[ht]
    \centering
    \includegraphics[width=\textwidth]{realdata-methods}
    \caption[
        Posterior means and 50\%/95\% HPD intervals for parameters of the
        BA network model, fitted to two HIV datasets using two phylogenetic
        reconstruction methods.
    ]{
        Posterior means (points), 50\% HPD intervals (notches), and 95\%
        HPD intervals (lines) for parameters of the BA network model, fitted to
        two HIV datasets using two phylogenetic reconstruction methods.
    }
    \label{fig:methods}
\end{figure}

\begin{figure}[ht]
    \centering
    \includegraphics[width=\textwidth]{realdata-bootstrap}
    \caption[
        Posterior means and 50\%/95\% HPD intervals for parameters of the
        BA network model, fitted to original and bootstrap replicate alignments
        of two HIV datasets.
    ]{
        Posterior means (points), 50\% HPD intervals (notches), and 95\%
        HPD intervals (lines) for parameters of the BA network model, fitted to
        original and bootstrap replicate alignments of two HIV datasets.
    }
    \label{fig:boot}
\end{figure}

\begin{figure}[ht]
  \centering
  \includegraphics{derivatives.pdf}
  \caption[
    First and second time derivatives of epidemic growth curves at time of
    sampling for various values of $I$ and $N$.
  ]{
    First and second time derivatives of epidemic growth curves at time
    of sampling for various values of $I$ and $I$. Networks were simulated
    under the BA model with $\alpha = 1.0$, $m = 2$, and $N$ varied along the
    values shown on the $x$-axis. Transmission trees were sampled at the time
    when $I$ nodes were infected ($y$-axis). Logistic growth curves were fit to
    epidemic trajectories derived from the transmission trees, and their first
    and second derivatives were calculated at the time of sampling. Contours
    show first derivatives, while colours indicate second derivatives. Values
    of $I$ and $N$ used in simulation experiments with ABC are indicated by
    diamonds.
  }
  \label{fig:derivs}
\end{figure}

\end{document}
