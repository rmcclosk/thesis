Models of the spread of disease in a population often make the simplifying
assumption that the population is homogeneously mixed, or is divided into
homogeneously mixed compartments. However, human populations have complex
structures formed by social contacts, which can have a significant influence on
the rate {\color{blue}\uline{and pattern}} of epidemic spread. Contact
{\color{red}\sout{network models}} {\color{blue}\uline{networks}} capture this
structure by explicitly representing each contact that could possibly lead to a
transmission. {\color{blue}\uline{Contact network models parameterize the
structure of these networks, but \ldots }}

We developed a method based on approximate Bayesian computation (ABC) for
estimating structural parameters of the contact network underlying an
observed viral phylogeny. The method combines adaptive sequential Monte Carlo
for ABC, Gillespie simulation for propagating epidemics though networks, and a
{\color{blue}\uline{previously developed}} kernel-based tree similarity score.
{\color{blue}\uline{Our method offers the potential to quantitatively
investigate contact network structure from phylogenies derived from viral
sequence data, as a complementary approach to traditional epidemiological
investigation.}}

We used the method to {\color{red}\sout{fit}}
{\color{blue}\uline{investigate}} the Barab\'{a}si-Albert network model.
{\color{blue}\uline{This model incorporates the mechanism of preferential
attachment observed in real world social and sexual networks, whereby
individuals with a higher number of connections attract new contacts at an
elevated rate (``the rich get richer'').}}
{\color{red}\sout{to simulated transmission trees and applied it to viral
phylogenies estimated from six real-world HIV sequence datasets.}} Using
simulated data, we found that the strength of preferential attachment and the
number of infected nodes in the network could often be accurately estimated. On
the other hand, the mean degree of the network, as well as the total number of
nodes, appeared to be weakly or non-identifiable with ABC. 

Finally, the Barab\'{a}si-Albert model was fit to six real-world HIV datasets.
We observed substantial heterogeneity in the parameter estimates on real
datasets, with point estimates for the preferential attachment power ranging
from 0.06 to 1.05. 
{\color{blue}\uline{Point estimates of the strength of preferential attachment 
were higher in injection drug user populations, potentially indicating\ldots}}
{\color{red}\sout{These results underscore the importance of considering
contact structures when performing phylodynamic inference.}}
