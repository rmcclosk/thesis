Models of the spread of disease in a population often make the simplifying
assumption that the population is homogeneously mixed, or is divided into
homogeneously mixed compartments. However, human populations have complex
structures formed by social contacts, which can have a significant influence on
the rate of epidemic spread. Contact network models capture this structure by
explicitly representing each contact which could possibly lead to a
transmission. We developed a method based on kernel-assisted approximate
Bayesian computation (ABC) for estimating structural parameters of the contact
network underlying an observed viral phylogeny. The method combines adaptive
sequential Monte Carlo for ABC, Gillespie simulation for propagating epidemics
though networks, and a kernel-based tree similarity score. We used the method
to fit the Barab\'{a}si-Albert network model to simulated transmission trees,
and also applied it to viral phylogenies estimated from five published HIV
sequence datasets. On simulated data, we found that the preferential attachment
power and the number of infected nodes in the network can often be accurately
estimated. On the other hand, the mean degree of the network, as well as the
total number of nodes, appeared to be weakly or non-identifiable with
kernel-assisted ABC. We observed substantial heterogeneity in the parameter
estimates on real datasets, with point estimates for the preferential
attachment power ranging from 0.06 to 1.05. These results underscore the
importance of considering contact structures when performing phylodynamic
inference. Our method offers the potential to quantitatively investigate the
contact network structure underlying viral epidemics.
