Models of the spread of disease in a population often make the simplifying
assumption that the population is homogeneously mixed, or is divided into
homogeneously mixed compartments. However, human populations have complex
structures formed by social contacts, which can have a significant influence on
the rate {\color{blue}\uline{and pattern}} of epidemic spread. Contact
{\color{red}\sout{network models}} {\color{blue}\uline{networks}} capture this
structure by explicitly representing each contact that could possibly lead to a
transmission. {\color{blue}\uline{Contact network models parameterize the
structure of these networks, but estimating their parameters from contact data
requires extensive, often prohibitive, epidemiological investigation. }}

We developed a method based on approximate Bayesian computation (ABC) for
estimating structural parameters of the contact network underlying an
observed viral phylogeny. The method combines adaptive sequential Monte Carlo
for ABC, Gillespie simulation for propagating epidemics though networks, and a
{\color{blue}\uline{previously developed}} kernel-based tree similarity score.
{\color{blue}\uline{Our method offers the potential to quantitatively
investigate contact network structure from phylogenies derived from viral
sequence data, complementing traditional epidemiological methods.}}

We applied our method to {\color{red}\sout{fit}}
{\color{blue}\uline{investigate}} the Barab\'{a}si-Albert network model.
{\color{blue}\uline{This model incorporates the preferential attachment
mechanism observed in real world social and sexual networks, whereby
individuals with more connections attract new contacts at an elevated rate
(``the rich get richer'').}}
{\color{red}\sout{to simulated transmission trees and applied it to viral
phylogenies estimated from six real-world HIV sequence datasets.}} Using
simulated data, we found that the strength of preferential attachment and the
number of infected nodes could often be accurately estimated. However, the mean
degree of the network and the total number of nodes appeared to be weakly- or
non-identifiable with ABC. 

Finally, the Barab\'{a}si-Albert model was fit to six real world HIV datasets,
and substantial heterogeneity in the parameter estimates was observed.
{\color{red}\sout{Point estimates}}{\color{blue}\uline{Depending on the choice
of prior, posterior means}} for the preferential attachment power ranged from
{\color{red}\sout{0.06 to 1.05}} {\color{blue}\uline{0.26, slightly less than
logarithmic, to 1.00, exactly linear.}} {\color{blue}\uline{Point estimates of
the strength of preferential attachment were higher in two injection drug user
populations, potentially indicating that high-degree ''hub'' nodes may play a
role in epidemics among this risk group.}} Our results underscore the
importance of considering contact structures when {\color{red}\sout{performing
phylodynamic inference}}{\color{blue}\uline{investigating viral outbreaks.}}
