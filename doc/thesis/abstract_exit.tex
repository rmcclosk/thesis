Models of the spread of disease in a population often make the simplifying assumption that the population is homogeneously mixed, or is divided into homogeneously mixed compartments. However, human populations have complex structures formed by social contacts, which can have a significant influence on the rate and pattern of epidemic spread. Contact networks capture this structure by explicitly representing each contact that could possibly lead to a transmission. Contact network models parameterize the structure of these networks, but estimating their parameters from contact data requires extensive, often prohibitive, epidemiological investigation.

We developed a method based on approximate Bayesian computation (ABC) for estimating structural parameters of the contact network underlying an observed viral phylogeny. The method combines adaptive sequential Monte Carlo for ABC, Gillespie simulation for propagating epidemics though networks, and a previously developed kernel-based tree similarity score. Our method offers the potential to quantitatively investigate contact network structure from phylogenies derived from viral sequence data, complementing traditional epidemiological methods.

We applied our method to the Barab\'{a}si-Albert network model. This model incorporates the preferential attachment mechanism observed in real world social and sexual networks, whereby individuals with more connections attract new contacts at an elevated rate (``the rich get richer''). Using simulated data, we found that the strength of preferential attachment and the number of infected nodes could often be accurately estimated. However, the mean degree of the network and the total number of nodes appeared to be weakly- or non-identifiable with ABC.

 Finally, the Barab\'{a}si-Albert model was fit to eleven real world HIV datasets, and substantial heterogeneity in the parameter estimates was observed. Posterior means for the preferential attachment power were all sub-linear, consistent with literature results. We found that the strength of preferential attachment was higher in injection drug user populations, potentially indicating that high-degree ``superspreader'' nodes may play a role in epidemics among this risk group. Our results underscore the importance of considering contact structures when investigating viral outbreaks. 
