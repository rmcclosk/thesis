\begin{figure}[H]
  \centering
  \includegraphics[width=\textwidth]{kernel-I-tree.pdf}
  \caption[Simulated transmission trees under three different values of BA parameter $I$]{
    Simulated transmission trees under three different values of BA parameter
    $I$. Epidemics were simulated on \gls{BA} networks with parameters $\alpha$
    = 1.0, $m = 2$, and $N = 5000$. Epidemics were simulated until $I = 500$,
    1000, or 2000 nodes were infected. Transmission trees were created by
    sampling 500 infected nodes. For higher \gls{I} values, the network was
    closer to saturation at the time of sampling, resulting in longer terminal
    branches as the waiting time until the next transmission increased.
  }
  \label{fig:Itrees}
\end{figure}

\begin{figure}[H]
  \centering
  \includegraphics[width=\textwidth]{kernel-m-tree.pdf}
  \caption[Simulated transmission trees under three different values of BA parameter $m$]{
    Simulated transmission trees under three different values of BA parameter
    $m$. Epidemics were simulated on \gls{BA} networks with parameters $\alpha$
    = 1.0, $N = 5000$, and $m$ = 2, 3, or 4. Epidemics were simulated until $I
    = 1000$ nodes were infected. Transmission trees were created by sampling
    500 infected nodes.
  }
  \label{fig:mtrees}
\end{figure}

\begin{figure}[H]
  \centering
  \includegraphics[width=\textwidth]{kernel-N-tree.pdf}
  \caption[Simulated transmission trees under three different values of BA parameter $N$]{
    Simulated transmission trees under three different values of BA parameter
    $N$. Epidemics were simulated on \gls{BA} networks with parameters $\alpha$
    = 1.0, $m = 2$, and $N$ = 3000, 5000, or 8000. Epidemics were simulated
    until $I = 1000$ nodes were infected. Transmission trees were created by
    sampling 500 infected nodes. For lower $N$ values, the network was closer
    to saturation at the time of sampling, resulting in longer waiting times
    until the next transmission and longer terminal branch lengths.
  }
  \label{fig:Ntrees}
\end{figure}

\begin{figure}[H]
  \centering
  \includegraphics{kernel-alpha-crossv.pdf}
  \caption[Cross validation accuracy of classifiers for \gls{BA} model parameter
    \gls{alpha} for eight epidemic scenarios.]
  {
    Cross validation accuracy of classifiers for \gls{BA} model parameter
    \gls{alpha} for eight epidemic scenarios. Solid lines and points are
    $R^2$ of tree kernel \gls{kSVR} under various kernel meta-parameters.
    Dashed and dotted lines are $R^2$ of linear regression against Sackin's
    index, and \gls{SVR} using \gls{nltt}.
  }
  \label{fig:alphacrossv}
\end{figure}

\begin{figure}[H]
  \centering
  \includegraphics{kernel-I-crossv.pdf}
  \caption[Cross validation accuracy of classifiers for \gls{BA} model parameter
    \gls{I} for eight epidemic scenarios.]
  {
    Cross validation accuracy of classifiers for \gls{BA} model parameter
    \gls{I} for eight epidemic scenarios. Solid lines and points are $R^2$ of
    tree kernel \gls{kSVR} under various kernel meta-parameters. Dashed and
    dotted lines are $R^2$ of linear regression against Sackin's index, and
    \gls{SVR} using \gls{nltt}.
  }
  \label{fig:Icrossv}
\end{figure}

\begin{figure}[H]
  \centering
  \includegraphics{kernel-m-crossv.pdf}
  \caption[Cross validation accuracy of classifiers for \gls{BA} model parameter
    \gls{m} for eight epidemic scenarios.]
  {
    Cross validation accuracy of classifiers for \gls{BA} model parameter
    \gls{m} for eight epidemic scenarios. Solid lines and points are $R^2$ of
    tree kernel \gls{kSVR} under various kernel meta-parameters. Dashed and
    dotted lines are $R^2$ of linear regression against Sackin's index, and
    \gls{SVR} using \gls{nltt}.
  }
  \label{fig:mcrossv}
\end{figure}

\begin{figure}[H]
  \centering
  \includegraphics{kernel-N-crossv.pdf}
  \caption[Cross validation accuracy of classifiers for \gls{BA} model parameter
    \gls{N} for eight epidemic scenarios.]
  {
    Cross validation accuracy of classifiers for \gls{BA} model parameter
    \gls{N} for eight epidemic scenarios. Solid lines and points are $R^2$ of
    tree kernel \gls{kSVR} under various kernel meta-parameters. Dashed and
    dotted lines are $R^2$ of linear regression against Sackin's index, and
    \gls{SVR} using \gls{nltt}.
  }
  \label{fig:Ncrossv}
\end{figure}

\begin{figure}[H]
  \centering
  \includegraphics{kernel-alpha-kpca.pdf}
  \caption{
    Kernel principal components projection of trees simulated under three
    different values of \gls{BA} parameter \gls{alpha}, for eight epidemic
    scenarios.
  }
  \label{fig:alphakpca}
\end{figure}

\begin{figure}[H]
  \centering
  \includegraphics{kernel-I-kpca.pdf}
  \caption{
    Kernel principal components projection of trees simulated under three
    different values of \gls{BA} parameter \gls{I}, for eight epidemic
    scenarios.
  }
  \label{fig:Ikpca}
\end{figure}

\begin{figure}[H]
  \centering
  \includegraphics{kernel-m-kpca.pdf}
  \caption{
    Kernel principal components projection of trees simulated under three
    different values of \gls{BA} parameter \gls{m}, for eight epidemic
    scenarios.
  }
  \label{fig:mkpca}
\end{figure}

\begin{figure}[H]
  \centering
  \includegraphics{kernel-N-kpca.pdf}
  \caption{
    Kernel principal components projection of trees simulated under three
    different values of \gls{BA} parameter \gls{N}, for eight epidemic
    scenarios.
  }
  \label{fig:Nkpca}
\end{figure}

\begin{figure}[H]
  \centering
  \includegraphics{gridsearch-alpha-kernel.pdf}
  \caption{                                              
    Grid search kernel scores for testing trees simulated under various             
    \gls{alpha} values. The other \gls{BA} parameters were fixed at \gls{I} =
    1000, \gls{N} = 5000, and \gls{m} = 2. 
  }        
  \label{fig:gridalpha}
\end{figure}

\begin{figure}[H]
  \centering
  \includegraphics{gridsearch-I-kernel.pdf}
  \caption{                                              
    Grid search kernel scores for testing trees simulated under various             
    \gls{I} values. The other \gls{BA} parameters were fixed at \gls{alpha} =
    1.0, \gls{N} = 5000, and \gls{m} = 2. 
  }        
  \label{fig:gridI}
\end{figure}

\begin{figure}[H]
  \centering
  \includegraphics{gridsearch-m-kernel.pdf}
  \caption{                                              
    Grid search kernel scores for testing trees simulated under various             
    \gls{m} values. The other \gls{BA} parameters were fixed at \gls{alpha} =
    1.0, \gls{I} = 1000, and \gls{N} = 5000.
  }        
  \label{fig:gridm}
\end{figure}

\begin{figure}[H]
  \centering
  \includegraphics{gridsearch-N-kernel.pdf}
  \caption{                                              
    Grid search kernel scores for testing trees simulated under various             
    \gls{N} values. The other \gls{BA} parameters were fixed at \gls{alpha} =
    1.0, \gls{I} = 1000, and \gls{m} = 2.
  }        
  \label{fig:gridN}
\end{figure}

\begin{figure}[H]
  \centering
  \includegraphics{gridsearch-alpha-point-estimate.pdf}
  \caption[
      Point estimates of preferential attachment power \gls{alpha} of
      \acrlong{BA} network model, obtained on simulated trees with
      kernel-score-based grid search.]
  {
      Point estimates of preferential attachment power \gls{alpha} of
      \acrlong{BA} network model, obtained on simulated trees with
      kernel-score-based grid search. Test trees were simulated according to
      several values of \gls{alpha} ($x$-axis) with other model parameters
      fixed at \gls{m} = 2, \gls{N} = 5000, and \gls{I} = 1000. The test trees
      were compared to trees simulated along a narrowly spaced grid of
      \gls{alpha} values using the tree kernel, with the same values of the
      other parameters. The grid value with the highest median kernel score was
      taken as a point estimate for \gls{alpha} ($y$-axis).
  }
  \label{fig:gridptalpha}
\end{figure}

\begin{figure}[H]
  \centering
  \includegraphics{gridsearch-I-point-estimate.pdf}
  \caption[ Point estimates of prevalence at time of sampling \gls{I} of
      \acrlong{BA} network model, obtained on simulated trees with
      kernel-score-based grid search. ]
  {
      Point estimates of prevalence at time of sampling \gls{I} of \acrlong{BA}
      network model, obtained on simulated trees with kernel-score-based grid
      search. Test trees were simulated according to several values of \gls{I}
      ($x$-axis) with other model parameters fixed at \gls{alpha} = 1, \gls{m}
      = 2, and \gls{N} = 5000. The test trees were compared to trees simulated
      along a narrowly spaced grid of \gls{I} values using the tree kernel,
      with the same values of the other parameters. The grid value with the
      highest median kernel score was taken as a point estimate for \gls{I}
      ($y$-axis).
  }
  \label{fig:gridptI}
\end{figure}

\begin{figure}[H]
  \centering
  \includegraphics{gridsearch-m-point-estimate.pdf}
  \caption[
      Point estimates of number of edges per vertex \gls{m} of \acrlong{BA}
      network model, obtained on simulated trees with kernel-score-based grid
      search.
  ]
  {
      Point estimates of number of edges per vertex \gls{m} of \acrlong{BA}
      network model, obtained on simulated trees with kernel-score-based grid
      search. Test trees were simulated according to several values of \gls{m}
      ($x$-axis) with other model parameters fixed at \gls{alpha} = 1, \gls{I}
      = 1000, and \gls{N} = 5000. The test trees were compared to trees
      simulated along a narrowly spaced grid of \gls{m} values using the tree
      kernel, with the same values of the other parameters. The grid value with
      the highest median kernel score was taken as a point estimate for \gls{m}
      ($y$-axis).
  }
  \label{fig:gridptm}
\end{figure}

\begin{figure}[H]
  \centering
  \includegraphics{gridsearch-N-point-estimate.pdf}
  \caption[
      Point estimates of number of edges per vertex \gls{N} of \acrlong{BA}
      network model, obtained on simulated trees with kernel-score-based grid
      search.
  ]{
      Point estimates of number of edges per vertex \gls{N} of \acrlong{BA}
      network model, obtained on simulated trees with kernel-score-based grid
      search. Test trees were simulated according to several values of \gls{N}
      ($x$-axis) with other model parameters fixed at \gls{alpha} = 1, \gls{m}
      = 2, and \gls{I} = 1000. The test trees were compared to trees simulated
      along a narrowly spaced grid of \gls{N} values using the tree kernel,
      with the same values of the other parameters. The grid value with the
      highest median kernel score was taken as a point estimate for \gls{m}
      ($y$-axis).
  }
  \label{fig:gridptN}
\end{figure}

\begin{figure}[H]
  \centering
  \includegraphics{abc-point-estimate-m3.pdf}
  \caption[
    \Acrlong{MAP} point estimates for \gls{BA} model parameters obtained by
    running \software{netabc} on simulated data, for simulations with \gls{m} = 3.
  ]{
    \Acrlong{MAP} point estimates for \gls{BA} model parameters obtained by
    running \software{netabc} on simulated data, for simulations with \gls{m} = 3.
    Dashed lines indicate true values. (A) Estimates of \gls{alpha} and \gls{I}
    which were varied in these simulations against known values. (B) Estimates
    of \gls{m} and \gls{N} which were held fixed in these simulations at the
    values \gls{m} = 3 and \gls{N} = 5000.
  }        
  \label{fig:abcptm3}
\end{figure}

\begin{figure}[H]
  \centering
  \includegraphics{abc-point-estimate-m4.pdf}
  \caption[
    \Acrlong{MAP} point estimates for \gls{BA} model parameters obtained by
    running \software{netabc} on simulated data, for simulations with \gls{m} = 3.
  ]{
    \Acrlong{MAP} point estimates for \gls{BA} model parameters obtained by
    running \software{netabc} on simulated data, for simulations with \gls{m} = 4.
    Dashed lines indicate true values. (A) Estimates of \gls{alpha} and \gls{I}
    which were varied in these simulations against known values. (B) Estimates
    of \gls{m} and \gls{N} which were held fixed in these simulations at the
    values \gls{m} = 4 and \gls{N} = 5000.
  }        
  \label{fig:abcptm4}
\end{figure}

\begin{figure}[ht]
  \centering
  \includegraphics{alpha-gamma.pdf}
  \caption{
      Relationship between preferential attachment power parameter $\alpha$
      and power law exponent $\gamma$ for networks simulated under the BA
      network model with $N$ = 5000 and $m$ = 2.
  }
  \label{fig:gamma}
\end{figure}

\begin{figure}[ht]
  \centering
  \includegraphics{sir-trajectories.pdf}
  \caption[Average trajectories of epidemics simulated according to the
  \gls{SIR} model over full and \gls{BA} networks.]
  {
      Average trajectories of epidemics simulated according to the \gls{SIR}
      model over full and \gls{BA} networks. 100 epidemics each were simulated
      over completely connected networks and networks generated from the
      \gls{BA} model. The network parameters were $\alpha = 1$ and $m = 2$.
      The transmission rate along each edge of a network $G$ was taken as
      $2/|E(G)|$, and the removal rate of each vertex was taken as $1/|V(G)|$.
      These choices ensured that both networks had the same waiting time until
      the next transmission at the beginning of the epidemic.
  }
  \label{fig:sir}
\end{figure}

\begin{figure}[ht]
    \centering
    \includegraphics{mixed-posterior}
    \caption[Approximate marginal posterior distributions of BA model
        parameters obtained using kernel-ABC for a network with heterogeneous
        node behaviour.]
    {
        Approximate marginal posterior distributions of Barab\'asi-Albert
        model parameters obtained using kernel-ABC for a network with
        heterogeneous node behaviour. Half of the nodes were attached with
        $\alpha = 0.5$, and the other half with $\alpha = 1.5$ (vertical
        dashed lines, top left). Other parameter values were $m = 2$, $I =
        1000$, and $N = 5000$ (vertical dashed lines, other than top left).
        Shaded areas indicate 95\% highest posterior density intervals.
    }
    \label{fig:mixed}
\end{figure}

\begin{figure}[ht]
    \centering
    \includegraphics{peerdriven-posterior}
    \caption[Approximate marginal posterior distributions of Barab\'asi-Albert
        model parameters obtained using kernel-ABC for a network with
        peer-driven sampling.]
    {
        Approximate marginal posterior distributions of Barab\'asi-Albert
        model parameters obtained using kernel-ABC for a network with
        peer-driven sampling. An epidemic was simulated in the usual fashion,
        but rather than being sampled at random, infected nodes were sampled
        with a probability two times higher if they had any sampled neighbours
        in the contact network. Vertical dashed lines indicate true parameter
        values, and shaded areas indicate 95\% highest posterior density
        intervals.
    }
    \label{fig:peerdriven}
\end{figure}

\begin{figure}[ht]
  \includegraphics{realdata-hpd-bc-m2}
  \vspace{8pt}
  \caption[
      Maximum \textit{a posteriori} point estimates and 95\% HPD intervals for
      parameters of the BA network model, fitted to five published HIV datasets
      with \software{netabc} using the prior $m \sim DiscreteUniform(2, 5)$.
  ]{
      Maximum \textit{a posteriori} point estimates and 95\% HPD intervals for
      parameters of the BA network model, fitted to five published HIV datasets
      with \software{netabc} using the prior $m \sim$ DiscreteUniform(2, 5).
      $x$-axes indicate regions of nonzero prior density. In particular, the
      prior on $m$ was DiscreteUniform(2, 5).
  }
  \label{fig:abchpdm2}
\end{figure}

\begin{figure}[H]
  \includegraphics{bctree-posterior}
  \caption[
      Approximate marginal posterior distributions of BA model parameters for
      BC data.
  ]
  {
      Approximate marginal posterior distributions of BA model parameters for
      BC data. Vertical lines indicate maximum \textit{a posteriori} estimates,
      and shaded areas are 95\% highest posterior density intervals. $x$-axis
      indicates regions of nonzero prior density.
  }
  \label{fig:bctree}
\end{figure}

\begin{figure}[H]
  \includegraphics{cuevas2009-posterior}
  \caption[
      Approximate marginal poosterior distributions of BA model parameters for
      \textcite{cuevas2009hiv} data. 
  ]{
      Approximate marginal poosterior distributions of BA model parameters for
      \textcite{cuevas2009hiv} data. Vertical lines indicate maximum \textit{a
      posteriori} estimates, and shaded areas are 95\% highest posterior
      density intervals. $x$-axis indicates regions of nonzero prior density.
  }
  \label{fig:cuevas}
\end{figure}

\begin{figure}[H]
  \includegraphics{li2015-posterior}
  \caption[
      Approximate marginal posterior distributions of BA model parameters for
      \textcite{li2015hiv} data. 
  ]{
      Approximate marginal posterior distributions of BA model parameters for
      \textcite{li2015hiv} data. Vertical lines indicate maximum \textit{a
      posteriori} estimates, and shaded areas are 95\% highest posterior
      density intervals. $x$-axis indicates regions of nonzero prior density.
  }
  \label{fig:li}
\end{figure}

\begin{figure}[H]
  \includegraphics{niculescu2015-posterior}
  \caption[
      Approximate marginal posterior distributions of BA model parameters for
      \textcite{niculescu2015recent} data.
  ]{
      Approximate marginal posterior distributions of BA model parameters for
      \textcite{niculescu2015recent} data. Vertical lines indicate maximum
      \textit{a posteriori} estimates, and shaded areas are 95\% highest
      posterior density intervals. $x$-axis indicates regions of nonzero prior
      density.
  }
  \label{fig:niculescu}
\end{figure}

\begin{figure}[H]
  \includegraphics{novitsky2014-posterior}
  \caption[
      Approximate marginal posterior distributions of BA model parameters for
      \textcite{novitsky2013phylogenetic, novitsky2014impact} data.
  ]{
      Approximate marginal posterior distributions of BA model parameters for
      \textcite{novitsky2013phylogenetic, novitsky2014impact} data. Vertical
      lines indicate maximum \textit{a posteriori} estimates, and shaded areas
      are 95\% highest posterior density intervals. $x$-axis indicates regions
      of nonzero prior density.
  }
  \label{fig:novitsky}
\end{figure}

\begin{figure}[H]
  \includegraphics{wang2015-posterior}
  \caption[
      Approximate marginal posterior distributions of BA model parameters for
      \textcite{wang2015targeting} data.
  ]{
      Approximate marginal posterior distributions of BA model parameters for
      \textcite{wang2015targeting} data. Vertical lines indicate maximum
      \textit{a posteriori} estimates, and shaded areas are 95\% highest
      posterior density intervals. $x$-axis indicates regions of nonzero prior
      density.
  }
  \label{fig:wang}
\end{figure}

\begin{figure}[H]
\centering
\includegraphics{{abc-posterior/0.0_1000_2_5000_0}.pdf}
\caption[Approximate marginal posterior distributions of BA model parameters for a simulated transmission tree with $\alpha$ = 0.0, $I$ = 1000, $m$ = 2, and $N$ = 5000.]{
Approximate marginal posterior distributions of BA model parameters obtained by applying 	extit{netabc} to a simulated transmission tree with BA parameter values $alpha$ = 0.0, $I$ = 1000, $m$ = 2, and $N$ = 5000. Vertical dashed lines indicate true values. Shaded areas are 95\% highest posterior density intervals. $x$-axes indicate regions of nonzero prior density.
}
\label{fig:0.0-1000-2-5000-0}
\end{figure}
\clearpage

\begin{figure}[H]
\centering
\includegraphics{{abc-posterior/0.5_1000_2_5000_0}.pdf}
\caption[Approximate marginal posterior distributions of BA model parameters for a simulated transmission tree with $\alpha$ = 0.5, $I$ = 1000, $m$ = 2, and $N$ = 5000.]{
Approximate marginal posterior distributions of BA model parameters obtained by applying 	extit{netabc} to a simulated transmission tree with BA parameter values $alpha$ = 0.5, $I$ = 1000, $m$ = 2, and $N$ = 5000. Vertical dashed lines indicate true values. Shaded areas are 95\% highest posterior density intervals. $x$-axes indicate regions of nonzero prior density.
}
\label{fig:0.5-1000-2-5000-0}
\end{figure}
\clearpage

\begin{figure}[H]
\centering
\includegraphics{{abc-posterior/1.0_1000_2_5000_0}.pdf}
\caption[Approximate marginal posterior distributions of BA model parameters for a simulated transmission tree with $\alpha$ = 1.0, $I$ = 1000, $m$ = 2, and $N$ = 5000.]{
Approximate marginal posterior distributions of BA model parameters obtained by applying 	extit{netabc} to a simulated transmission tree with BA parameter values $alpha$ = 1.0, $I$ = 1000, $m$ = 2, and $N$ = 5000. Vertical dashed lines indicate true values. Shaded areas are 95\% highest posterior density intervals. $x$-axes indicate regions of nonzero prior density.
}
\label{fig:1.0-1000-2-5000-0}
\end{figure}
\clearpage

\begin{figure}[H]
\centering
\includegraphics{{abc-posterior/1.5_1000_2_5000_0}.pdf}
\caption[Approximate marginal posterior distributions of BA model parameters for a simulated transmission tree with $\alpha$ = 1.5, $I$ = 1000, $m$ = 2, and $N$ = 5000.]{
Approximate marginal posterior distributions of BA model parameters obtained by applying 	extit{netabc} to a simulated transmission tree with BA parameter values $alpha$ = 1.5, $I$ = 1000, $m$ = 2, and $N$ = 5000. Vertical dashed lines indicate true values. Shaded areas are 95\% highest posterior density intervals. $x$-axes indicate regions of nonzero prior density.
}
\label{fig:1.5-1000-2-5000-0}
\end{figure}
\clearpage

\begin{figure}[H]
\centering
\includegraphics{{abc-posterior/0.0_2000_2_5000_0}.pdf}
\caption[Approximate marginal posterior distributions of BA model parameters for a simulated transmission tree with $\alpha$ = 0.0, $I$ = 2000, $m$ = 2, and $N$ = 5000.]{
Approximate marginal posterior distributions of BA model parameters obtained by applying 	extit{netabc} to a simulated transmission tree with BA parameter values $alpha$ = 0.0, $I$ = 2000, $m$ = 2, and $N$ = 5000. Vertical dashed lines indicate true values. Shaded areas are 95\% highest posterior density intervals. $x$-axes indicate regions of nonzero prior density.
}
\label{fig:0.0-2000-2-5000-0}
\end{figure}
\clearpage

\begin{figure}[H]
\centering
\includegraphics{{abc-posterior/0.5_2000_2_5000_0}.pdf}
\caption[Approximate marginal posterior distributions of BA model parameters for a simulated transmission tree with $\alpha$ = 0.5, $I$ = 2000, $m$ = 2, and $N$ = 5000.]{
Approximate marginal posterior distributions of BA model parameters obtained by applying 	extit{netabc} to a simulated transmission tree with BA parameter values $alpha$ = 0.5, $I$ = 2000, $m$ = 2, and $N$ = 5000. Vertical dashed lines indicate true values. Shaded areas are 95\% highest posterior density intervals. $x$-axes indicate regions of nonzero prior density.
}
\label{fig:0.5-2000-2-5000-0}
\end{figure}
\clearpage

\begin{figure}[H]
\centering
\includegraphics{{abc-posterior/1.0_2000_2_5000_0}.pdf}
\caption[Approximate marginal posterior distributions of BA model parameters for a simulated transmission tree with $\alpha$ = 1.0, $I$ = 2000, $m$ = 2, and $N$ = 5000.]{
Approximate marginal posterior distributions of BA model parameters obtained by applying 	extit{netabc} to a simulated transmission tree with BA parameter values $alpha$ = 1.0, $I$ = 2000, $m$ = 2, and $N$ = 5000. Vertical dashed lines indicate true values. Shaded areas are 95\% highest posterior density intervals. $x$-axes indicate regions of nonzero prior density.
}
\label{fig:1.0-2000-2-5000-0}
\end{figure}
\clearpage

\begin{figure}[H]
\centering
\includegraphics{{abc-posterior/1.5_2000_2_5000_0}.pdf}
\caption[Approximate marginal posterior distributions of BA model parameters for a simulated transmission tree with $\alpha$ = 1.5, $I$ = 2000, $m$ = 2, and $N$ = 5000.]{
Approximate marginal posterior distributions of BA model parameters obtained by applying 	extit{netabc} to a simulated transmission tree with BA parameter values $alpha$ = 1.5, $I$ = 2000, $m$ = 2, and $N$ = 5000. Vertical dashed lines indicate true values. Shaded areas are 95\% highest posterior density intervals. $x$-axes indicate regions of nonzero prior density.
}
\label{fig:1.5-2000-2-5000-0}
\end{figure}
\clearpage

\begin{figure}[H]
\centering
\includegraphics{{abc-posterior/0.0_1000_3_5000_0}.pdf}
\caption[Approximate marginal posterior distributions of BA model parameters for a simulated transmission tree with $\alpha$ = 0.0, $I$ = 1000, $m$ = 3, and $N$ = 5000.]{
Approximate marginal posterior distributions of BA model parameters obtained by applying 	extit{netabc} to a simulated transmission tree with BA parameter values $alpha$ = 0.0, $I$ = 1000, $m$ = 3, and $N$ = 5000. Vertical dashed lines indicate true values. Shaded areas are 95\% highest posterior density intervals. $x$-axes indicate regions of nonzero prior density.
}
\label{fig:0.0-1000-3-5000-0}
\end{figure}
\clearpage

\begin{figure}[H]
\centering
\includegraphics{{abc-posterior/0.5_1000_3_5000_0}.pdf}
\caption[Approximate marginal posterior distributions of BA model parameters for a simulated transmission tree with $\alpha$ = 0.5, $I$ = 1000, $m$ = 3, and $N$ = 5000.]{
Approximate marginal posterior distributions of BA model parameters obtained by applying 	extit{netabc} to a simulated transmission tree with BA parameter values $alpha$ = 0.5, $I$ = 1000, $m$ = 3, and $N$ = 5000. Vertical dashed lines indicate true values. Shaded areas are 95\% highest posterior density intervals. $x$-axes indicate regions of nonzero prior density.
}
\label{fig:0.5-1000-3-5000-0}
\end{figure}
\clearpage

\begin{figure}[H]
\centering
\includegraphics{{abc-posterior/1.0_1000_3_5000_0}.pdf}
\caption[Approximate marginal posterior distributions of BA model parameters for a simulated transmission tree with $\alpha$ = 1.0, $I$ = 1000, $m$ = 3, and $N$ = 5000.]{
Approximate marginal posterior distributions of BA model parameters obtained by applying 	extit{netabc} to a simulated transmission tree with BA parameter values $alpha$ = 1.0, $I$ = 1000, $m$ = 3, and $N$ = 5000. Vertical dashed lines indicate true values. Shaded areas are 95\% highest posterior density intervals. $x$-axes indicate regions of nonzero prior density.
}
\label{fig:1.0-1000-3-5000-0}
\end{figure}
\clearpage

\begin{figure}[H]
\centering
\includegraphics{{abc-posterior/1.5_1000_3_5000_0}.pdf}
\caption[Approximate marginal posterior distributions of BA model parameters for a simulated transmission tree with $\alpha$ = 1.5, $I$ = 1000, $m$ = 3, and $N$ = 5000.]{
Approximate marginal posterior distributions of BA model parameters obtained by applying 	extit{netabc} to a simulated transmission tree with BA parameter values $alpha$ = 1.5, $I$ = 1000, $m$ = 3, and $N$ = 5000. Vertical dashed lines indicate true values. Shaded areas are 95\% highest posterior density intervals. $x$-axes indicate regions of nonzero prior density.
}
\label{fig:1.5-1000-3-5000-0}
\end{figure}
\clearpage

\begin{figure}[H]
\centering
\includegraphics{{abc-posterior/0.0_2000_3_5000_0}.pdf}
\caption[Approximate marginal posterior distributions of BA model parameters for a simulated transmission tree with $\alpha$ = 0.0, $I$ = 2000, $m$ = 3, and $N$ = 5000.]{
Approximate marginal posterior distributions of BA model parameters obtained by applying 	extit{netabc} to a simulated transmission tree with BA parameter values $alpha$ = 0.0, $I$ = 2000, $m$ = 3, and $N$ = 5000. Vertical dashed lines indicate true values. Shaded areas are 95\% highest posterior density intervals. $x$-axes indicate regions of nonzero prior density.
}
\label{fig:0.0-2000-3-5000-0}
\end{figure}
\clearpage

\begin{figure}[H]
\centering
\includegraphics{{abc-posterior/0.5_2000_3_5000_0}.pdf}
\caption[Approximate marginal posterior distributions of BA model parameters for a simulated transmission tree with $\alpha$ = 0.5, $I$ = 2000, $m$ = 3, and $N$ = 5000.]{
Approximate marginal posterior distributions of BA model parameters obtained by applying 	extit{netabc} to a simulated transmission tree with BA parameter values $alpha$ = 0.5, $I$ = 2000, $m$ = 3, and $N$ = 5000. Vertical dashed lines indicate true values. Shaded areas are 95\% highest posterior density intervals. $x$-axes indicate regions of nonzero prior density.
}
\label{fig:0.5-2000-3-5000-0}
\end{figure}
\clearpage

\begin{figure}[H]
\centering
\includegraphics{{abc-posterior/1.0_2000_3_5000_0}.pdf}
\caption[Approximate marginal posterior distributions of BA model parameters for a simulated transmission tree with $\alpha$ = 1.0, $I$ = 2000, $m$ = 3, and $N$ = 5000.]{
Approximate marginal posterior distributions of BA model parameters obtained by applying 	extit{netabc} to a simulated transmission tree with BA parameter values $alpha$ = 1.0, $I$ = 2000, $m$ = 3, and $N$ = 5000. Vertical dashed lines indicate true values. Shaded areas are 95\% highest posterior density intervals. $x$-axes indicate regions of nonzero prior density.
}
\label{fig:1.0-2000-3-5000-0}
\end{figure}
\clearpage

\begin{figure}[H]
\centering
\includegraphics{{abc-posterior/1.5_2000_3_5000_0}.pdf}
\caption[Approximate marginal posterior distributions of BA model parameters for a simulated transmission tree with $\alpha$ = 1.5, $I$ = 2000, $m$ = 3, and $N$ = 5000.]{
Approximate marginal posterior distributions of BA model parameters obtained by applying 	extit{netabc} to a simulated transmission tree with BA parameter values $alpha$ = 1.5, $I$ = 2000, $m$ = 3, and $N$ = 5000. Vertical dashed lines indicate true values. Shaded areas are 95\% highest posterior density intervals. $x$-axes indicate regions of nonzero prior density.
}
\label{fig:1.5-2000-3-5000-0}
\end{figure}
\clearpage

\begin{figure}[H]
\centering
\includegraphics{{abc-posterior/0.0_1000_4_5000_0}.pdf}
\caption[Approximate marginal posterior distributions of BA model parameters for a simulated transmission tree with $\alpha$ = 0.0, $I$ = 1000, $m$ = 4, and $N$ = 5000.]{
Approximate marginal posterior distributions of BA model parameters obtained by applying 	extit{netabc} to a simulated transmission tree with BA parameter values $alpha$ = 0.0, $I$ = 1000, $m$ = 4, and $N$ = 5000. Vertical dashed lines indicate true values. Shaded areas are 95\% highest posterior density intervals. $x$-axes indicate regions of nonzero prior density.
}
\label{fig:0.0-1000-4-5000-0}
\end{figure}
\clearpage

\begin{figure}[H]
\centering
\includegraphics{{abc-posterior/0.5_1000_4_5000_0}.pdf}
\caption[Approximate marginal posterior distributions of BA model parameters for a simulated transmission tree with $\alpha$ = 0.5, $I$ = 1000, $m$ = 4, and $N$ = 5000.]{
Approximate marginal posterior distributions of BA model parameters obtained by applying 	extit{netabc} to a simulated transmission tree with BA parameter values $alpha$ = 0.5, $I$ = 1000, $m$ = 4, and $N$ = 5000. Vertical dashed lines indicate true values. Shaded areas are 95\% highest posterior density intervals. $x$-axes indicate regions of nonzero prior density.
}
\label{fig:0.5-1000-4-5000-0}
\end{figure}
\clearpage

\begin{figure}[H]
\centering
\includegraphics{{abc-posterior/1.0_1000_4_5000_0}.pdf}
\caption[Approximate marginal posterior distributions of BA model parameters for a simulated transmission tree with $\alpha$ = 1.0, $I$ = 1000, $m$ = 4, and $N$ = 5000.]{
Approximate marginal posterior distributions of BA model parameters obtained by applying 	extit{netabc} to a simulated transmission tree with BA parameter values $alpha$ = 1.0, $I$ = 1000, $m$ = 4, and $N$ = 5000. Vertical dashed lines indicate true values. Shaded areas are 95\% highest posterior density intervals. $x$-axes indicate regions of nonzero prior density.
}
\label{fig:1.0-1000-4-5000-0}
\end{figure}
\clearpage

\begin{figure}[H]
\centering
\includegraphics{{abc-posterior/1.5_1000_4_5000_0}.pdf}
\caption[Approximate marginal posterior distributions of BA model parameters for a simulated transmission tree with $\alpha$ = 1.5, $I$ = 1000, $m$ = 4, and $N$ = 5000.]{
Approximate marginal posterior distributions of BA model parameters obtained by applying 	extit{netabc} to a simulated transmission tree with BA parameter values $alpha$ = 1.5, $I$ = 1000, $m$ = 4, and $N$ = 5000. Vertical dashed lines indicate true values. Shaded areas are 95\% highest posterior density intervals. $x$-axes indicate regions of nonzero prior density.
}
\label{fig:1.5-1000-4-5000-0}
\end{figure}
\clearpage

\begin{figure}[H]
\centering
\includegraphics{{abc-posterior/0.0_2000_4_5000_0}.pdf}
\caption[Approximate marginal posterior distributions of BA model parameters for a simulated transmission tree with $\alpha$ = 0.0, $I$ = 2000, $m$ = 4, and $N$ = 5000.]{
Approximate marginal posterior distributions of BA model parameters obtained by applying 	extit{netabc} to a simulated transmission tree with BA parameter values $alpha$ = 0.0, $I$ = 2000, $m$ = 4, and $N$ = 5000. Vertical dashed lines indicate true values. Shaded areas are 95\% highest posterior density intervals. $x$-axes indicate regions of nonzero prior density.
}
\label{fig:0.0-2000-4-5000-0}
\end{figure}
\clearpage

\begin{figure}[H]
\centering
\includegraphics{{abc-posterior/0.5_2000_4_5000_0}.pdf}
\caption[Approximate marginal posterior distributions of BA model parameters for a simulated transmission tree with $\alpha$ = 0.5, $I$ = 2000, $m$ = 4, and $N$ = 5000.]{
Approximate marginal posterior distributions of BA model parameters obtained by applying 	extit{netabc} to a simulated transmission tree with BA parameter values $alpha$ = 0.5, $I$ = 2000, $m$ = 4, and $N$ = 5000. Vertical dashed lines indicate true values. Shaded areas are 95\% highest posterior density intervals. $x$-axes indicate regions of nonzero prior density.
}
\label{fig:0.5-2000-4-5000-0}
\end{figure}
\clearpage

\begin{figure}[H]
\centering
\includegraphics{{abc-posterior/1.0_2000_4_5000_0}.pdf}
\caption[Approximate marginal posterior distributions of BA model parameters for a simulated transmission tree with $\alpha$ = 1.0, $I$ = 2000, $m$ = 4, and $N$ = 5000.]{
Approximate marginal posterior distributions of BA model parameters obtained by applying 	extit{netabc} to a simulated transmission tree with BA parameter values $alpha$ = 1.0, $I$ = 2000, $m$ = 4, and $N$ = 5000. Vertical dashed lines indicate true values. Shaded areas are 95\% highest posterior density intervals. $x$-axes indicate regions of nonzero prior density.
}
\label{fig:1.0-2000-4-5000-0}
\end{figure}
\clearpage

\begin{figure}[H]
\centering
\includegraphics{{abc-posterior/1.5_2000_4_5000_0}.pdf}
\caption[Approximate marginal posterior distributions of BA model parameters for a simulated transmission tree with $\alpha$ = 1.5, $I$ = 2000, $m$ = 4, and $N$ = 5000.]{
Approximate marginal posterior distributions of BA model parameters obtained by applying 	extit{netabc} to a simulated transmission tree with BA parameter values $alpha$ = 1.5, $I$ = 2000, $m$ = 4, and $N$ = 5000. Vertical dashed lines indicate true values. Shaded areas are 95\% highest posterior density intervals. $x$-axes indicate regions of nonzero prior density.
}
\label{fig:1.5-2000-4-5000-0}
\end{figure}
\clearpage


