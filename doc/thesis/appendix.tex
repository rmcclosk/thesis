

\chapter{Mathematical models, likelihood, and Bayesian inference}
\label[appendix]{chp:prelim}

A \defn{mathematical model} is a formal description of a hypothesized
relationship between some observed data, $x$ and outcomes $y$. A
\defn{parametric} model defines a family of possible relationships between data
and outcomes, parameterized by one or more numeric parameters $\theta$. A
\defn{statistical} model describes the relationship between data and outcomes
in terms of probabilities. Statistical models define, either explicitly or
implicitly, the probability of observing $y$ given $x$ and, if the model is
parametric, $\theta$. Note that it is entirely possible to have no data $x$,
only observed outcomes $y$. In this case, a model would describe the process by
which $y$ is generated.

To illustrate these concepts, consider the well-known linear model. For
clarity, we will restrict our attention to the case of one-dimensional data and
outcomes where $x = \set{x_1, \ldots, x_n}$ and $y = \set{y_1, \ldots, y_n}$
are vectors of real numbers. The linear model postulates that the outcomes are
linearly related to the data, modulo some noise introduced by measurement
error, environmental fluctuations, and other external factors. Formally, $y_i =
\beta x_i + \varepsilon_i$, where $\beta$ is the slope of the linear
relationship, and $\varepsilon_i$ is the error associated with measurement $i$.
We can make this model a statistical one by hypothesizing a distribution for
the error terms $\varepsilon_i$; most commonly, it is assumed that they are
normally distributed with variance $\sigma$. In mathematical terms, $y_i \sim
\beta x_i + \N(0, \sigma^2)$, where ``$\sim$'' means ``is distributed as''. We
can see from this formulation that the model is parametric, with parameters
$\theta$ = ($\beta$, $\sigma$). Moreover, we can write down the probability
density $\pi$ of observing outcome $y_i$ given the parameters,
\[
  \pi(y \mid \beta, \sigma) = 
  \prod_{i=1}^n f_{\N(0, \sigma^2)} (y_i - \beta x_i),
\]
where $f_{\N(0, \sigma^2)}$ is the probability density of the normal
distribution with mean zero and variance $\sigma^2$. Note that we are treating
the $x_i$ as fixed quantities and therefore have not conditioned the
probability density on $x$. Also, we have assumed that all the $y_i$ are
independent.

For a general model, the probability density of $y$ given the parameters
$\theta$ is also known as the \defn{likelihood}, written $\L$, of $\theta$.
That is, $\L(\theta \mid y) = f(y \mid \theta)$ for the model's \gls{pdf} $f$.
The higher the value of the likelihood, the more likely the observations $y$
are under the model. Thus, the likelihood provides a natural criterion for
fitting the model parameters: we want to pick $\theta$ such that the
probability density of our observed outcomes $y$ is as high as possible. The
parameters that optimize the likelihood are known as the \textit{\gls{ML}}
estimates, denoted $\hat{\theta}$. That is,
\[
  \hat{\theta} = \argmax_\theta\; \L(\theta \mid y).
\]
\Gls{ML} estimation is usually performed with numerical optimization. In the
simplest terms, many possible values for $\theta$ are examined, $\L(\theta \mid
y)$ is calculated for each, and the parameters that produce the highest value
are accepted. Many sophisticated numerical optimization methods exist, although
they may not be guaranteed to find the true \gls{ML} estimates if the
likelihood function is multi-modal. 

\Gls{ML} estimation makes use only of the data and outcomes to estimate the
model parameters $\theta$. However, it is frequently the case that the
investigator has some additional information or belief about what $\theta$ are
likely to be. For example, in the linear regression case, the instrument used
to measure the outcomes may have a well-known margin of error, or the sign of
the slope may be obvious from previous experiments. The Bayesian approach to
model fitting makes use of this information by codifying the investigator's
beliefs as a \defn{prior distribution} on the parameters, denoted
$\pi(\theta)$. Instead of considering only the likelihood, Bayesian inference
focuses on the product of the likelihood and the prior, $f(y \mid \theta)
\pi(\theta)$. Bayes' theorem tells us that this product is related to the
\textit{posterior distribution} on $\theta$,
\begin{align}
  f(\theta \mid y) 
    = \frac{f(y \mid \theta) \pi(\theta)}
           {\int f(y \mid \theta) \pi(\theta) \d \theta}.
  \label{eq:bayes}
\end{align}
In principle, $f(y \mid \theta) \pi(\theta)$ can be optimized numerically just
like $\L(\theta \mid y)$, which would also optimize the posterior distribution.
The resulting optimal parameters are called the \gls{MAP} estimates. However,
from a Bayesian perspective, $\theta$ is not a fixed quantity to be estimated,
but rather a random variable with an associated distribution (the posterior).
Therefore, the \gls{MAP} estimate by itself is of limited value without
associated statistics about the posterior distribution, such as the mean or
credible intervals. Unfortunately, to calculate such statistics, it is
necessary to evaluate the normalizing constant in the denominator of
\cref{eq:bayes}, which is almost always an intractable integral.

A popular method for circumventing the normalizing constant is the use of
\gls{MCMC} to obtain a sample from the posterior distribution. \Gls{MCMC} works
by defining a Markov chain {\color{red}\sout{whose states are indexed by
possible model parameters. The transition probability from state $\theta_1$ to
state $\theta_2$ is taken to be}} {\color{blue}\uline{on the space of possible
model parameters. The transition density from parameters $\theta_1$ to
$\theta_2$ is taken to be}}
\[
  \min\left(1, \frac{f(y \mid \theta_2) \pi(\theta_2) q(\theta_2, \theta_1)}
                    {f(y \mid \theta_1) \pi(\theta_2) q(\theta_1, \theta_2)} \right),
\]
where $q(\theta, \theta')$ is a symmetric \defn{proposal distribution} used in
the algorithm to generate the chain. The stationary distribution of this Markov
chain is equal to the posterior distribution on $\theta$. Therefore, if a long
enough random walk is performed on the chain, the distribution of states
visited will be a Monte Carlo approximation of $f(\theta \mid y)$, from
which we can calculate statistics of interest. Actually performing this random
walk is straightforward and can be accomplished via the Metropolis-Hastings
algorithm~\autocite{metropolis1953equation,hastings1970monte} (\cref{alg:mh}).

\begin{algorithm}
  \caption{Metropolis-Hastings algorithm for Markov chain Monte Carlo.}
  \begin{algorithmic}
    \State Draw $\theta$ according to the prior $\pi(\theta)$
    \Loop
      \State Propose $\theta'$ according to $q(\theta, \theta')$
      \State Accept $\theta \gets \theta'$ with probability
      $\min \left( 1, 
       \dfrac{f(y \mid \theta') \pi(\theta') q(\theta', \theta)}
             {f(y \mid \theta\phantom{'}) \pi(\theta\phantom{'}) q(\theta, \theta')}
       \right)$
    \EndLoop
  \end{algorithmic}
  \label{alg:mh}
\end{algorithm}


\chapter{Additional plots}

\begin{figure}[ht]
  \centering
  \includegraphics{leventhal2012fig1.pdf}
  \caption[
    Reproduction of Figure 1A from Leventhal \textit{et al.} (2012) used to
    check the accuracy of our implementation of Gillespie simulation.
  ]{
    Reproduction of Figure 1A from Leventhal \textit{et al.} (2012) used to
    check the accuracy of our implementation of Gillespie simulation.
    Transmission trees were simulated over three types of network, with
    pathogen transmissibility varying from 0 to 1. Sackin's index was
    calculated for each simulated transmission tree. Lines indicate median
    Sackin's index values, and shaded areas are interquartile ranges.
  }
  \label{fig:leventhal}
\end{figure}

\begin{figure}[ht]
  \centering
  \includegraphics{smc-test.pdf}
  \caption[
      Approximation of mixture of Gaussians used by Del Moral \textit{et al.}
      (2012) and Sisson \textit{et al.} (2009) to test adaptive
      \gls{ABC}-\gls{SMC}.
    ]{
      Approximation of mixture of Gaussians used by Del Moral \textit{et al.}
      (2012) and Sisson \textit{et al.} (2009) to test SMC. Solid black line
      indicates true distribution. Grey shaded area shows \gls{ABC}
      approximation obtained with our implementation of adaptive
      \gls{ABC}-\gls{SMC}, using 10000 particles with one simulated data point
      per particle.
    }
  \label{fig:smctest}
\end{figure}

\begin{figure}[ht]
  \centering
  \includegraphics{smc-test-bimodal.pdf}
  \caption[
    Approximation of mixture of two Gaussians used to test convergence of
    adaptive \gls{ABC}-\gls{SMC} algorithm to a bimodal distribution.
  ]{
    Approximation of mixture of two Gaussians used to test convergence of SMC
    algorithm to a bimodal distribution. Solid black line indicates true
    distribution. Grey shaded area shows \gls{ABC}-\gls{SMC} approximation
    obtained with our implementation, using 10000 particles with one simulated
    data point per particle.
  }
  \label{fig:smctest2}
\end{figure}

\begin{figure}[ht]
  \centering
  \includegraphics[width=\textwidth]{kernel-I-tree.pdf}
  \caption[Simulated transmission trees under three different values of BA parameter $I$]{
    Simulated transmission trees under three different values of BA parameter
    $I$. Epidemics were simulated on \gls{BA} networks with parameters $\alpha$
    = 1.0, $m = 2$, and $N = 5000$. Epidemics were simulated until $I = 500$,
    1000, or 2000 nodes were infected. Transmission trees were created by
    sampling 500 infected nodes. For higher \gls{I} values, the network was
    closer to saturation at the time of sampling, resulting in longer terminal
    branches as the waiting time until the next transmission increased.
  }
  \label{fig:Itrees}
\end{figure}

\begin{figure}[ht]
  \centering
  \includegraphics[width=\textwidth]{kernel-m-tree.pdf}
  \caption[Simulated transmission trees under three different values of BA parameter $m$]{
    Simulated transmission trees under three different values of BA parameter
    $m$. Epidemics were simulated on \gls{BA} networks with parameters $\alpha$
    = 1.0, $N = 5000$, and $m$ = 2, 3, or 4. Epidemics were simulated until $I
    = 1000$ nodes were infected. Transmission trees were created by sampling
    500 infected nodes.
  }
  \label{fig:mtrees}
\end{figure}

\begin{figure}[ht]
  \centering
  \includegraphics[width=\textwidth]{kernel-N-tree.pdf}
  \caption[Simulated transmission trees under three different values of BA parameter $N$]{
    Simulated transmission trees under three different values of BA parameter
    $N$. Epidemics were simulated on \gls{BA} networks with parameters $\alpha$
    = 1.0, $m = 2$, and $N$ = 3000, 5000, or 8000. Epidemics were simulated
    until $I = 1000$ nodes were infected. Transmission trees were created by
    sampling 500 infected nodes. For lower $N$ values, the network was closer
    to saturation at the time of sampling, resulting in longer waiting times
    until the next transmission and longer terminal branch lengths.
  }
  \label{fig:Ntrees}
\end{figure}

\begin{figure}[ht]
  \centering
  \includegraphics{kernel-alpha-crossv.pdf}
  \caption[Cross validation accuracy of classifiers for \gls{BA} model parameter
    \gls{alpha} for eight epidemic scenarios.]
  {
    Cross validation accuracy of classifiers for \gls{BA} model parameter
    \gls{alpha} for eight epidemic scenarios. Solid lines and points are
    $R^2$ of tree kernel \gls{kSVR} under various kernel meta-parameters.
    Dashed and dotted lines are $R^2$ of linear regression against Sackin's
    index, and \gls{SVR} using \gls{nltt}.
  }
  \label{fig:alphacrossv}
\end{figure}

\begin{figure}[ht]
  \centering
  \includegraphics{kernel-I-crossv.pdf}
  \caption[Cross validation accuracy of classifiers for \gls{BA} model parameter
    \gls{I} for eight epidemic scenarios.]
  {
    Cross validation accuracy of classifiers for \gls{BA} model parameter
    \gls{I} for eight epidemic scenarios. Solid lines and points are $R^2$ of
    tree kernel \gls{kSVR} under various kernel meta-parameters. Dashed and
    dotted lines are $R^2$ of linear regression against Sackin's index, and
    \gls{SVR} using \gls{nltt}.
  }
  \label{fig:Icrossv}
\end{figure}

\begin{figure}[ht]
  \centering
  \includegraphics{kernel-m-crossv.pdf}
  \caption[Cross validation accuracy of classifiers for \gls{BA} model parameter
    \gls{m} for eight epidemic scenarios.]
  {
    Cross validation accuracy of classifiers for \gls{BA} model parameter
    \gls{m} for eight epidemic scenarios. Solid lines and points are $R^2$ of
    tree kernel \gls{kSVR} under various kernel meta-parameters. Dashed and
    dotted lines are $R^2$ of linear regression against Sackin's index, and
    \gls{SVR} using \gls{nltt}.
  }
  \label{fig:mcrossv}
\end{figure}

\begin{figure}[ht]
  \centering
  \includegraphics{kernel-N-crossv.pdf}
  \caption[Cross validation accuracy of classifiers for \gls{BA} model parameter
    \gls{N} for eight epidemic scenarios.]
  {
    Cross validation accuracy of classifiers for \gls{BA} model parameter
    \gls{N} for eight epidemic scenarios. Solid lines and points are $R^2$ of
    tree kernel \gls{kSVR} under various kernel meta-parameters. Dashed and
    dotted lines are $R^2$ of linear regression against Sackin's index, and
    \gls{SVR} using \gls{nltt}.
  }
  \label{fig:Ncrossv}
\end{figure}

\clearpage

\begin{figure}[ht]
  \centering
  \includegraphics{kernel-alpha-kpca.pdf}
  \caption{
    Kernel principal components projection of trees simulated under three
    different values of \gls{BA} parameter \gls{alpha}, for eight epidemic
    scenarios.
  }
  \label{fig:alphakpca}
\end{figure}

\begin{figure}[ht]
  \centering
  \includegraphics{kernel-I-kpca.pdf}
  \caption{
    Kernel principal components projection of trees simulated under three
    different values of \gls{BA} parameter \gls{I}, for eight epidemic
    scenarios.
  }
  \label{fig:Ikpca}
\end{figure}

\begin{figure}[ht]
  \centering
  \includegraphics{kernel-m-kpca.pdf}
  \caption{
    Kernel principal components projection of trees simulated under three
    different values of \gls{BA} parameter \gls{m}, for eight epidemic
    scenarios.
  }
  \label{fig:mkpca}
\end{figure}

\begin{figure}[ht]
  \centering
  \includegraphics{kernel-N-kpca.pdf}
  \caption{
    Kernel principal components projection of trees simulated under three
    different values of \gls{BA} parameter \gls{N}, for eight epidemic
    scenarios.
  }
  \label{fig:Nkpca}
\end{figure}

\clearpage

\begin{figure}[ht]
  \centering
  \includegraphics{gridsearch-alpha-kernel.pdf}
  \caption{                                              
    Grid search kernel scores for testing trees simulated under various             
    \gls{alpha} values. The other \gls{BA} parameters were fixed at \gls{I} =
    1000, \gls{N} = 5000, and \gls{m} = 2. 
  }        
  \label{fig:gridalpha}
\end{figure}

\begin{figure}[ht]
  \centering
  \includegraphics{gridsearch-I-kernel.pdf}
  \caption{                                              
    Grid search kernel scores for testing trees simulated under various             
    \gls{I} values. The other \gls{BA} parameters were fixed at \gls{alpha} =
    1.0, \gls{N} = 5000, and \gls{m} = 2. 
  }        
  \label{fig:gridI}
\end{figure}

\begin{figure}[ht]
  \centering
  \includegraphics{gridsearch-m-kernel.pdf}
  \caption{                                              
    Grid search kernel scores for testing trees simulated under various             
    \gls{m} values. The other \gls{BA} parameters were fixed at \gls{alpha} =
    1.0, \gls{I} = 1000, and \gls{N} = 5000.
  }        
  \label{fig:gridm}
\end{figure}

\begin{figure}[ht]
  \centering
  \includegraphics{gridsearch-N-kernel.pdf}
  \caption{                                              
    Grid search kernel scores for testing trees simulated under various             
    \gls{N} values. The other \gls{BA} parameters were fixed at \gls{alpha} =
    1.0, \gls{I} = 1000, and \gls{m} = 2.
  }        
  \label{fig:gridN}
\end{figure}

\begin{figure}[ht]
  \centering
  \includegraphics{gridsearch-alpha-point-estimate.pdf}
  \caption[
      Point estimates of preferential attachment power \gls{alpha} of
      \acrlong{BA} network model, obtained on simulated trees with
      kernel-score-based grid search.]
  {
      Point estimates of preferential attachment power \gls{alpha} of
      \acrlong{BA} network model, obtained on simulated trees with
      kernel-score-based grid search. Test trees were simulated according to
      several values of \gls{alpha} ($x$-axis) with other model parameters
      fixed at \gls{m} = 2, \gls{N} = 5000, and \gls{I} = 1000. The test trees
      were compared to trees simulated along a narrowly spaced grid of
      \gls{alpha} values using the tree kernel, with the same values of the
      other parameters. The grid value with the highest median kernel score was
      taken as a point estimate for \gls{alpha} ($y$-axis).
  }
  \label{fig:gridptalpha}
\end{figure}

\begin{figure}[ht]
  \centering
  \includegraphics{gridsearch-I-point-estimate.pdf}
  \caption[ Point estimates of prevalence at time of sampling \gls{I} of
      \acrlong{BA} network model, obtained on simulated trees with
      kernel-score-based grid search. ]
  {
      Point estimates of prevalence at time of sampling \gls{I} of \acrlong{BA}
      network model, obtained on simulated trees with kernel-score-based grid
      search. Test trees were simulated according to several values of \gls{I}
      ($x$-axis) with other model parameters fixed at \gls{alpha} = 1, \gls{m}
      = 2, and \gls{N} = 5000. The test trees were compared to trees simulated
      along a narrowly spaced grid of \gls{I} values using the tree kernel,
      with the same values of the other parameters. The grid value with the
      highest median kernel score was taken as a point estimate for \gls{I}
      ($y$-axis).
  }
  \label{fig:gridptI}
\end{figure}

\begin{figure}[ht]
  \centering
  \includegraphics{gridsearch-m-point-estimate.pdf}
  \caption[
      Point estimates of number of edges per vertex \gls{m} of \acrlong{BA}
      network model, obtained on simulated trees with kernel-score-based grid
      search.
  ]
  {
      Point estimates of number of edges per vertex \gls{m} of \acrlong{BA}
      network model, obtained on simulated trees with kernel-score-based grid
      search. Test trees were simulated according to several values of \gls{m}
      ($x$-axis) with other model parameters fixed at \gls{alpha} = 1, \gls{I}
      = 1000, and \gls{N} = 5000. The test trees were compared to trees
      simulated along a narrowly spaced grid of \gls{m} values using the tree
      kernel, with the same values of the other parameters. The grid value with
      the highest median kernel score was taken as a point estimate for \gls{m}
      ($y$-axis).
  }
  \label{fig:gridptm}
\end{figure}

\begin{figure}[ht]
  \centering
  \includegraphics{gridsearch-N-point-estimate.pdf}
  \caption[
      Point estimates of number of edges per vertex \gls{N} of \acrlong{BA}
      network model, obtained on simulated trees with kernel-score-based grid
      search.
  ]{
      Point estimates of number of edges per vertex \gls{N} of \acrlong{BA}
      network model, obtained on simulated trees with kernel-score-based grid
      search. Test trees were simulated according to several values of \gls{N}
      ($x$-axis) with other model parameters fixed at \gls{alpha} = 1, \gls{m}
      = 2, and \gls{I} = 1000. The test trees were compared to trees simulated
      along a narrowly spaced grid of \gls{N} values using the tree kernel,
      with the same values of the other parameters. The grid value with the
      highest median kernel score was taken as a point estimate for \gls{m}
      ($y$-axis).
  }
  \label{fig:gridptN}
\end{figure}

\clearpage

%\begin{figure}[ht]
%  \centering
%  \includegraphics{abc-point-estimate-m1.pdf}
%  \caption[
%    \Acrlong{MAP} point estimates for \gls{BA} model parameters obtained by
%    running \software{netabc} on simulated data, for simulations with \gls{m} = 1.
%  ]{
%    \Acrlong{MAP} point estimates for \gls{BA} model parameters obtained by
%    running \software{netabc} on simulated data, for simulations with \gls{m} = 3.
%    Dashed lines indicate true values. (A) Estimates of \gls{alpha} and \gls{I}
%    which were varied in these simulations against known values. (B) Estimates
%    of \gls{m} and \gls{N} which were held fixed in these simulations at the
%    values \gls{m} = 1 and \gls{N} = 5000.
%  }        
%  \label{fig:abcptm1}
%\end{figure}

\begin{figure}[ht]
  \centering
  \includegraphics{abc-point-estimate-m3.pdf}
  \caption[
    \Acrlong{MAP} point estimates for \gls{BA} model parameters obtained by
    running \software{netabc} on simulated data, for simulations with \gls{m} = 3.
  ]{
    \Acrlong{MAP} point estimates for \gls{BA} model parameters obtained by
    running \software{netabc} on simulated data, for simulations with \gls{m} = 3.
    Dashed lines indicate true values. (A) Estimates of \gls{alpha} and \gls{I}
    which were varied in these simulations against known values. (B) Estimates
    of \gls{m} and \gls{N} which were held fixed in these simulations at the
    values \gls{m} = 3 and \gls{N} = 5000.
  }        
  \label{fig:abcptm3}
\end{figure}

\begin{figure}[ht]
  \centering
  \includegraphics{abc-point-estimate-m4.pdf}
  \caption[
    \Acrlong{MAP} point estimates for \gls{BA} model parameters obtained by
    running \software{netabc} on simulated data, for simulations with \gls{m} = 3.
  ]{
    \Acrlong{MAP} point estimates for \gls{BA} model parameters obtained by
    running \software{netabc} on simulated data, for simulations with \gls{m} = 4.
    Dashed lines indicate true values. (A) Estimates of \gls{alpha} and \gls{I}
    which were varied in these simulations against known values. (B) Estimates
    of \gls{m} and \gls{N} which were held fixed in these simulations at the
    values \gls{m} = 4 and \gls{N} = 5000.
  }        
  \label{fig:abcptm4}
\end{figure}

\begin{figure}[ht]
  \centering
  \includegraphics{alpha-gamma.pdf}
  \caption{
      Relationship between preferential attachment power parameter $\alpha$
      and power law exponent $\gamma$ for networks simulated under the BA
      network model with $N$ = 5000 and $m$ = 2.
  }
  \label{fig:gamma}
\end{figure}

\begin{figure}[ht]
    \centering
    \includegraphics{mixed-posterior}
    \caption[Approximate marginal posterior distributions of BA model
        parameters obtained using kernel-assisted ABC for a network with heterogeneous
        node behaviour.]
    {
        Approximate marginal posterior distributions of Barab\'asi-Albert
        model parameters obtained using kernel-assisted ABC for a network with
        heterogeneous node behaviour. Half of the nodes were attached with
        $\alpha = 0.5$, and the other half with $\alpha = 1.5$ (vertical
        dashed lines, top left). Other parameter values were $m = 2$, $I =
        1000$, and $N = 5000$ (vertical dashed lines, other than top left).
        Shaded areas indicate 95\% highest posterior density intervals.
    }
    \label{fig:mixed}
\end{figure}

\begin{figure}[ht]
    \centering
    \includegraphics{peerdriven-posterior}
    \caption[Approximate marginal posterior distributions of Barab\'asi-Albert
        model parameters obtained using kernel-assisted ABC for a network with
        peer-driven sampling.]
    {
        Approximate marginal posterior distributions of Barab\'asi-Albert
        model parameters obtained using kernel-assisted ABC for a network with
        peer-driven sampling. An epidemic was simulated in the usual fashion,
        but rather than being sampled at random, infected nodes were sampled
        with a probability two times higher if they had any sampled neighbours
        in the contact network. Vertical dashed lines indicate true parameter
        values, and shaded areas indicate 95\% highest posterior density
        intervals.
    }
    \label{fig:peerdriven}
\end{figure}

\begin{figure}[ht]
  \includegraphics{realdata-hpd-bc-m2}
  \vspace{8pt}
  \caption[
      Maximum \textit{a posteriori} point estimates and 95\% HPD intervals for
      parameters of the BA network model, fitted to five published HIV datasets
      with \software{netabc} using the prior $m \sim DiscreteUniform(2, 5)$.
  ]{
      Maximum \textit{a posteriori} point estimates and 95\% HPD intervals for
      parameters of the BA network model, fitted to five published HIV datasets
      with \software{netabc} using the prior $m \sim$ DiscreteUniform(2, 5).
      $x$-axes indicate regions of nonzero prior density. In particular, the
      prior on $m$ was DiscreteUniform(2, 5).
  }
  \label{fig:abchpdm2}
\end{figure}

\clearpage

\begin{figure}[ht]
  %\includegraphics{bctree-posterior}
  \caption[
      Approximate marginal posterior distributions of BA model parameters for
      BC data.
  ]
  {
      Approximate marginal posterior distributions of BA model parameters for
      BC data. Vertical lines indicate maximum \textit{a posteriori} estimates,
      and shaded areas are 95\% highest posterior density intervals. $x$-axis
      indicates regions of nonzero prior density.
  }
  \label{fig:bctree}
\end{figure}

\begin{figure}[ht]
  \includegraphics{cuevas2009-posterior}
  \caption[
      Approximate marginal posterior distributions of BA model parameters for
      \textcite{cuevas2009hiv} data. 
  ]{
      Approximate marginal posterior distributions of BA model parameters for
      \textcite{cuevas2009hiv} data. Vertical lines indicate maximum \textit{a
      posteriori} estimates, and shaded areas are 95\% highest posterior
      density intervals. $x$-axis indicates regions of nonzero prior density.
  }
  \label{fig:cuevas}
\end{figure}

\begin{figure}[ht]
  \includegraphics{li2015-posterior}
  \caption[
      Approximate marginal posterior distributions of BA model parameters for
      \textcite{li2015hiv} data. 
  ]{
      Approximate marginal posterior distributions of BA model parameters for
      \textcite{li2015hiv} data. Vertical lines indicate maximum \textit{a
      posteriori} estimates, and shaded areas are 95\% highest posterior
      density intervals. $x$-axis indicates regions of nonzero prior density.
  }
  \label{fig:li}
\end{figure}

\begin{figure}[ht]
  \includegraphics{niculescu2015-posterior}
  \caption[
      Approximate marginal posterior distributions of BA model parameters for
      \textcite{niculescu2015recent} data.
  ]{
      Approximate marginal posterior distributions of BA model parameters for
      \textcite{niculescu2015recent} data. Vertical lines indicate maximum
      \textit{a posteriori} estimates, and shaded areas are 95\% highest
      posterior density intervals. $x$-axis indicates regions of nonzero prior
      density.
  }
  \label{fig:niculescu}
\end{figure}

\begin{figure}[ht]
  \includegraphics{novitsky2014-posterior}
  \caption[
      Approximate marginal posterior distributions of BA model parameters for
      \textcite{novitsky2013phylogenetic, novitsky2014impact} data.
  ]{
      Approximate marginal posterior distributions of BA model parameters for
      \textcite{novitsky2013phylogenetic, novitsky2014impact} data. Vertical
      lines indicate maximum \textit{a posteriori} estimates, and shaded areas
      are 95\% highest posterior density intervals. $x$-axis indicates regions
      of nonzero prior density.
  }
  \label{fig:novitsky}
\end{figure}

\begin{figure}[ht]
  \includegraphics{wang2015-posterior}
  \caption[
      Approximate marginal posterior distributions of BA model parameters for
      \textcite{wang2015targeting} data.
  ]{
      Approximate marginal posterior distributions of BA model parameters for
      \textcite{wang2015targeting} data. Vertical lines indicate maximum
      \textit{a posteriori} estimates, and shaded areas are 95\% highest
      posterior density intervals. $x$-axis indicates regions of nonzero prior
      density.
  }
  \label{fig:wang}
\end{figure}

\clearpage

\begin{figure}[ht]
\centering
\includegraphics{{abc-posterior/0.0_1000_2_5000_0}.pdf}
\caption[Approximate marginal posterior distributions of BA model parameters for a simulated transmission tree with $\alpha$ = 0.0, $I$ = 1000, $m$ = 2, and $N$ = 5000.]{
Approximate marginal posterior distributions of BA model parameters obtained by applying \software{netabc} to a simulated transmission tree with BA parameter values $\alpha$ = 0.0, $I$ = 1000, $m$ = 2, and $N$ = 5000. Vertical dashed lines indicate true values. Shaded areas are 95\% highest posterior density intervals. $x$-axes indicate regions of nonzero prior density.
}
\label{fig:0.0-1000-2-5000-0}
\end{figure}
\clearpage

\begin{figure}[ht]
\centering
\includegraphics{{abc-posterior/0.5_1000_2_5000_0}.pdf}
\caption[Approximate marginal posterior distributions of BA model parameters for a simulated transmission tree with $\alpha$ = 0.5, $I$ = 1000, $m$ = 2, and $N$ = 5000.]{
Approximate marginal posterior distributions of BA model parameters obtained by applying \software{netabc} to a simulated transmission tree with BA parameter values $\alpha$ = 0.5, $I$ = 1000, $m$ = 2, and $N$ = 5000. Vertical dashed lines indicate true values. Shaded areas are 95\% highest posterior density intervals. $x$-axes indicate regions of nonzero prior density.
}
\label{fig:0.5-1000-2-5000-0}
\end{figure}
\clearpage

\begin{figure}[ht]
\centering
\includegraphics{{abc-posterior/1.0_1000_2_5000_0}.pdf}
\caption[Approximate marginal posterior distributions of BA model parameters for a simulated transmission tree with $\alpha$ = 1.0, $I$ = 1000, $m$ = 2, and $N$ = 5000.]{
Approximate marginal posterior distributions of BA model parameters obtained by applying \software{netabc} to a simulated transmission tree with BA parameter values $\alpha$ = 1.0, $I$ = 1000, $m$ = 2, and $N$ = 5000. Vertical dashed lines indicate true values. Shaded areas are 95\% highest posterior density intervals. $x$-axes indicate regions of nonzero prior density.
}
\label{fig:1.0-1000-2-5000-0}
\end{figure}
\clearpage

\begin{figure}[ht]
\centering
\includegraphics{{abc-posterior/1.5_1000_2_5000_0}.pdf}
\caption[Approximate marginal posterior distributions of BA model parameters for a simulated transmission tree with $\alpha$ = 1.5, $I$ = 1000, $m$ = 2, and $N$ = 5000.]{
Approximate marginal posterior distributions of BA model parameters obtained by applying \software{netabc} to a simulated transmission tree with BA parameter values $\alpha$ = 1.5, $I$ = 1000, $m$ = 2, and $N$ = 5000. Vertical dashed lines indicate true values. Shaded areas are 95\% highest posterior density intervals. $x$-axes indicate regions of nonzero prior density.
}
\label{fig:1.5-1000-2-5000-0}
\end{figure}
\clearpage

\begin{figure}[ht]
\centering
\includegraphics{{abc-posterior/0.0_2000_2_5000_0}.pdf}
\caption[Approximate marginal posterior distributions of BA model parameters for a simulated transmission tree with $\alpha$ = 0.0, $I$ = 2000, $m$ = 2, and $N$ = 5000.]{
Approximate marginal posterior distributions of BA model parameters obtained by applying \software{netabc} to a simulated transmission tree with BA parameter values $\alpha$ = 0.0, $I$ = 2000, $m$ = 2, and $N$ = 5000. Vertical dashed lines indicate true values. Shaded areas are 95\% highest posterior density intervals. $x$-axes indicate regions of nonzero prior density.
}
\label{fig:0.0-2000-2-5000-0}
\end{figure}
\clearpage

\begin{figure}[ht]
\centering
\includegraphics{{abc-posterior/0.5_2000_2_5000_0}.pdf}
\caption[Approximate marginal posterior distributions of BA model parameters for a simulated transmission tree with $\alpha$ = 0.5, $I$ = 2000, $m$ = 2, and $N$ = 5000.]{
Approximate marginal posterior distributions of BA model parameters obtained by applying \software{netabc} to a simulated transmission tree with BA parameter values $\alpha$ = 0.5, $I$ = 2000, $m$ = 2, and $N$ = 5000. Vertical dashed lines indicate true values. Shaded areas are 95\% highest posterior density intervals. $x$-axes indicate regions of nonzero prior density.
}
\label{fig:0.5-2000-2-5000-0}
\end{figure}
\clearpage

\begin{figure}[ht]
\centering
\includegraphics{{abc-posterior/1.0_2000_2_5000_0}.pdf}
\caption[Approximate marginal posterior distributions of BA model parameters for a simulated transmission tree with $\alpha$ = 1.0, $I$ = 2000, $m$ = 2, and $N$ = 5000.]{
Approximate marginal posterior distributions of BA model parameters obtained by applying \software{netabc} to a simulated transmission tree with BA parameter values $\alpha$ = 1.0, $I$ = 2000, $m$ = 2, and $N$ = 5000. Vertical dashed lines indicate true values. Shaded areas are 95\% highest posterior density intervals. $x$-axes indicate regions of nonzero prior density.
}
\label{fig:1.0-2000-2-5000-0}
\end{figure}
\clearpage

\begin{figure}[ht]
\centering
\includegraphics{{abc-posterior/1.5_2000_2_5000_0}.pdf}
\caption[Approximate marginal posterior distributions of BA model parameters for a simulated transmission tree with $\alpha$ = 1.5, $I$ = 2000, $m$ = 2, and $N$ = 5000.]{
Approximate marginal posterior distributions of BA model parameters obtained by applying \software{netabc} to a simulated transmission tree with BA parameter values $\alpha$ = 1.5, $I$ = 2000, $m$ = 2, and $N$ = 5000. Vertical dashed lines indicate true values. Shaded areas are 95\% highest posterior density intervals. $x$-axes indicate regions of nonzero prior density.
}
\label{fig:1.5-2000-2-5000-0}
\end{figure}
\clearpage

\begin{figure}[ht]
\centering
\includegraphics{{abc-posterior/0.0_1000_3_5000_0}.pdf}
\caption[Approximate marginal posterior distributions of BA model parameters for a simulated transmission tree with $\alpha$ = 0.0, $I$ = 1000, $m$ = 3, and $N$ = 5000.]{
Approximate marginal posterior distributions of BA model parameters obtained by applying \software{netabc} to a simulated transmission tree with BA parameter values $\alpha$ = 0.0, $I$ = 1000, $m$ = 3, and $N$ = 5000. Vertical dashed lines indicate true values. Shaded areas are 95\% highest posterior density intervals. $x$-axes indicate regions of nonzero prior density.
}
\label{fig:0.0-1000-3-5000-0}
\end{figure}
\clearpage

\begin{figure}[ht]
\centering
\includegraphics{{abc-posterior/0.5_1000_3_5000_0}.pdf}
\caption[Approximate marginal posterior distributions of BA model parameters for a simulated transmission tree with $\alpha$ = 0.5, $I$ = 1000, $m$ = 3, and $N$ = 5000.]{
Approximate marginal posterior distributions of BA model parameters obtained by applying \software{netabc} to a simulated transmission tree with BA parameter values $\alpha$ = 0.5, $I$ = 1000, $m$ = 3, and $N$ = 5000. Vertical dashed lines indicate true values. Shaded areas are 95\% highest posterior density intervals. $x$-axes indicate regions of nonzero prior density.
}
\label{fig:0.5-1000-3-5000-0}
\end{figure}
\clearpage

\begin{figure}[ht]
\centering
\includegraphics{{abc-posterior/1.0_1000_3_5000_0}.pdf}
\caption[Approximate marginal posterior distributions of BA model parameters for a simulated transmission tree with $\alpha$ = 1.0, $I$ = 1000, $m$ = 3, and $N$ = 5000.]{
Approximate marginal posterior distributions of BA model parameters obtained by applying \software{netabc} to a simulated transmission tree with BA parameter values $\alpha$ = 1.0, $I$ = 1000, $m$ = 3, and $N$ = 5000. Vertical dashed lines indicate true values. Shaded areas are 95\% highest posterior density intervals. $x$-axes indicate regions of nonzero prior density.
}
\label{fig:1.0-1000-3-5000-0}
\end{figure}
\clearpage

\begin{figure}[ht]
\centering
\includegraphics{{abc-posterior/1.5_1000_3_5000_0}.pdf}
\caption[Approximate marginal posterior distributions of BA model parameters for a simulated transmission tree with $\alpha$ = 1.5, $I$ = 1000, $m$ = 3, and $N$ = 5000.]{
Approximate marginal posterior distributions of BA model parameters obtained by applying \software{netabc} to a simulated transmission tree with BA parameter values $\alpha$ = 1.5, $I$ = 1000, $m$ = 3, and $N$ = 5000. Vertical dashed lines indicate true values. Shaded areas are 95\% highest posterior density intervals. $x$-axes indicate regions of nonzero prior density.
}
\label{fig:1.5-1000-3-5000-0}
\end{figure}
\clearpage

\begin{figure}[ht]
\centering
\includegraphics{{abc-posterior/0.0_2000_3_5000_0}.pdf}
\caption[Approximate marginal posterior distributions of BA model parameters for a simulated transmission tree with $\alpha$ = 0.0, $I$ = 2000, $m$ = 3, and $N$ = 5000.]{
Approximate marginal posterior distributions of BA model parameters obtained by applying \software{netabc} to a simulated transmission tree with BA parameter values $\alpha$ = 0.0, $I$ = 2000, $m$ = 3, and $N$ = 5000. Vertical dashed lines indicate true values. Shaded areas are 95\% highest posterior density intervals. $x$-axes indicate regions of nonzero prior density.
}
\label{fig:0.0-2000-3-5000-0}
\end{figure}
\clearpage

\begin{figure}[ht]
\centering
\includegraphics{{abc-posterior/0.5_2000_3_5000_0}.pdf}
\caption[Approximate marginal posterior distributions of BA model parameters for a simulated transmission tree with $\alpha$ = 0.5, $I$ = 2000, $m$ = 3, and $N$ = 5000.]{
Approximate marginal posterior distributions of BA model parameters obtained by applying \software{netabc} to a simulated transmission tree with BA parameter values $\alpha$ = 0.5, $I$ = 2000, $m$ = 3, and $N$ = 5000. Vertical dashed lines indicate true values. Shaded areas are 95\% highest posterior density intervals. $x$-axes indicate regions of nonzero prior density.
}
\label{fig:0.5-2000-3-5000-0}
\end{figure}
\clearpage

\begin{figure}[ht]
\centering
\includegraphics{{abc-posterior/1.0_2000_3_5000_0}.pdf}
\caption[Approximate marginal posterior distributions of BA model parameters for a simulated transmission tree with $\alpha$ = 1.0, $I$ = 2000, $m$ = 3, and $N$ = 5000.]{
Approximate marginal posterior distributions of BA model parameters obtained by applying \software{netabc} to a simulated transmission tree with BA parameter values $\alpha$ = 1.0, $I$ = 2000, $m$ = 3, and $N$ = 5000. Vertical dashed lines indicate true values. Shaded areas are 95\% highest posterior density intervals. $x$-axes indicate regions of nonzero prior density.
}
\label{fig:1.0-2000-3-5000-0}
\end{figure}
\clearpage

\begin{figure}[ht]
\centering
\includegraphics{{abc-posterior/1.5_2000_3_5000_0}.pdf}
\caption[Approximate marginal posterior distributions of BA model parameters for a simulated transmission tree with $\alpha$ = 1.5, $I$ = 2000, $m$ = 3, and $N$ = 5000.]{
Approximate marginal posterior distributions of BA model parameters obtained by applying \software{netabc} to a simulated transmission tree with BA parameter values $\alpha$ = 1.5, $I$ = 2000, $m$ = 3, and $N$ = 5000. Vertical dashed lines indicate true values. Shaded areas are 95\% highest posterior density intervals. $x$-axes indicate regions of nonzero prior density.
}
\label{fig:1.5-2000-3-5000-0}
\end{figure}
\clearpage

\begin{figure}[ht]
\centering
\includegraphics{{abc-posterior/0.0_1000_4_5000_0}.pdf}
\caption[Approximate marginal posterior distributions of BA model parameters for a simulated transmission tree with $\alpha$ = 0.0, $I$ = 1000, $m$ = 4, and $N$ = 5000.]{
Approximate marginal posterior distributions of BA model parameters obtained by applying \software{netabc} to a simulated transmission tree with BA parameter values $\alpha$ = 0.0, $I$ = 1000, $m$ = 4, and $N$ = 5000. Vertical dashed lines indicate true values. Shaded areas are 95\% highest posterior density intervals. $x$-axes indicate regions of nonzero prior density.
}
\label{fig:0.0-1000-4-5000-0}
\end{figure}
\clearpage

\begin{figure}[ht]
\centering
\includegraphics{{abc-posterior/0.5_1000_4_5000_0}.pdf}
\caption[Approximate marginal posterior distributions of BA model parameters for a simulated transmission tree with $\alpha$ = 0.5, $I$ = 1000, $m$ = 4, and $N$ = 5000.]{
Approximate marginal posterior distributions of BA model parameters obtained by applying \software{netabc} to a simulated transmission tree with BA parameter values $\alpha$ = 0.5, $I$ = 1000, $m$ = 4, and $N$ = 5000. Vertical dashed lines indicate true values. Shaded areas are 95\% highest posterior density intervals. $x$-axes indicate regions of nonzero prior density.
}
\label{fig:0.5-1000-4-5000-0}
\end{figure}
\clearpage

\begin{figure}[ht]
\centering
\includegraphics{{abc-posterior/1.0_1000_4_5000_0}.pdf}
\caption[Approximate marginal posterior distributions of BA model parameters for a simulated transmission tree with $\alpha$ = 1.0, $I$ = 1000, $m$ = 4, and $N$ = 5000.]{
Approximate marginal posterior distributions of BA model parameters obtained by applying \software{netabc} to a simulated transmission tree with BA parameter values $\alpha$ = 1.0, $I$ = 1000, $m$ = 4, and $N$ = 5000. Vertical dashed lines indicate true values. Shaded areas are 95\% highest posterior density intervals. $x$-axes indicate regions of nonzero prior density.
}
\label{fig:1.0-1000-4-5000-0}
\end{figure}
\clearpage

\begin{figure}[ht]
\centering
\includegraphics{{abc-posterior/1.5_1000_4_5000_0}.pdf}
\caption[Approximate marginal posterior distributions of BA model parameters for a simulated transmission tree with $\alpha$ = 1.5, $I$ = 1000, $m$ = 4, and $N$ = 5000.]{
Approximate marginal posterior distributions of BA model parameters obtained by applying \software{netabc} to a simulated transmission tree with BA parameter values $\alpha$ = 1.5, $I$ = 1000, $m$ = 4, and $N$ = 5000. Vertical dashed lines indicate true values. Shaded areas are 95\% highest posterior density intervals. $x$-axes indicate regions of nonzero prior density.
}
\label{fig:1.5-1000-4-5000-0}
\end{figure}
\clearpage

\begin{figure}[ht]
\centering
\includegraphics{{abc-posterior/0.0_2000_4_5000_0}.pdf}
\caption[Approximate marginal posterior distributions of BA model parameters for a simulated transmission tree with $\alpha$ = 0.0, $I$ = 2000, $m$ = 4, and $N$ = 5000.]{
Approximate marginal posterior distributions of BA model parameters obtained by applying \software{netabc} to a simulated transmission tree with BA parameter values $\alpha$ = 0.0, $I$ = 2000, $m$ = 4, and $N$ = 5000. Vertical dashed lines indicate true values. Shaded areas are 95\% highest posterior density intervals. $x$-axes indicate regions of nonzero prior density.
}
\label{fig:0.0-2000-4-5000-0}
\end{figure}
\clearpage

\begin{figure}[ht]
\centering
\includegraphics{{abc-posterior/0.5_2000_4_5000_0}.pdf}
\caption[Approximate marginal posterior distributions of BA model parameters for a simulated transmission tree with $\alpha$ = 0.5, $I$ = 2000, $m$ = 4, and $N$ = 5000.]{
Approximate marginal posterior distributions of BA model parameters obtained by applying \software{netabc} to a simulated transmission tree with BA parameter values $\alpha$ = 0.5, $I$ = 2000, $m$ = 4, and $N$ = 5000. Vertical dashed lines indicate true values. Shaded areas are 95\% highest posterior density intervals. $x$-axes indicate regions of nonzero prior density.
}
\label{fig:0.5-2000-4-5000-0}
\end{figure}
\clearpage

\begin{figure}[ht]
\centering
\includegraphics{{abc-posterior/1.0_2000_4_5000_0}.pdf}
\caption[Approximate marginal posterior distributions of BA model parameters for a simulated transmission tree with $\alpha$ = 1.0, $I$ = 2000, $m$ = 4, and $N$ = 5000.]{
Approximate marginal posterior distributions of BA model parameters obtained by applying \software{netabc} to a simulated transmission tree with BA parameter values $\alpha$ = 1.0, $I$ = 2000, $m$ = 4, and $N$ = 5000. Vertical dashed lines indicate true values. Shaded areas are 95\% highest posterior density intervals. $x$-axes indicate regions of nonzero prior density.
}
\label{fig:1.0-2000-4-5000-0}
\end{figure}
\clearpage

\begin{figure}[ht]
\centering
\includegraphics{{abc-posterior/1.5_2000_4_5000_0}.pdf}
\caption[Approximate marginal posterior distributions of BA model parameters for a simulated transmission tree with $\alpha$ = 1.5, $I$ = 2000, $m$ = 4, and $N$ = 5000.]{
Approximate marginal posterior distributions of BA model parameters obtained by applying \software{netabc} to a simulated transmission tree with BA parameter values $\alpha$ = 1.5, $I$ = 2000, $m$ = 4, and $N$ = 5000. Vertical dashed lines indicate true values. Shaded areas are 95\% highest posterior density intervals. $x$-axes indicate regions of nonzero prior density.
}
\label{fig:1.5-2000-4-5000-0}
\end{figure}
\clearpage



%\chapter{GLM results}
%<<glm_tables, echo=FALSE, warning=FALSE, error=FALSE, results="asis">>=
%    format.glm <- function (df) {
%        df[Parameter == "alpha", Parameter := "$\\alpha$"]
%        df[,p := paste0("$", sapply(p, pp, eq=FALSE), "$")]
%        df[,Estimate := paste0("$", sapply(Estimate, pp, eq=FALSE, thres=NA), "$")]
%        df[,`Standard error` := paste0("$", sapply(`Standard error`, pp, eq=FALSE, thres=NA), "$")]
%        setnames(df, "p", "$p$-value")
%        df
%    }
%
%    alpha.glm <- format.glm(alpha.glm)
%    I.glm <- format.glm(I.glm)
%    #m.glm <- format.glm(m.glm)
%    #N.glm <- format.glm(N.glm)
%
%    capt.short <- paste("Parameters of a fitted GLM relating error in estimated",
%                        "$\\alpha$ to true values of BA parameters.")
%    capt <- paste(capt.short, "GLM was fitted with a Gaussian distribution and",
%                  "inverse link function. Coefficients are interpretable as",
%                  "additive effects on the inverse of the mean error.")
%    print(xtable(alpha.glm, label="tab:glmalpha", caption=c(capt, capt.short)),
%          include.rownames=FALSE, sanitize.text.function=identity,
%          sanitize.colnames.function=identity)
%
%    capt.short <- paste("Parameters of a fitted GLM relating error in estimated",
%                        "$I$ to true values of BA parameters.")
%    capt <- paste(capt.short, "GLM was fitted with a gaussian distribution and",
%                  "inverse link function. Coefficients are interpretable as",
%                  "additive effects on the inverse of the mean error.")
%    print(xtable(I.glm, label="tab:glmI", caption=c(capt, capt.short)),
%          include.rownames=FALSE, sanitize.text.function=identity,
%          sanitize.colnames.function=identity)
%
%    #capt.short <- paste("Parameters of a fitted GLM relating error in estimated",
%    #                    "$m$ to true values of BA parameters.")
%    #capt <- paste(capt.short, "GLM was fitted with a Poisson distribution and",
%    #              "log link function. Coefficients are interpretable as",
%    #              "additive effects on the exponential of the mean error.")
%    #print(xtable(m.glm, label="tab:glmm", caption=c(capt, capt.short)),
%    #      include.rownames=FALSE, sanitize.text.function=identity,
%    #      sanitize.colnames.function=identity)
%
%    #capt.short <- paste("Parameters of a fitted GLM relating error in estimated",
%    #                    "$N$ to true values of BA parameters.")
%    #capt <- paste(capt.short, "GLM was fitted with a Gaussian distribution and",
%    #              "inverse link function. Coefficients are interpretable as",
%    #              "additive effects on the inverse of the mean error.")
%    #print(xtable(N.glm, label="tab:glmN", caption=c(capt, capt.short)),
%    #      include.rownames=FALSE, sanitize.text.function=identity,
%    #      sanitize.colnames.function=identity)
%@
