\documentclass[12pt]{article}
\usepackage[T1]{fontenc}
\usepackage{mathptmx}
\usepackage{fullpage}
\usepackage{amsmath,amssymb}
\usepackage[garamondx,cmbraces]{newtxmath}
\usepackage{setspace}
\usepackage{parskip}
\begin{document}

\onehalfspacing

In the time since circulating the draft thesis, and submitting the paper, I
have undertaken some additional work and restructuring. \\

\begin{itemize}
  \item I have done some more literature searching and found five additional
    published datasets to which my method could be applied. Out of these
    emerged an interesting trend: IDU networks seem to have higher $\alpha$
    than others. This is in line with the findings of a paper of Drombowski et
    al. 2013, who found that IDU networks are much to typical BA networks than
    sexual networks are.
  \item Two of these new datasets had sequenced two HIV genes, which we
    analysed separately. I found that estimates of $\alpha$ were very robust
    to the choice of gene. $I$ was less robust, with \textit{env} genes tending
    leading to higher estimated values.
  \item There is another study (de Blasio et al. 2007) which estimates $\alpha$
    from partner count data and I compare my estimates to theirs. They are
    similar.
  \item Prompted by one of Alex's comments, I found that the posterior mean is
    a more conventional estimator than the maximum \textit{a posteriori} (MAP)
    value, and have replaced MAP estimates with posterior means throughout the
    thesis.
  \item I searched the literature for estimated power law exponent ($\gamma$)
    values from real datasets (Table 2.9). Unfortunately, no pattern could be
    observed in the published estimates with respect to risk groups, genders,
    or time durations. 
  \item There are only two example posterior distributions now, instead of
    many in the text.
  \item The ``basic'' Bayesian statistics material (ie. what a posterior
    distribution is, what a parametric model is) have been moved into Appendix
    A instead of the introduction section on ABC.
  \item I found an analysis of the Barab\'asi-Albert model in the physics
    literature (Krapivsky et al. 2000) which demonstrates that the degree
    distribution is only a power law when $\alpha = 1$. I acknowledge this but
    do not believe it detracts from the validity of teh results, especially in
    light of the above mentioned de Blasio et al. paper. I added to the
    supplemental materials a series of power law fits to the degree
    distributions of BA networks with $\alpha \neq 1$ (Figure A.25), which
    shows that, although the power law is cleary an imperfect fit, the slope
    is still reasonably representative of the true distribution.
\end{itemize}

The following changes have been made at the request of the anonymous reviewers 
of the paper based on this work. \\

\begin{itemize}
  \item I repeated two of the real data analyses with trees created with RAxML
    and LSD rather than FastTree and root-to-tip regression. The results for
    $\alpha$ are very similar, but the estimates of $I$ are not very robust to
    the change of methods. This indicates weaker identifiability of $I$ than of
    $\alpha$.
  \item Figure 2.9 (formerly 2.8), which previously showed each simulation as
    an individual point, has been replaced by a set of stratified box plots
    showing the point estimates against true values.
  \item I had evaluated the ``classifiers'' for different parameter values
    using $R^2$, which doisn't make sense for classifiers. In fact, an SVR is a
    regressor, not a classifier, so the $R^2$ were for predicted against actual
    values. In any event, I have redone the analyses using SVMs instead of SVRs
    and regressions. The results are qualitatively very similar.
  \item I show two-dimensinal marginal posterior distributions in addition to
    the 1-dimensional ones, which provide evidence of parameter correlations
    (specifically, between $I$ and $N$).
  \item The reviewers pointed out that there are two sources of error: bad ABC
    approximation, and a noisy posterior. To disambiguate these, I performed
    marginal ABC estimation by fixing some parameters to known values. The
    estimates of all but $m$ improved by about 50\%.
  \item I also tried increasing the SMC settings to get better posterior
    resolution. It didn't make much difference (Figure 2.15). In fact it made
    the estimates slightly worse.
\end{itemize}

Changes that I have made in response to your specific comments are outlined
below. Minor grammatical and phrasing changes have been omitted. \\

\subsubsection*{Sally}

\begin{quote}
  \itshape
  In general (and partly suggested by Alex), I would highly recommend that you
  make minor revisions throughout to sign post to the reader why you are about
  to launch into, as you introduce a new topic.  You are strong on the
  technical details, but I often found myself wondering what the broader
  context was.  I think maybe you credit your reader for being super smart and
  not wanting to spell out the motivation for each section, but we aren’t that
  smart!
\end{quote}

Significantly more exposition has been added throughout the thesis. In
particular, each section of Chapter 2 now has some introductory material.
Sections 2.2.1 and 2.2.2 serve as motivation and a road map, respectively, for
the synthetic data experiments.

\begin{quote}
  \itshape
  For example:

  * How does sequential Monte Carlo (section 1.4) an ABC (section 1.5) fit into
  your overall thesis goals?  How do the ``particles'' introduced on page 13
  relate to the network structure and transmission trees just discussed?
  Basically, link the previous sections to sections 1.4 and 1.5.
\end{quote}

Several additional explanatory paragraphs have been prepended to both of these
sections. In particular, for SMC, ``In this work, the distribution of interest
is the posterior distribution 1.3. The particles are particular values of the
parameters $\theta$ of the contact network model being studied.'' For ABC,
``the observed data y is an estimated transmission tree for a viral epidemic
under investigation. The model in question is a contact network model with
parameters $\theta$ . The simulated dataset $z$ is a transmission tree,
obtained by first generating a contact network under the model, and then
simulating the spread of an epidemic over that network.''

\begin{quote}
  \itshape

  * Chapter 2 ends abruptly and needs a connecting paragraph.  How does this overview prepare you (and the reader) to develop the methods in Chapter 3?
\end{quote}

I think you are referring to Chapter 1 (Introduction) not Chapter 2. In any
case, I have added both an introductory paragraph (the last in the Objective
section) and a concluding paragraph to this chapter.

\begin{quote}
  \itshape

  * Similarly, Chapter 3 needs an introductory paragraph or two.  Late in your
  thesis, you note that netabc is the first attempt to use phylogenetic data
  (-> transmission trees) to infer network properties.  Remind the reader of
  this here and give them an overview of how you will proceed with the
  inference -- at a very broad-scale level.
\end{quote}

An introductory paragraph has been added which broadly describes the contents
of the chapter.

\begin{quote}
  \itshape
  
  * I really needed a road map for section 2.1.2.  Tell the reader why you are
  using different approaches for inference and what you expect their strengths
  and weaknesses to be.  I had to dig to tell what the differences were between
  the Classifiers subsection and the Grid search subsection.  Did the first
  co-estimate all parameters while the second focused on one at a time?  Were
  different methods use (yes - several versus only kernel)?  [Say so]
\end{quote}

Section 2.2.2 now motivates and describes, at a high level, each of the
synthetic data experiments and why they were done. In addition, the sections
describing each of the experiments (headings under 2.2.3) now contain one or
two introductory paragraphs each.

\begin{quote}
  \itshape

  * I found the discussion of the HIV datasets could also have used some
  greater interpretation.  For these datasets, what has been estimated as the
  likely (or at least known) number of infections?  How does this relate to the
  inferred I values?
\end{quote}

I have carefully read each of the studies for any available information about
the HIV prevalence in the study populations. In the case of the Chinese
datasets, I also consulted an article which reported the results of a
nationwide prevalence survey by city. Unfortunately, no general conclusions
could be drawn about how our estimates compare to values estimated with other
means. For some datasets (eg. IDU/Canada), $I$ was apparently overestimated,
while for some (MSM/USA) it was most likely underestimated.


\begin{quote}
  \itshape
  
  In addition, other more major comments are:

  * In the grid search, you held all of the other unknown parameters to their
  true values (if I understand correctly from Table 2.2), right?  You really
  have to make this explicit in the methods and discuss both there and in the
  results that this will give an overly confident view of the parameter values,
  compared with performing a grid search but allowing the other parameters to
  be inferred.  [Unless the latter is what you actually did.]
\end{quote}

This has been made much clearer both in section 2.2.2, and in the subsections
of both the methods and results where the grid search estimates are reported.
Figure captions emphasize this point as well.

\begin{quote}
  \itshape
  
  * I would consider adding box plots with interquartile ranges in places like
  Figure 2.7.  The 95\% CI are simply too broad to be able to compare the
  accuracy of the inferences.
\end{quote}

I felt that box plots would be confusing, since we do not report the median and
quantiles, but rather the posterior mean and highest density intervals.
However, I have added notches each plot of this type to indicate the 50\%
highest density intervals, which are similar to the interquartile range.

\begin{quote}
  \itshape

  * When you say that alpha is the most accurately estimated parameter (p. 37
  and elsewhere), I think that this is unfair because it is based on absolute
  differences and alpha, by definition, is a smaller number than I and N, say.
  Visually inspecting Figure 2.7, ``I'' is much better estimated (hugging the
  line more closely).  I think it would be better to discuss the proportional
  deviation (or $R^2$)  rather than absolute errors in this section.  Similarly,
  you compared some data points as being easier to estimate than others (e.g.,
  alpha = 1 and 1.5) but I do not see a trend that justifies picking out the
  best fitting two points and singling them out as ``special''.  In other words,
  your p value needs to be corrected for the fact that you might take any
  combination of the alpha values that seemed, by eye, to hug the regression
  line most closely.  Similarly, on p. 39 you note that the errors for I were
  higher when alpha was at least 1, but this is not at all visible and I doubt
  it is true (based on the bottom left panel in Figure 2.8), even before
  correcting for the fact that you’ve arbitrarily chosen one of several
  possible comparisons to make.
\end{quote}

Figure 2.7 shows only one out of ten replicate simulations for each parameter
value, so it may be somewhat visually misleading with respect to which
parameter is estimated best overall. I elected to represent the data this way,
instead of using box plots with all the simulations, in order to show the 95\%
and 50\% highest density intervals. I have noted this in the text. 

In any event, $I$ and $\alpha$ were in fact estimated with comparable accuracy. 
I now simply report the accuracy rates for each parameter and make some
qualitative observations about the distributions without claiming any
statistical significance.

With respect to $p$-value correction, I have also removed the ad-hoc
statistical tests in the analysis of the ABC experiments and replaced them with
two $p$-value corrected GLMs (Tables 2.5 and 2.6). \\

\subsubsection*{Alex}

\begin{quote}
  \itshape
  
  High-level comments:

  - Terminology: 'kernel ABC' in the way you use it ('ABC with epsilon
  threshold where the distance happens to be computed using the kernel trick')
  might be confusing to the ABC folks, since there is already some work on
  'kernel ABC' used for a different meaning ('ABC where there is not a hard
  threshold on the distance, instead, a kernel is used to weigh the generated
  hypotheses', see for example http://arxiv.org/abs/1009.5736). 
\end{quote}

I have replaced ``kernel-ABC'' with either ``ABC'' or \textit{netabc}
throughout the thesis.

\begin{quote}
  \itshape

  - The material in section 2 is all there and good, but I feel it could be
  reorganized a little bit to make it easier to follow. I would suggest making
  one section with only the description of the method. Then another section for
  the experiments on synthetic data. And then one more for real data analysis.
  The section on synthetic data would also benefit from a short introductory
  paragraph explaining that you first present exploratory data analysis
  (without SMC-ABC) to assess the strength of the signal, and then do the
  SMC-ABC analysis (and subdivide the section into corresponding subsection). 
\end{quote}

The chapter is now divided into four sections: description of the method,
sythetic data analyses, real data analyses, and discussion.

\begin{quote}
  \itshape
  Detailed comments:

  ``were not estimable with kernel-ABC'' -> ''appeared to be weakly or
  non-identifiable with kernel-ABC''. Estimable is typically reserved for the
  more specialized cases of linear models.
\end{quote}

I have replaced all instances of ``estimable'' with either ``identifiable'' or
another term appropriate to the context (ie. parameters were estimable $\to$
classifiers were accurate, etc.).

\begin{quote}
  \itshape

  Argument at bottom of page 14/132 and top of page 15/132: I didn't fully
  understand why you needed labelling---I suspect it will become clearer later
  in the manuscript but in terms of exposition it is a bit hard to follow.
  Perhaps move this section to later, i.e. after having introduced more
  notation? Also, note that showing that ML/Bayes requires a search over an
  exponential space does not necessarily rules out ML/Bayes. For example
  searching over tree is NP hard yet ML/Bayes are still good options in
  practice. A more convincing argument would be that the natural formulation of
  the problem falls in what is called a 'doubly-intractable problem', i.e. a
  search problem over an intractable space where evaluation of the likelihood
  of individual candidates is also computationally hard.
\end{quote}

I have moved that argument down into Chapter 2 (section 2.1.4), and instead
focus on a higher level discussion for the objective. Section 2.1.4 also
contains some explanation of the idea of doubly-intractable problems, and a new
figure showing that a transmission tree topology has several possible internal
node labellings.

\begin{quote}
  \itshape

  ``The tips of a phylogeny, that is, the nodes without any descendants,
  correspond to extant, or observed, taxa.'' -> in a phylodynamics context,
  this is not so accurate of a characterization of the tips.
\end{quote}

I have changed the sentance to read as follows: ``The \emph{internal nodes}            
correspond to their common ancestors, usually extinct (although occasionally
the internal nodes may be observed as well, eg. Jombart et al. 2011).''

\begin{quote}
  \itshape
  Same page: "large matches " -> what does that mean?

  A question, perhaps stupid, on the tree kernel, more specifically on the
  parts of the definition comparing branch lengths: won't that be sensitive to
  the order in which the branch lengths are enumerated? Since the children in a
  tree are unordered, do you need to sort them or something like that?
\end{quote}

Two paragraphs have been added to section 1.2.4 describing each of the terms in
more detail. The paragraphs also clarify that we usually ladderize the tree
before applying the tree kernel.

\begin{quote}
  \itshape

  ``Sequential importance sampling (SIS) [113] is one type of SMC method,'' ->
  it is rather a precursor of, technically.
\end{quote}

This is mentioned both in the section describing SIS and in the introductory
paragraphs at the beginning of the section.

\begin{quote}
  \itshape

  Equation on top of page 27/132, although technically correct, is not quite
  what is the basis for IS; note that the form used would have weights at the
  top and bottom (top for weighting, bottom for self-normalizing).
\end{quote}

When trying to understand IS, I found the formulation I had previously written
to be more intuitive. However, I have replaced it with the expression
\[
  \int f(x) \pi(x) dx = \frac{1}{Z} \int w(x) \eta(x) f(x) dx,
\]
which I think is more conventional. Here, $\pi$ is the distribution of interest,
$Z$ is its (unknown) normalizing constant, $f$ is an integrable function whose
expectation we want to calculate under $\pi$, $w$ is the importance weight, and
$\eta$ is the importance distribution.

\begin{quote}
  \itshape

  ``then $\eta(x)$ can be approximated by a Monte Carlo estimate.'' : in IS we
  do not approximate the importance distribution, rather we pick a simple one
  and sample from it exactly.
\end{quote}

This was a poor choice of words on my part. I have edited the statement as
follows: ``Since $\eta$ can be sampled from exactly, and $\gamma$ and $f$ can
both be evaluated pointwise, the integral $\int w(x) \eta(x) f(x) \d x$ can be
approximated by a Monte Carlo estimate.''

\begin{quote}
  \itshape

  ``The recursive definition eq. (1.3) suggests an algorithm for obtaining a
  sample from $\pi$ (algorithm 1).'' -> note that this algorithm does not
  create an exact sample from $\pi$, but rater an approximate sample, an
  important distinction to make.
\end{quote}

The sentence now reads as follows: ``The recursive definition 1.3 suggests an
algorithm for obtaining a sample from $\eta$ and using it to obtain an
approximate sample from $\pi$ by IS.''

\begin{quote}
  \itshape

  In SIS, you only need to normalize the weights in the final step, and ideally
  you shouldn't do resampling in the last iteration. (similarly, in SMC, you
  should not do resampling in the final iteration. Technically, you still get a
  consistent algorithm if you do resampling in the last step, but for
  efficiency reasons it should be avoided if possible, this is motivated by a
  theoretical analysis of the variance of the weights, described for example in
  the book by Del Moral, 'Feynman-Kac Formulae, Genealogical and Interacting
  Particle Systems with Applications')
\end{quote}

I'm glad you brought this to my attention, as I was making this mistake in all
the experiments. Fortunately it only affected post-processing, so I didn't have
to rerun everything. I have removed the final resampling step from all the
algorithms in this section, and from my analyses as well. When discussing this
point, you mentioned that the posterior expectation is a commonly used
estimator, which prompted me to replace my MAP estimates with posterior means
throughout.

\begin{quote}
  \itshape

  Your background on SMC is currently organized as follows:
  - IS
  - SIS
  - SMC samplers
  Note that a perhaps more natural organization of these ideas (in the sense
  that each in this progression introduces one fundamental new thing at the
  time) would be:
  - IS
  - SIS (which adds online capability)
  - SMC (which introduces the resampling operator to fight degeneracy)
  - 'SMC samplers' (which introduces the capacity to approach spaces that
  cannot be expressed as product spaces)
  Note that confusingly, 'SMC' and 'SMC samplers' are used for different ideas
  (the latter builds on the former). 
\end{quote}

I did not initially realize that the resampling was introduced as part of SMC
and is not part of SIS. I have moved resampling into a separate subsection
about only SMC, so the section now has the suggested structure.

\begin{quote}
  \itshape

  Related comment: one important idea missing in the background is the notion
  of particle degeneracy, which is the fundamental reason why resampling is
  theoretically justified.
\end{quote}

This point has been added to the section describing SMC: ``\ldots the
cumulative errors at each sequential step tend to push many of the weights to
very low values. This results in a poor approximation to $\pi$, since only a
few particles retain high importance weights after all $d$ sequential steps, a
problem known as \emph{particle degeneracy}.''

\begin{quote}
  \itshape
  
  ``data, x and outcomes y'' -> `outcome' seems a weird terminology name here.
  Perhaps covariate would be better.
\end{quote}

I had actually intended $y$ to be the data and $x$ to be the covariates: I was
thinking of covariates as ``inputs'' and data as ``outputs''. This text has
been moved into Appendix A and changed to read ``observed data $y$ and
covariates $x$.''

\begin{quote}
  \itshape

  ``suitably chosen distance $\rho$'' -> It might be good to include a short
  paragraph expending on this, since these conditions are not only for
  theoretical analysis but also have a big impact in practice. 
\end{quote}

There is now a subheading ``Distance functions and summary statistics in ABC''
which describes sufficient statistics and references some methods to choose
optimal summary statistics.

\begin{quote}
  \itshape

  ``We use the variable x'' -> point out that this definition is unrelated to
  the definition of x used earlier in this section
\end{quote}

Rather than change the wording, the previous usage has been moved into Appendix
A.

\begin{quote}
  \itshape

  ``Classifiers for BA model parameters based on tree shape'' -> specify
  explicitly here that you are doing synthetic data experiments. 
\end{quote}

This is now emphasized in both the new section 2.2.2 and the introductory
paragraphs below the relevant subheading in section 2.2.3.

\end{document}
