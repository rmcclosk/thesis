Contact network structure has a substantial impact on epidemic trajectory
\autocite{barthelemy2005dynamical, volz2007susceptible, volz2008sir}, and a few
methods have been developed to estimate network parameters from epidemiological
data \autocite{volz2007susceptible, britton2002bayesian,
groendyke2011bayesian}. It is known that contact network structure can have a
substantial impact on transmission tree shape \autocite{o2010contact,
robinson2013dynamics, leventhal2012inferring, colijn2014phylogenetic,
villandre2016assessment}.

Our work had three main aims. First, we developed a method to estimate contact
network model parameters from phylogenetic data. This method widens the field
of models which may be investigated in a phylodynamic context. Second, we
investigated the parameters of the \gls{BA} model. We determined through
simulation studies that the preferential attachment power $\alpha$ and number
of infected nodes $I$ had a substantial impact on transmission tree shape, and
could be estimated using our method. 

% ergm

An alternative approach is the deterministic framework outlined by
\textcite{morris1993epidemiology}, who proposes to apply the standard
compartmental modelling framework to contact networks by assigning each
individual their own compartment. Thus, each individual is associated with a
single \gls{ODE}, with the entire \gls{ODE} system parameterized by the
adjacency matrix of the contact network. \citeauthor{morris1993epidemiology}
proposes to use log-linear models to parameterize the matrix. This framework is
highly expressive, and allows straightforward incorporation of time-dependent
dynamics. However, simulating a transmission tree would require the numerical
solution of a very large system of \glspl{ODE}. Given the large number of
simulations required for kernel-\gls{ABC}, it is not clear if this method would
be computationally feasible in this context.

% spatial networks, see section 5.4 in keeling2005networks

The two-step process of simulating a contact network and subsequently allowing
an epidemic to spread over that network carries with it the assumption that the
contact network is static over the duration of the epidemic. Clearly this
assumption is invalid, as people make and break partnerships on a regular
basis. Our work has not addressed this assumption, primarily due to our desire
to avoid the additional complexity required to address the dynamic nature of
networks. This simplifying assumption is made by most studies using contact
network models in an epidemiological context~\autocite{welch2011statistical,
bansal2007individual}. However, in principle, kernel-\gls{ABC} could be
adapted to dynamic contact networks by using a method such as that developed by
\textcite{robinson2012dynamics} to simulate a dynamic contact network, while
concurrently simulating the spread of an epidemic.

% Caveats

It is important to note that our kernel-ABC method takes a transmission tree as
input, rather than a viral phylogeny. Thus, we have left the estimation of a
transmission tree up to the user. There were two reasons for this choice. First
and foremost, we wished again to avoid extra complexity and keep the number of
estimated parameters small. In theory, it is possible to incorporate the
process by which a viral phylogeny is generated along with a transmission tree
into our method, for example by simulating within-host dynamics. Although this
may be an avenue for future extension, we felt that it would obscure the
primary purpose of this work, which is to study contact network parameters.
Second, there are a number of different methods available for inferring
transmission trees~\autocite{didelot2014bayesian, ypma2012unravelling,
jombart2011reconstructing, cottam2008integrating, poon2015phylodynamic}, some
of which incorporate geographic and/or epidemiological data not accommodated by
our method. We therefore felt it would be best to allow researchers to use
their own preferred tree building method.

Our use of the \gls{BA} model makes several simplifying assumptions. First, we
assume homogeneity across the network with respect to node behaviour and
transmission risk. In reality, the attraction to high-degree nodes seems likely
to vary among individuals, as does their risk of transmitting or contracting
the virus. We have also assumed that all transmission risks are symmetric,
which is clearly false for all known modes of \gls{HIV} transmission, and that
infected individuals never recover but remain infectious indefinitely. These
assumptions were made for the purpose of keeping the model as simple as
possible, since this is the very first attempt to fit a contact network model
in a phylodynamic context. However, the Gillespie simulation algorithm built
into \software{netabc} can handle arbitrary transmission and removal rates
which need not be homogeneous across the network. Moreover, it is possible to
use kernel-\gls{ABC} to fit a model which relaxes some or all of these
assumptions, which may be a fruitful avenue for future investigation.
