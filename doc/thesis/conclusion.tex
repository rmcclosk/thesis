Due to the rapid advancement of nucleotide sequencing technology, viral
sequence data data have become increasingly feasible to collect on a population
level. Through phylodynamic methods, these data offer a window into
epidemiological processes which would otherwise be virtually impossible to
study on a realistic scale. 

This thesis developed \software{netabc}, a computer program implementing a
statistical inference method for contact network parameters from viral
phylogenetic data. \software{Netabc} brings together the areas of viral
phylodynamics and network epidemiology, which have only intersected in a very
limited fashion thus far~\autocite{welch2011statistical}. The use of
kernel-\gls{ABC}, a likelihood-free method, makes it possible to fit network
models to phylogenies without calculating intractable likelihoods.

Although phylodynamic methods have been developed to fit a wide variety of
epidemiological models to phylogenetic data assuming homogeneous
mixing~\autocite{volz2012complex, rasmussen2014phylodynamic}, our method is
able to fit models not requiring this assumption. We believe this capability
will be of broad interest to the molecular evolution and epidemiology
community, as it widens the field of epidemiological parameters which may be
investigated through viral sequence data. In addition, the characterization of
local contact networks could be valuable from a public health perspective, such
as for investigating optimal vaccination
strategies~\autocite{keeling2005networks, peng2013vaccination,
ma2013importance, rushmore2014network}. This information could assist in
curtailing current epidemics, as well as preventing future epidemics of
different diseases over the same contact network.

The particular model we have investigated uses a preferential attachment
mechanism to generate scale-free networks resembling real-world social and
sexual networks~\autocite{liljeros2001web, schneeberger2004scale,
colgate1989risk}. Of the four parameters we considered, the preferential
attachment power \gls{alpha} was the most readily estimable. Estimating
\gls{alpha} with traditional epidemiological methods is challenging due to the
requirement of sampling the high-degree nodes making up the tail of the power
law distribution, although approaches such as respondent-driven
sampling~\autocite{heckathorn1997respondent} \add{or collection of longitudinal
partner count data~\autocite{de2007preferential}} may be effective.

In closing, \software{netabc} combines phylodynamics, contact network
epidemiology, approximate Bayesian computation, and sequential Monte Carlo 
to provide a source of insight into network structures complementary to
traditional epidemiology.
