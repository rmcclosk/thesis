Contact network structure has a substantial impact on epidemic trajectory
\autocite{barthelemy2005dynamical, volz2007susceptible, volz2008sir}, and a few
methods have been developed to estimate network parameters from epidemiological
data \autocite{volz2007susceptible, britton2002bayesian,
groendyke2011bayesian}. It is known that contact network structure can have a
substantial impact on transmission tree shape \autocite{o2010contact,
robinson2013dynamics, leventhal2012inferring, colijn2014phylogenetic,
villandre2016assessment}.

Our work had three main aims. First, we developed a method to estimate contact
network model parameters from phylogenetic data, based on kernel approximate
Bayesian computation. This method widens the field of models which may be
investigated in a phylodynamic context.

Our second aim was to investigate the parameters of the \gls{BA} model, in
terms of their impact on tree shape and our ability to reconstruct them with
kernel-\gls{ABC}. We determined through simulation studies that the
preferential attachment power $\alpha$ and number of infected nodes $I$ had a
substantial impact on transmission tree shape, and could be estimated using our
method.

The final aim of our work was to apply our method to real world \gls{HIV}
datasets. This analysis indicated heterogeneity in the contact network
structures underlying several distinct local epidemics. The five datasets
analysed fell into three categories (Figure~\ref{fig:abchpd}). First, we
estimated a preferential attachment power between 0.5 and 1 for the epidemics
studied by \textcite{cuevas2009hiv} and \textcite{li2015hiv}, with credible
intervals occupying nearly the entire region from 0 to 1.
\citeauthor{cuevas2009hiv} studied a group of newly diagnosed individuals in
the Basque Country, Spain. Although the individuals were of mixed risk groups,
and therefore unlikely to comprise a single contact network, a high proportion
of them (47\%) grouped into local clusters based on genetic distance. The low
estimated attachment power for these data is consistent with the sampled
sequences comprising many distinct sub-networks rather than a single connected
network. \citeauthor{li2015hiv} sampled a large number of acutely infected MSM
in Shanghai, China, in which we identified a large cluster from the phylogeny
using a patristic distance cutoff~\autocite{poon2015impact}. The low attachment
power estimated for this dataset was surprising given the high phylogenetic
relatedness of the sequences. It is possible that the number and diversity of
circulating recombinant forms in the data introduced errors into our estimated
viral phylogeny.

For the outbreaks studied by \textcite{niculescu2015recent} and
\textcite{wang2015targeting}, the estimated $\alpha$ was close to one, with a
narrower credible interval than for the other studies.
\citeauthor{niculescu2015recent} studied a recent outbreak among Romanian
injection drug users (IDU), while \citeauthor{wang2015targeting} sampled
acutely infected MSM in Beijing, China. Both studies discovered a high degree
of phylogenetic relatedness owing to the recent infection times and homogeneous
risk groups of the studied populations. The estimated number of infections for
these datasets were also quite low, although the HPD interval for
\citeauthor{wang2015targeting} was much wider than that for
\citeauthor{niculescu2015recent}.

The final studied dataset was an outlier in terms of estimated parameters.
\textcite{novitsky2013phylogenetic} sampled approximately 44\% of the
HIV-infected individuals in the northern area of Mochudi, Botswana. Additional
sampling in a later study~\autocite{novitsky2014impact} brought the genotyping
coverage up to 70\%. Even with such a high sampling coverage, we did not detect
any large clusters using patristic distance, and therefore chose to analyze a
subtree instead. Estimates of $\alpha$ and $N$ both had very wide HPD
intervals and were markedly different from the other datasets. The estimated
number of infected nodes was also extremely high, much higher than the
estimated HIV prevalence of the town. Several factors may have contributed to
these results. First, the authors note that the their sample was 75\% female.
In a primarily heterosexual risk environment, removal of a disproportionate
number of males from the network could obfuscate the true network structure,
for example if the majority of highly connected nodes were of one gender.
Second, the town in question was in close proximity to the country's capital,
and the authors indicated that a high amount of migration takes place between
the two locations. This suggests that the contact network may include a much
larger group based in the capital city, which would explain the high estimate
of $I$.

% ergm

An alternative approach is the deterministic framework outlined by
\textcite{morris1993epidemiology}, who proposes to apply the standard
compartmental modelling framework to contact networks by assigning each
individual their own compartment. Thus, each individual is associated with a
single \gls{ODE}, with the entire \gls{ODE} system parameterized by the
adjacency matrix of the contact network. \citeauthor{morris1993epidemiology}
proposes to use log-linear models to parameterize the matrix. This framework is
highly expressive, and allows straightforward incorporation of time-dependent
dynamics. However, simulating a transmission tree would require the numerical
solution of a very large system of \glspl{ODE}. Given the large number of
simulations required for kernel-\gls{ABC}, it is not clear if this method would
be computationally feasible in this context.

% spatial networks, see section 5.4 in keeling2005networks

The two-step process of simulating a contact network and subsequently allowing
an epidemic to spread over that network carries with it the assumption that the
contact network is static over the duration of the epidemic. Clearly this
assumption is invalid, as people make and break partnerships on a regular
basis. Our work has not addressed this assumption, primarily due to our desire
to avoid the additional complexity required to address the dynamic nature of
networks. This simplifying assumption is made by most studies using contact
network models in an epidemiological context~\autocite{welch2011statistical,
bansal2007individual}. However, in principle, kernel-\gls{ABC} could be
adapted to dynamic contact networks by using a method such as that developed by
\textcite{robinson2012dynamics} to simulate a dynamic contact network, while
concurrently simulating the spread of an epidemic.

% Caveats

It is important to note that our kernel-ABC method takes a transmission tree as
input, rather than a viral phylogeny. Thus, we have left the estimation of a
transmission tree up to the user. There were two reasons for this choice. First
and foremost, we wished again to avoid extra complexity and keep the number of
estimated parameters small. In theory, it is possible to incorporate the
process by which a viral phylogeny is generated along with a transmission tree
into our method, for example by simulating within-host dynamics. Although this
may be an avenue for future extension, we felt that it would obscure the
primary purpose of this work, which is to study contact network parameters.
Second, there are a number of different methods available for inferring
transmission trees~\autocite{didelot2014bayesian, ypma2012unravelling,
jombart2011reconstructing, cottam2008integrating, poon2015phylodynamic}, some
of which incorporate geographic and/or epidemiological data not accommodated by
our method. We therefore felt it would be best to allow researchers to use
their own preferred tree building method.

Our use of the \gls{BA} model makes several simplifying assumptions. First, we
assume homogeneity across the network with respect to node behaviour and
transmission risk. In reality, the attraction to high-degree nodes seems likely
to vary among individuals, as does their risk of transmitting or contracting
the virus. We have also assumed that all transmission risks are symmetric,
which is clearly false for all known modes of \gls{HIV} transmission, and that
infected individuals never recover but remain infectious indefinitely. These
assumptions were made for the purpose of keeping the model as simple as
possible, since this is the very first attempt to fit a contact network model
in a phylodynamic context. However, the Gillespie simulation algorithm built
into \software{netabc} can handle arbitrary transmission and removal rates
which need not be homogeneous across the network. Moreover, it is possible to
use kernel-\gls{ABC} to fit a model which relaxes some or all of these
assumptions, which may be a fruitful avenue for future investigation.
