\section{Extensions}

% ergm

An alternative approach is the deterministic framework outlined by
\textcite{morris1993epidemiology}, who proposes to apply the standard
compartmental modelling framework to contact networks by assigning each
individual their own compartment. Thus, each individual is associated with a
single \gls{ODE}, with the entire \gls{ODE} system parameterized by the
adjacency matrix of the contact network. \citeauthor{morris1993epidemiology}
proposes to use log-linear models to parameterize the matrix. This framework is
highly expressive, and allows straightforward incorporation of time-dependent
dynamics. However, simulating a transmission tree would require the numerical
solution of a very large system of \glspl{ODE}. Given the large number of
simulations required for kernel-\gls{ABC}, it is not clear if this method would
be computationally feasible in this context.

% spatial networks, see section 5.4 in keeling2005networks

% Caveats

It is important to note that our kernel-ABC method takes a transmission tree as
input, rather than a viral phylogeny. Thus, we have left the estimation of a
transmission tree up to the user. There were two reasons for this choice. First
and foremost, we wished to avoid the extra complexity and keep the number of
estimated parameters small. In theory, it is possible to incorporate the
process by which a viral phylogeny is generated along with a transmission tree
into our method, for example by simulating within-host dynamics. Although this
may be an avenue for future extension, we felt that it would obscure the
primary purpose of this work, which is to study contact network parameters.
Second, there are a number of different methods available for inferring
transmission trees~\autocite{didelot2014bayesian, yima2012unravelling,
jombart2011reconstructing, cottam2008integrating, poon2015phylodynamic}, some
of which incorporate geographic and/or epidemiological data not accomodated by
our method. We therefore felt it would be best to allow researchers to use
their own preferred tree building method.
