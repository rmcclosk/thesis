Due to the rapid advancement of nucleotide sequencing technology, viral
sequence data data have become increasingly feasible to collect on a population
level. Through phylodynamic methods, these data offer a window into
epidemiological processes which would otherwise be virtually impossible to
study on a realistic scale. 

This thesis developed \software{netabc}, a computer program implementing a
statistical inference method for contact network parameters from viral
phylogenetic data. \software{Netabc} brings together the areas of viral
phylodynamics and network epidemiology, which have only intersected in a very
limited fashion thus far~\autocite{welch2011statistical}. The use of
kernel-\gls{ABC}, a likelihood-free method, makes it possible to fit network
models to phylogenies without calculating intractible likelihoods.

Although phylodynamic methods have been developed to fit a wide variety of
epidemiological models to phylogenetic data~\autocite{volz2012complex,
rasmussen2014phylodynamic}, our method is the first (to our knowledge) which
can fit models that do not assume panmixia. We believe this capability will be
of broad interest to the molecular evolution and epidemiology community, as it
widens the field of epidemiological parameters which may be investigated
through viral sequence data. In addition, the characterization of local contact
networks could be valuable from a public health perspective, such as for
investigating optimal vaccination strategies. This information could assist in
curtailing current epimemics, as well as preventing future epidemics of
different diseases over the same contact network.
