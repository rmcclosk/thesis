

\label{sec:disc}

\subsection{\software{Netabc}: uses, limitations, and possible extensions}

Contact networks can have a strong influence on epidemic progression, and are
potentially useful as a public health tool~\autocite{wang2015targeting,
little2014using}. Despite this, few methods exist for investigating contact
network parameters in a phylodynamic framework \autocite[although see][for
related work]{groendyke2011bayesian, volz2008sir, brown2011transmission,
leventhal2012inferring, greenbaum2016inference}. Kernel-ABC is a model-agnostic
method which can be used to investigate any quantity that affects tree
shape~\autocite{poon2015phylodynamic}. In this work, we developed
\software{netabc}, a method based on kernel-ABC to infer the parameters of a
contact network model. The method is general, meaning that it can be used to
infer parameters of any network model, as long as it allows simulated networks
can be easily generated. We have included generators for the \gls{BA} model
discussed here, as well as the \gls{ER} and \gls{WS} network models.
Instructions for adding additional models are available in the project's online
documentation. We have made \software{netabc} publicly available at
\url{github.com/rmcclosk/netabc} under a permissive open source license, to
encourage other researchers to apply and extend our method.

Several alternative network models and modelling frameworks have been developed
which may provide useful future targets for kernel-assisted \gls{ABC}. Waring
models~\autocite{irwin1963place,handcock2004likelihood} are a more flexible
type of preferential attachment model which permit a subset of attachments to
be formed non-preferentially. These models were used by
\textcite{brown2011transmission} to characterize the transmission network in
the United Kingdom. \Glspl{ERGM}~\autocite{robins2007introduction} are a
flexible and expressive parameterization of contact networks in terms of
statistics of network features such as pairs and triads.
\textcite{goodreau2006assessing} evaluated the effect of several different
\gls{ERGM} parameterizations on transmission tree shape and effective
population size. The author suggested the use of \glspl{ERGM} as a general
framework for estimation of epidemiological quantities related to \gls{HIV}
transmission. Except for a few special cases, simulating a network according to
an \gls{ERGM} generally requires \gls{MCMC}, which would be too computationally
intensive to integrate into \software{netabc} as it currently stands. To fit
\gls{ERGM} with kernel-assisted \gls{ABC}, one possibility would be to consider the
network itself as a parameter to be modified by the \gls{MCMC} kernel.
Other network modelling frameworks include the partnership-centric formulation
developed by~\textcite{eames2002modeling} and the log-linear adjacency matrix
parameterization applied by~\textcite{morris1993epidemiology}.

%\Textcite{morris1993epidemiology} proposes to apply the standard compartmental
%modelling framework to contact networks by assigning each individual their own
%compartment. Thus, each individual is associated with a single \gls{ODE}, with
%the entire \gls{ODE} system parameterized by the adjacency matrix of the
%contact network. 
%The author proposes to use log-linear models to parameterize the adjacency
%5matrix of a contact network. This framework is highly expressive, and allows
%straightforward incorporation of time-dependent dynamics. 
%However, simulating a transmission tree would require the numerical solution of
%a very large system of \glspl{ODE}, which may be computationally prohibitive
%given the number of simulations required for kernel-assisted \gls{ABC}.

The two-step process of simulating a contact network and subsequently allowing
an epidemic to spread over that network carries with it the assumption that the
contact network is static over the duration of the epidemic. Clearly this
assumption is invalid, as people make and break partnerships on a regular
basis. Addressing the impact of this simplifying assumption is outside the
scope of this work. However, the same assumption is made by most studies using
contact network models in an epidemiological
context~\autocite{welch2011statistical, bansal2007individual}. In principle,
kernel-assisted \gls{ABC} could be adapted to dynamic contact networks by using
a method such as that developed by \textcite{robinson2012dynamics} to simulate
such a network, while concurrently simulating the spread of an epidemic.

It is important to note that \software{netabc} takes a transmission tree as
input, rather than a viral phylogeny. In reality, true transmission trees are
not available and must be estimated; these estimates are often based on the
viral phylogeny. Although this has been demonstrated to be a fair
approximation~\autocite[e.g.][]{leitner1996accurate}, and is frequently used in
practice~\autocite[e.g.][]{stadler2013uncovering}, the topologies of a viral
phylogeny and transmission tree can differ
significantly~\autocite{ypma2013relating, hall2015epidemic} due to within-host
evolution and the sampling process. We have left the estimation of a
transmission tree up to the user. In theory, it is possible to incorporate the
process by which a viral phylogeny is generated along with a transmission tree
into our method, for example by simulating within-host dynamics. {\color{blue}
\uline{Indeed, progress has very recently been demonstrated on this front in a
talk by \citeauthor{giardina2016inference}, who have independently developed in
a method similar to ours to fit contact network models to phylogenetic data
that additionally incorporates a within-host evolutionary model.}} Although
this may be an avenue for future extension, we felt that it would obscure the
primary purpose of this work, which is to study contact network parameters. In
addition, there are a number of different methods available for inferring
transmission trees~\autocite{didelot2014bayesian, ypma2012unravelling,
jombart2011reconstructing, cottam2008integrating, hall2015epidemic}, some of
which incorporate geographic and/or epidemiological data not accommodated by
our method. We therefore felt it would be best to allow researchers to use
their own preferred method of constructing a transmission tree.

Our implementation of \gls{SMC} uses a simple multinomial scheme to sample
particles from the population according to their weights. Several other
sampling strategies have been developed~\autocite{douc2005comparison}, and it
is possible that the use of a more sophisticated technique might increase the
algorithm's accuracy. Finally, the \gls{ABC}-\gls{SMC} algorithm is
computationally intensive, taking about a day when run on 20 cores in parallel
with the settings described in the methods section. Implementing
parallelization using \gls{MPI}, rather than \gls{POSIX} threads as we have
done here, would allow the program to be run over a larger number of cores on
multiple CPUs in parallel.

\subsection{Analysis of \acrlong{BA} model with synthetic data}

The preferential attachment power \gls{alpha} had a very strong influence on
tree shape in the range of values we considered
(\cref{fig:alphatrees,fig:kpca}). Although the tree kernel was the most
effective classifier for \gls{alpha}, a Sackin's index of tree imbalance
performed nearly as well (\cref{fig:rsquared}). This result was intuitive: high
\gls{alpha} values produce networks with few well-connected ``superspreader''
nodes which are involved in a large number of transmissions, resulting in a
highly unbalanced ladder-like tree structure (\cref{fig:alphatrees}). 
There was no observable bias in the estimates of \gls{alpha} obtained with
\software{netabc}, however {\color{red}\sout{the variation in these estimates
was higher for $\alpha < 1$ than for $\alpha \geq 1$}} {\color{blue}\uline{there
appeared to be weaker identifiability for \gls{alpha} $<$ 1 than for \gls{alpha}
$\geq$ 1}} (\cref{fig:abcptm2,tab:abchpd}). The relationship between
\gls{alpha} and the power law exponent \gls{gamma} may explain this result
(\cref{fig:gamma}). The \gls{gamma} values associated with \gls{alpha} = 0 and
\gls{alpha} = 0.5 are nearly identical (about 2.28 for \gls{alpha} = 0 and 2.33
for \gls{alpha} = 0.5 with \gls{N} = 5000 and \gls{m} = 2). In other words, the
degree distributions of networks with \gls{alpha} $< 1$ are similar to each
other, which may result in similarity of corresponding transmission trees as
well. 
%The accuracy of \gls{alpha} estimates also increased when the value of the
%\gls{m} parameter was higher (compare
%\cref{fig:abcptm2,fig:abcptm3,fig:abcptm4}). 

The \gls{I} parameter, representing the prevalence at the time of sampling, was
also generally identifiable, although it was slightly over-estimated for both
cases we considered with kernel-assisted \gls{ABC}. The dynamics of the \gls{SI} model,
and the coalescent process~\autocite{kingman1982coalescent}, offer a potential
explanation for the identifiability of \gls{I}. In our simulations, we assumed
that all discordant edges shared the same transmission rate, so that the
waiting time until the next transmission in the entire network was always
inversely proportional to the number of discordant edges. In the initial phase
of the epidemic, when \gls{I} is small, each new transmission results in many
new discordant edges. Hence, there is an early exponential growth phase,
producing many short branches near the root of the tree. As the epidemic gets
closer to saturating the network, the number of discordant edges decays,
causing longer waiting times. The distribution of coalescence times in the tree
should therefore be informative about \gls{I}~\autocite{volz2009phylodynamics}.
This information is captured by the tree kernel, and also by the \gls{nltt}
statistic, which both performed quite well in classifying \gls{I}
(\cref{fig:rsquared}). 

The number of nodes in the network, \gls{N}, exhibited the most variation in
terms of its effect on tree shape. There was almost no difference between trees
simulated under different \gls{N} values when the number of infected nodes
\gls{I} was small. There is an intuitive explanation for this result, namely
that adding additional nodes does not change the edge density or overall shape
of a \gls{BA} network. This can be illustrated by imagining that we add a small
number of nodes to a network after the epidemic simulation has already been
completed. It is possible that none of these new nodes attains a connection to
any infected node. Thus, running the simulation again on the new, larger
network could produce the exact same transmission tree as before. On the other
hand, when \gls{I} is large relative to \gls{N}, the coalescent dynamics
discussed above also apply. That is, the waiting times until the next infection
increase, resulting in longer coalescence times toward the tips. The relative
accuracy of the \gls{nltt} in these situations
(\cref{fig:rsquared,fig:Ncrossv}) corroborates this hypothesis, as the
\gls{nltt} uses only information about the coalescence times. When all \gls{BA}
parameters were simultaneously estimated with kernel-assisted \gls{ABC}, \gls{N} was
nearly always over-estimated by approximately a factor of two
(\cref{fig:abcptm2,tab:abchpd}). One factor which may have contributed to this
bias was our choice of prior distribution. Since the prior for \gls{I} and
\gls{N} was jointly uniform on a region where $I \leq N$, more prior weight was
assigned to higher \gls{N} values. Another contributing factor relates to the
dynamics of the \gls{SI} model and the coalescent process.

\gls{I} and \gls{N} were both systematically over-estimated by
\software{netabc}, although the bias was more severe for \gls{N} than for
\gls{I}. The number of infected individuals follows a logistic growth curve
under the \gls{SI} model. This kind of growth curve has three qualitative
phases: a slow ramp-up, an exponential growth phase, and a slow final phase
when the susceptible population is almost depleted. The waiting times until the
next transmission, which determine the coalescence times in the tree, are
dependent on the growth phase of the epidemic. Therefore, we hypothesize that
it is the growth phase at the time of sampling which most affects tree shape,
rather than the specific values of \gls{I} or \gls{N}. To investigate this
hypothesis, we simulated transmission trees over networks on a grid of \gls{I}
and \gls{N} values in the region of uniform prior density. We fit logistic
growth curves to the proportion of infected individuals over time, and
calculated the first and second derivatives of these curves at the time of
transmission tree sampling. These derivatives give us an indication of the
growth rates of the epidemics at the time of sampling. As shown in
\cref{fig:derivs}, there are bands along which both derivatives are similar
which contain the values we tested. These bands span mostly higher values of
\gls{N} and \gls{I} than the true values. Therefore, if \gls{N} and \gls{I} are
free to vary (as is the case in kernel-assisted \gls{ABC}), and our hypothesis is true,
both parameters will tend to be overestimated due to being less identifiable
within their own band. However, when \gls{N} is fixed at 5000, the derivatives
vary substantially along the \gls{I}-axis, which explains why the grid search
estimates of \gls{I} were accurate and unbiased (\cref{fig:gridptI}).

\begin{figure}[ht]
  \centering
  \includegraphics{derivatives.pdf}
  \caption[
    First and second derivatives of epidemic growth curves at time of sampling
    for various values of $I$ and $N$.
  ]{
    First and second derivatives of epidemic growth curves at time of sampling
    for various values of \gls{I} and \gls{N}. Networks were simulated under
    the \gls{BA} model with \gls{alpha} = 1.0, \gls{m} = 2, and \gls{N} varied 
    along the values shown on the $x$-axis. Transmission trees were sampled at
    the time when \gls{I} nodes were infected ($y$-axis). Logistic growth
    curves were fit to epidemic trajectories derived from the transmission
    trees, and their first and second derivatives were calculated at the time
    of sampling. Contours show first derivatives, while colours indicate second
    derivatives. Values of \gls{I} and \gls{N} used in simulation experiments
    with kernel-assisted \gls{ABC} are indicated by diamonds.
  }
  \label{fig:derivs}
\end{figure}

The \gls{m} parameter, which controls the number of connections added to the
network per vertex, did not have a measurable impact on tree shape and was not
identifiable with kernel-ABC. The exception to this was the value \gls{m} = 1,
which produces networks without cycles whose associated trees were more easily
distinguished. However, all the analyses presented here did not take the
absolute size of the transmission trees into account, as the branch lengths
were rescaled by their mean. Because higher \gls{m} values imply higher edge
density, an epidemic should spread more quickly for higher \gls{m} than lower
\gls{m} with the same per-edge transmission probability.  Hence, considering
the absolute height of the trees may improve our method's ability to
reconstruct \gls{m}.

In addition to the tree height, many summary statistics have been developed to
capture particular details of tree shape. Two of these, Sackin's index and the
ratio of internal to terminal branch lengths, were correlated with every
\gls{BA} parameter. Classifiers based on Sackin's index and the \gls{nltt}
similarity measure performed well in some cases, though poorly in others.
\gls{ABC} is often applied using a vector of summary
statistics~\autocite{marin2012approximate, sunnaaker2013approximate}, rather
than a kernel-based similarity score as we have done here. Methods have been
developed to select an optimal combination of summary statistics for a given
inference task~\autocite{fearnhead2012constructing}. Hence, an avenue for
future improvement of our method may be the inclusion of additional summary
statistics to supplement the tree kernel. In addition, all four parameters were
more accurately classified when the number of tips in the transmission trees
was larger, underscoring the importance of adequate sampling for accurate
phylodynamic inference.

For the more identifiable parameters, the credible intervals attained from the
marginal \gls{ABC} target distributions were much narrower than those obtained
through grid search, while point estimates were of comparable accuracy. This
was likely due to the fact that \gls{SMC} employs importance sampling to
approximate the posterior distribution, while grid search simply calculates a
distance metric which may not have any resemblance to the posterior.
Admittedly, our method of finding credible intervals from kernel scores along
the grid, namely by normalizing the scores to resemble a probability
distribution, was somewhat ad hoc, which may also have played a role.
Regardless, this result indicates that there is benefit to applying the more
sophisticated method, even if values for some of the parameters are known
\textit{a priori}, and especially if credible intervals are desired on the
parameters of interest.

As noted by \textcite{lintusaari2016identifiability}, uniform priors on model
parameters may translate to highly informative priors on quantities of
interest. We observed a non-linear relationship between the preferential
attachment power $\alpha$ and the power law exponent $\gamma$
(\cref{fig:gamma}). Therefore, placing a uniform prior on $\alpha$ between 0
and 2 is equivalent to placing an informative prior that $\gamma$ is close to
2. Therefore, if we were primarily interested in $\gamma$ rather than
$\alpha$, a more sensible choice of prior might have a shape informed by
\cref{fig:gamma} and be bounded above by approximately $\alpha$ = 1.5. This
would uniformly bound $\gamma$ in the region $2 \leq \gamma \leq 4$ commonly
reported in the network literature~\autocite{liljeros2001web,
schneeberger2004scale, colgate1989risk, brown2011transmission}. We note however
that \textcite{jones2003assessment} estimated $\gamma$ values greater than
four for some datasets, in one case as high as 17, indicating that a wider
range of permitted $\gamma$ values may be warranted.

The combination of method, model, and priors we employed did not produce
perfect estimates of any of the parameters. The estimates of \gls{alpha} were
the most accurate, although the variance of the estimates was high and the
confidence intervals were wide for \gls{alpha} $< 1$
(\cref{tab:abchpd,fig:abcptm2,fig:abcptm3,fig:abcptm4}). The estimates of
\gls{N} and \gls{I} were both biased, and the estimates of \gls{m} were largely
uninformative. Despite these issues, a major result of our our investigation is
that some contact network parameters have a measurable impact on tree shape
which can be used to perform statistical inference. Further refinements to
\software{netabc}, as well as the use of more sophisticated network models, may
improve the accuracy and precision of these estimates.

\subsection{Application to real world HIV data}

Our investigation of published HIV datasets indicated heterogeneity in the
contact network structures underlying several distinct local epidemics. When
interpreting these results, we caution that the BA model is quite simple and
most likely misspecified for these data. In particular, the average degree of a
node in the network is equal to $2m$, and therefore is constrained to be a
multiple of 2. Furthermore, we considered the case $m = 1$, where the network
has no cycles, to be implausible and therefore assigned it zero prior
probability in one set of analyses. This forced the average degree to be at
least four, which may be unrealistically high for sexual networks. The fact
that the estimated values of $\alpha$ differed substantially for three datasets
depending on whether or not $m = 1$ was allowed by the prior is further evidence
of this potential misspecification. However, we note that for two of the
datasets, the estimated values of $\alpha$ did not change much between priors,
and the estimates of $I$ were robust to the choice of prior for all datasets
studied. More sophisticated models, for example models incorporating
heterogeneity in node behaviour, are likely to provide a better fit to these
data.

{\color{red}\sout{With respect to the preferential attachment power $\alpha$,
the six datasets analysed fell into two categories} (\cref{fig:abchpd}).
\sout{First, we estimated a preferential attachment power close to 1,
indicating linear preferential attachment, for the BC data and the outbreaks
studied by} \textcite{niculescu2015recent} and \textcite{wang2015targeting}.
\sout{These values were robust to specifying different priors for $m$. All three
datasets were sampled from populations in which we would expect a high degree
of epidemiological relatedness:} \textcite{niculescu2015recent} \sout{studied a
recent outbreak among Romanian \gls{IDU},} \citeauthor{wang2015targeting}
\sout{sampled acutely infected MSM in Beijing, China, and the BC data
constituted a phylogenetic \gls{IDU} cluster. These are all contexts in which
we would expect some of the assumptions of the BA model, such as a connected
network, relatively high mean degree, and preferential attachment dynamics, to
hold.}

\sout{The remaining three datasets} (\textcite{cuevas2009hiv,
novitsky2014impact, li2015hiv}) \sout{had estimated values of $\alpha$ below
0.5 when $m = 1$ was included in the prior, but these were not robust to
changing the prior to exclude $m = 1$. For the} \citeauthor{cuevas2009hiv} 
\sout{data, model misspecification is likely partially responsible. While the
authors found that a large proportion of the samples were epidemiologically
linked, these were mainly in small local clusters rather than the single large
component postulated by the BA model. In addition, the mixed risk groups in the
dataset would be unlikely to significantly interact, further weakening any
global preferential attachment dynamics. The dataset studied by}
\textcite{novitsky2014impact} \sout{originated from a densely sampled
population where the predominant risk factor was believed to be heterosexual
exposure. Although the MAP estimate of $\alpha$ was almost unchanged when the
value $m = 1$ was excluded from the prior, the confidence interval shrank
substantially.}}



For all datasets we examined, the posterior mean estimates for \gls{alpha} were
sub-linear, ranging from 
    0.27
to 
    0.73.
The sub-linearity is consistent with the results of
\textcite{de2007preferential}, who developed a statistical inference method to
estimate the parameters of a more sophisticated \acrlong{PA} model
incorporating heterogeneous node behaviour. They found \gls{alpha} values
ranging from 0.26 to 0.62, depending on the gender and time period considered. 
Our estimates of \gls{alpha} for the \textcite{niculescu2015recent} was above
this range under both priors, as were the estimates for the
\textcite{wang2015targeting} data and the BC data when \gls{m} = 1 was
disallowed by the prior. The dataset investigated by
\textcite{de2007preferential} was derived from a survey of a random sample of
the Norwegian population, whereas our investigation focused on datasets from
known phylogenetic or geographic clusters of \gls{HIV} infected persons. It is
therefore unsurprising that we detected stronger \acrlong{PA} dynamics in some
cases. For instance, random sampling is much less likely to discover the
high-degree nodes characterizing the tail of the degree distribution, simply
because those individuals are rare in the general population. In addition, it
is plausible that \gls{HIV}-positive individuals are more likely to be highly
connected in their sexual networks, as the odds of acquiring \gls{HIV} increase
with the number of unprotected sexual contacts.

Both \textcite{de2007preferential} and \textcite{novitsky2014impact} studied
populations whose primary risk factor for \gls{HIV} infection was heterosexual
contact. \citeauthor{de2007preferential} explicitly excluded reported
homosexual contacts; \citeauthor{novitsky2014impact} did not, but noted that
heterosexual contact is the primary mode of transmission in Botswana where the
study was done. In the first of the two papers describing the Botswana
study~\autocite{novitsky2013phylogenetic}, the authors noted that their sample
was gender-biased, being composed of approximately 75\% women. Our estimate of
\gls{alpha} for these data was 
    0.55
        or 
    0.53,
depending on the prior on \gls{m}; \citeauthor{de2007preferential} estimated
0.54, 0.57, and 0.29 for 3-year, 5-year, and lifetime partnership networks
respectively for the female portion of their sample.

For both choices of prior on \gls{m}, the datasets derived from \gls{IDU}
populations had a higher estimated \acrlong{PA} power than the other datasets
(\cref{fig:abchpd,fig:abchpdm2}). This finding is in line with
\textcite{dombrowski2013topological}, who reanalyzed a network of \glspl{IDU}
in Brooklyn, USA, collected between 1991 and
1993~\autocite{friedman2006social}. They found that the the \gls{IDU} network 
resembled a \gls{BA} network much more closely than other social and sexual 
networks, and offered sociological explanations for the apparent \acrlong{PA}
dynamics in this population. Importantly, from a public health perspective,
the authors asserted that the removal of \emph{random} individuals from
\gls{IDU} networks may have the paradoxical effect of increasing the network's
epidemic susceptibility. When low-degree nodes are removed, as would occur
during a police crackdown, their network neighbours may turn to well-known
community members for advice or supplies, thus increasing the connectivity of
these high-degree nodes.

One somewhat surprising result was the difference between parameter estimates 
for the \textcite{li2015hiv} and \textcite{wang2015targeting} datasets. Both
groups studied cohorts of acutely infected \gls{MSM} in major Chinese cities
(Shanghai and Beijing respectively); yet, the \citeauthor{li2015hiv} data was 
estimated to have a lower \acrlong{PA} power and larger infected population
than the \textcite{wang2015targeting} data\ldots

% for discussion
In order to compare our results to existing literature on networks and
distributions of partner counts, we have reported estimated values for the
power law exponent \gls{gamma} of the real data sets we evaluated. However, the
posterior means for \gls{alpha} for all six datasets were less than one; the
degree distributions in this parameter range are stretched exponential, not
power law~\autocite{krapivsky2000connectivity}. As we show in
\cref{fig:powerlaw}, the power law fit does capture the slope of the degree
distribution fairly well, but the results should still be interpreted
cautiously. \textcite{krapivsky2000connectivity} showed that the power law
distribution can be maintained, with \gls{gamma} tuned to any desired value, by
a straightforward modification of the \gls{BA} model. The authors define the
``connection kernel'' $A_k$ the probability of a new connection to a node of
degree $k$, up to a normalizing constant. In the \gls{BA} model as we have
presented it here, $A_k = k^{\alpha} + 1$. Taking $A_k$ as any asymptotically
linear function will result in a power law distribution, with the exponent
\gls{gamma} determined by the properties of $A_k$. Implementing such a model
would be straightforward and seems a natural next step toward improving the
realism of the \gls{BA} model.

\textcite{rothenberg2007large} estimated the power law exponents for 15 major
network studies of sexual and injection drug use networks carried out between
1981 and 2000. The estimated \gls{gamma} values for ``all contacts'' (that is,
including both interviewed and non-interviewed individuals) were 
between 2.08 and 2.87 for all but one of the networks\ldots

%For both priors, the estimated prevalence was extremely high, in fact higher
%than the estimated HIV prevalence in the sampled region. The authors indicated
%that the source of the samples was a town in close proximity to the country's
%capital city, and suggested that there may have been a high degree of migration
%and partner interchange between the two locations. It is possible that the
%contact network underlying the subtree we investigated includes a much larger
%group based in the capital city, which would explain the high estimate of $I$.
%There is no clear explanation for the discrepancy between the two priors for
%the \textcite{li2015hiv} data, as the subset we analyzed formed a phylogenetic
%cluster and therefore was a good candidate for the BA model. However, nearly
%all the posterior density was assigned to $m = 1$ when this value was allowed,
%indicating that the network was more likely to have an acyclic tree structure.

{\color{blue}\uline{
The estimates of the prevalence \gls{I} were largely robust to the choice of
prior on \gls{m}} (\cref{fig:abchpd,fig:abchpdm2}).
\textcite{niculescu2015recent} \uline{reported that 494 new HIV infections were
diagnosed among the studied population (\glspl{IDU} in Bucharest) during the
study period. Our estimate of the prevalence was higher (X), although the 95\%
\gls{HPD} interval did contain the value 494. In addition, it is not
unreasonable that a substantial fraction of the \gls{HIV}-positive individuals
in an \gls{IDU} population could be undiagnosed, given that these populations
are often marginalized.}

\uline{The individuals sampled by} \textcite{wang2015targeting} \uline{were
recruited by following a prospective cohort of 2000 \gls{MSM} in Beijing,
China. Of the 2000, 179 new infections were identified during the study period.
This number is much lower than the estimated prevalence of X obtained with
\software{netabc}, however the authors did not claim to have sampled the entire
sexual network. In a nationwide survey,} \textcite{wu2013hiv} \uline{estimated
that there were 24,198 \gls{MSM} living in Beijing, of whom 5.7\%, or 1379,
were \gls{HIV}-infected, which is within the 95\% confidence interval of our
estimate of \gls{I}.}

\textcite{li2015hiv} \uline{studied an \gls{MSM} population in Shanghai, China.
Despite the apparent similarity to the context of the
}\textcite{wang2015targeting} \uline{study, the estimated \gls{HIV} prevalence
of the } \citeauthor{li2015hiv} \uline{data was higher (X). The aforementioned
survey~\autocite{wu2013hiv} estimated that there were 14,511 \gls{MSM} living
in Shanghai, with an \gls{HIV} prevalence of 6.8\% or approximately 987
individuals. This was within the 95\% confidence interval of the estimate
obtained with \software{netabc}. It is worth noting that} \citeauthor{li2015hiv}
\uline{claimed that there were 80,000 \gls{MSM} living in Shanghai, which is a
significantly different estimate than that obtained by} \textcite{wu2013hiv}.

\textcite{cuevas2009hiv} \uline{reported that 620 newly infected and 1500
chronically infected patients had been sampled during the study period, for a
total of 2120. This is substantially greater than the point estimate of X we
obtained with \gls{ABC}, even without considering the likely possibility that
there were more infected than sampled individuals. Again, this error may in
part be due to model misspecification, which highlights the importance of
selecting a model appropriate to the dataset at hand.} }

Our use of the \gls{BA} model makes several simplifying assumptions. First, we
assume homogeneity across the network with respect to node behaviour and
transmission risk. In reality, the attraction to high-degree nodes seems likely
to vary among individuals, as does their risk of transmitting or contracting
the virus. We have also assumed that all transmission risks are symmetric,
which is clearly false for all known modes of \gls{HIV} transmission, and that
infected individuals never recover but remain infectious indefinitely. These
assumptions were made for the purpose of keeping the model as simple as
possible, since this is the very first attempt to fit a contact network model
in a phylodynamic context. However, the Gillespie simulation algorithm built
into \software{netabc} can handle arbitrary transmission and removal rates
which need not be homogeneous across the network. Moreover, it is possible to
use \gls{ABC} to fit a model which relaxes some or all of these assumptions,
which may be a fruitful avenue for future investigation.  Despite the possible
misspecification, our estimates of the power law exponent $\gamma$ were within
the range of values reported in the literature (\cref{tab:gamma}).
