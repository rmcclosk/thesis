\subsection{\software{Netabc}: uses, limitations, and possible extensions}

The method behind \software{netabc} is model-agnostic, meaning that it can be
used to infer parameters of any network model, as long as it allows simulated
networks can be easily generated. We have included generators for the \gls{BA}
model discussed here, as well as the \gls{ER} and \gls{WS} network models.
Instructions for adding additional models are available in the project's online
documentation. We have made \software{netabc} publicly available at
\url{github.com/rmcclosk/netabc} under a permissive open source license, to
encourage other researchers to apply and extend our method.

In retrospect, there are a few algorithmic improvements which could have been
made to \software{netabc}. Firstly, our implementation uses simple multinomial
sampling for the resampling step, as well as for generating the final sample
from the target distribution.  Several other sampling strategies have been
developed~\autocite{douc2005comparison}, and it is possible that the use of a
more sophisticated scheme might increase the algorithm's accuracy. Secondly,
the model parameters \gls{I}, \gls{N}, and \gls{m} are all integer-valued, but
we treat them as continuous-valued when performing \gls{SMC}. This is not
inherently wrong, in that it simply assigns equal probability density to the
interval $[i, i+1)$ for all integers $i$. However, computational effort could
be wasted if particles are moved around within these intervals. To improve
efficiency, alternative proposals could be used for discrete-valued particles.

% spatial networks, see section 5.4 in keeling2005networks

The two-step process of simulating a contact network and subsequently allowing
an epidemic to spread over that network carries with it the assumption that the
contact network is static over the duration of the epidemic. Clearly this
assumption is invalid, as people make and break partnerships on a regular
basis. Our work has not addressed this assumption, primarily due to our desire
to avoid the additional complexity required to address the dynamic nature of
networks. This simplifying assumption is made by most studies using contact
network models in an epidemiological context~\autocite{welch2011statistical,
bansal2007individual}. However, in principle, kernel-\gls{ABC} could be
adapted to dynamic contact networks by using a method such as that developed by
\textcite{robinson2012dynamics} to simulate a dynamic contact network, while
concurrently simulating the spread of an epidemic.

An alternative approach is the deterministic framework outlined by
\textcite{morris1993epidemiology}, who proposes to apply the standard
compartmental modelling framework to contact networks by assigning each
individual their own compartment. Thus, each individual is associated with a
single \gls{ODE}, with the entire \gls{ODE} system parameterized by the
adjacency matrix of the contact network. \citeauthor{morris1993epidemiology}
proposes to use log-linear models to parameterize the matrix. This framework is
highly expressive, and allows straightforward incorporation of time-dependent
dynamics. However, simulating a transmission tree would require the numerical
solution of a very large system of \glspl{ODE}. Given the large number of
simulations required for kernel-\gls{ABC}, it is not clear if this method would
be computationally feasible in this context.

It is important to note that \software{netabc} takes a transmission tree as
input, rather than a viral phylogeny. Thus, we have left the estimation of a
transmission tree up to the user. There were two reasons for this choice. First
and foremost, we wished again to avoid extra complexity and keep the number of
estimated parameters small. In theory, it is possible to incorporate the
process by which a viral phylogeny is generated along with a transmission tree
into our method, for example by simulating within-host dynamics. Although this
may be an avenue for future extension, we felt that it would obscure the
primary purpose of this work, which is to study contact network parameters.
Second, there are a number of different methods available for inferring
transmission trees~\autocite{didelot2014bayesian, ypma2012unravelling,
jombart2011reconstructing, cottam2008integrating, poon2015phylodynamic}, some
of which incorporate geographic and/or epidemiological data not accommodated by
our method. We therefore felt it would be best to allow researchers to use
their own preferred tree building method.


\subsection{Analysis of \acrlong{BA} model}

The preferential attachment power \gls{alpha} had a very strong influence on
tree shape in the range of values we considered. Although the tree kernel was
the most effective classifier for \gls{alpha}, a tree balance statistic
performed nearly as well. This result was intuitive: high \gls{alpha} values
produce networks with few well-connected superspreader nodes which are involved
in a large number of transmissions, resulting in a highly unbalanced
ladder-like tree structure. The \gls{I} parameter, representing the prevalence
at the time of sampling, was also generally estimable. The dynamics of the
\gls{SI} model, and the coalescent process, offer a potential explanation for
this result. In the initial phase of the epidemic, when \gls{I} is small, each
new transmission results in potentially many new discordant edges, thus
decreasing the waiting time until the next transmission. Hence, there is an
early exponential growth phase, producing many short branches near the root of
the tree. As the epidemic gets closer to saturating the network, the number of
discordant edges decays, causing longer waiting times. The distribution of
coalescent times in the tree should therefore be informative about \gls{I}.
This information is captured by the tree kernel, and also by the \gls{nltt}
statistic, which both performed quite will in classifying \gls{I}
(\cref{fig:rsquared}).

The number of nodes in the network, \gls{N}, exhibited the most variation in
terms of being estimable. There was almost no difference between trees
simulated under different \gls{N} values when the number of infected nodes
\gls{I} was very small. There is an intuitive explanation for this result,
namely that adding additional nodes does not change the edge density or overall
shape of a \gls{BA} network. This can be illustrated by imagining that we add a
small number of nodes to a network after the epidemic simulation has already
been completed. It is possible that none of these new nodes attains a
connection to any infected node. Thus, running the simulation again on the new,
larger network could produce the exact same transmission tree as before. On the
other hand, when \gls{I} is large, the coalescent dynamics discussed above for
\gls{I} also apply, as evidenced by the relative accuracy of the \gls{nltt}.
The \gls{m} parameter, which controls the number of connections added to the
network per vertex, did not have a measurable impact on tree shape and was not
estimable with kernel-ABC. The exception to this was the value \gls{m} = 1,
which produces networks without cycles whose associated trees were more easily
distinguished. However, all the analyses presented here did not take the
absolute size of the transmission trees into account, as the branch lengths
were rescaled by their mean. Because higher \gls{m} values imply higher edge
density, an epidemic should spread more quickly for higher \gls{m} than lower
\gls{m} with the same per-edge transmission probability.  Hence, considering
the absolute height of the trees may improve our method's ability to
reconstruct \gls{m}.

In addition to the tree height, many summary statistics have been developed to
capture particular details of tree shape. Two of these, Sackin's index and the
ratio of internal to terminal branch lengths, were correlated with every
\gls{BA} parameter. Classifiers based on Sackin's index and the \gls{nltt}
similarity measure performed well in some cases, though poorly in others.
\gls{ABC} is often applied using a vector of summary statistics, rather than a
kernel-based similarity score as we have done here, and methods have been
developed to select an optimal combination of summary statistics for a given
inference task~\autocite{fearnhead2012constructing}. Hence, an avenue for
future improvement of our method may be the inclusion of additional summary
statistics to supplement the tree kernel. In addition, all four parameters were
more accurately classified when the number of tips in the transmission trees
was larger, underscoring the importance of adequate sampling for accurate
phylodynamic inference.

For the more estimable parameters, the credible intervals attained from the
marginal \gls{ABC} target distributions were much narrower than those obtained
through grid search, while point estimates were of comparable accuracy. This
was likely due to the fact that \gls{SMC} employs importance sampling to
approximate the posterior distribution, while grid search simply calculates a
distance metric which may not have any resemblance to the posterior.
Admittedly, our method of finding credible intervals from kernel scores along
the grid, namely by normalizing the scores to resemble a probability
distribution, was somewhat ad hoc, which may also have played a role.
Regardless, this result indicates that there is benefit to applying the more
sophisticated method, even if values for some of the parameters are known
\textit{a priori}, and especially if credible intervals are desired on the
parameters of interest.

\subsection{Application to HIV data}

The analysis of \gls{HIV} datasets indicated heterogeneity in the contact
network structures underlying several distinct local epidemics. The five
datasets analysed fell into three categories (Figure~\ref{fig:abchpd}). First,
we estimated a preferential attachment power between 0.5 and 1 for the
epidemics studied by \textcite{cuevas2009hiv} and \textcite{li2015hiv}, with
credible intervals occupying nearly the entire region from 0 to 1.
\citeauthor{cuevas2009hiv} studied a group of newly diagnosed individuals in
the Basque Country, Spain. Although the individuals were of mixed risk groups,
and therefore unlikely to comprise a single contact network, a high proportion
of them (47\%) grouped into local clusters based on genetic distance. The low
estimated attachment power for these data is consistent with the sampled
sequences comprising many distinct sub-networks rather than a single connected
network. \citeauthor{li2015hiv} sampled a large number of acutely infected MSM
in Shanghai, China, in which we identified a large cluster from the phylogeny
using a patristic distance cutoff~\autocite{poon2015impact}. The low attachment
power estimated for this dataset was surprising given the high phylogenetic
relatedness of the sequences. It is possible that the number and diversity of
circulating recombinant forms in the data introduced errors into our estimated
viral phylogeny.

For the outbreaks studied by \textcite{niculescu2015recent} and
\textcite{wang2015targeting}, the estimated $\alpha$ was close to one, with a
narrower credible interval than for the other studies.
\citeauthor{niculescu2015recent} studied a recent outbreak among Romanian
injection drug users (IDU), while \citeauthor{wang2015targeting} sampled
acutely infected MSM in Beijing, China. Both studies discovered a high degree
of phylogenetic relatedness owing to the recent infection times and homogeneous
risk groups of the studied populations. The estimated number of infections for
these datasets were also quite low, although the HPD interval for
\citeauthor{wang2015targeting} was much wider than that for
\citeauthor{niculescu2015recent}.

The final studied dataset was an outlier in terms of estimated parameters.
\textcite{novitsky2013phylogenetic} sampled approximately 44\% of the
HIV-infected individuals in the northern area of Mochudi, Botswana. Additional
sampling in a later study~\autocite{novitsky2014impact} brought the genotyping
coverage up to 70\%. Even with such a high sampling coverage, we did not detect
any large clusters using patristic distance, and therefore chose to analyze a
subtree instead. Estimates of $\alpha$ and $N$ both had very wide HPD intervals
and were markedly different from the other datasets. The estimated number of
infected nodes was also extremely high, much higher than the estimated HIV
prevalence of the town. Several factors may have contributed to these results.
First, the authors note that the their sample was 75\% female.  In a primarily
heterosexual risk environment, removal of a disproportionate number of males
from the network could obfuscate the true network structure, for example if the
majority of highly connected nodes were of one gender.  Second, the town in
question was in close proximity to the country's capital, and the authors
indicated that a high amount of migration takes place between the two
locations. This suggests that the contact network may include a much larger
group based in the capital city, which would explain the high estimate of $I$.

Our use of the \gls{BA} model makes several simplifying assumptions. First, we
assume homogeneity across the network with respect to node behaviour and
transmission risk. In reality, the attraction to high-degree nodes seems likely
to vary among individuals, as does their risk of transmitting or contracting
the virus. We have also assumed that all transmission risks are symmetric,
which is clearly false for all known modes of \gls{HIV} transmission, and that
infected individuals never recover but remain infectious indefinitely. These
assumptions were made for the purpose of keeping the model as simple as
possible, since this is the very first attempt to fit a contact network model
in a phylodynamic context. However, the Gillespie simulation algorithm built
into \software{netabc} can handle arbitrary transmission and removal rates
which need not be homogeneous across the network. Moreover, it is possible to
use kernel-\gls{ABC} to fit a model which relaxes some or all of these
assumptions, which may be a fruitful avenue for future investigation.
