

{\color{blue}\uline{
The chapter is organized as follows. One section is devoted to each of our
three research aims: the development of a method for inference of contact
network parameters from transmission trees} (\cref{sec:netabc}), \uline{the
investigation of the \acrlong{BA} network model using simulated data
}(\cref{sec:ba}), \uline{ and the application of the method to real-world
\gls{HIV} epidemics }(\cref{sec:hiv}). \uline{Each of these sections contains a
description of the methods used and, in the case of the latter two objectives,
the results of the experiments. The fourth and final section of the chapter
}(\cref{sec:disc}) \uline{contains a unified discussion of all three
objectives. }}

\section{\software{Netabc}: a computer program for estimation of contact
network parameters with kernel-assisted ABC}
\label{sec:netabc}

\software{Netabc} is a computer program to perform statistical inference of
contact network parameters from an estimated transmission tree using
\gls{ABC}. {\color{blue}\uline{As discussed in }\cref{sec:obj}, \uline{the principal
statistical algorithm used by \software{netabc} is adaptive
\gls{ABC}-\gls{SMC}~\autocite{del2012adaptive}. In addition, there are two
supplementary components which are specific to the domain of phylogenetics and
contact networks: Gillespie simulation~\autocite{gillespie1976general}, to
simulate transmission trees on contact networks; and the tree
kernel~\autocite{poon2013mapping}, which is used as the distance function in
\gls{ABC} to compare transmission trees~\autocite{poon2015phylodynamic}} (see
\cref{sec:abc}).} We give a high-level overview of the program here, before
describing these components in detail. \software{Netabc} takes as input an
estimated transmission tree, which can be derived from a viral phylogeny by
rooting and time-scaling as described in \cref{subsec:phylodynamics} or
estimated by other methods~\autocite{cottam2008integrating,
jombart2011reconstructing, ypma2012unravelling, morelli2012bayesian,
didelot2014bayesian, hall2015epidemic}. We variously refer to this estimated
transmission tree as the observed tree, input tree, or true tree.

As described in \cref{sec:smc}, \software{netabc} keeps track of a population
of particles $x^{(k)}$, each of which contains particular parameter values
$\theta^{(k)}$ for the {\color{blue}\uline{contact network}} model we are
trying to fit {\color{blue}\uline{to the input tree}}. A small number of
contact networks $z^{(k)}$ are generated under the model for each particle, in
accordance with that particle's parameters. An epidemic is simulated over each
of these networks using Gillespie simulation, and by keeping track of its
progress, a transmission tree is obtained. Thus, each particle becomes
associated with several simulated transmission trees. These trees are compared
to the input tree using the tree kernel. Particles are weighted according to
the similarity of their associated simulated trees with the true tree, with
more similar trees receiving higher weights. The particles are iteratively
perturbed to explore the parameter space, and particles with simulated trees
too distant from the true tree are periodically dropped and resampled. Once a
convergence criterion is attained, the final set of particles is used as a
Monte Carlo approximation to the target distribution of \gls{ABC}, which is
assumed to resemble the posterior distribution on model parameters (see
\cref{sec:abc}). A graphical schematic of this algorithm is given in
\cref{fig:abcsmc}.

\begin{figure}
    \includegraphics{abc-smc.pdf}
    \caption[Graphical schematic of the ABC-SMC algorithm implemented in \software{netabc}.]{
      Graphical schematic of the \gls{ABC}-\gls{SMC} algorithm implemented in
      \software{netabc}. Particles are initially drawn from their prior
      distributions, making the initial population a Monte Carlo approximation
      to the prior. At each iteration, particles are perturbed, and a distance
      threshold around the true tree contracts. Particles are rejected, and
      eventually resampled, when all their associated simulated trees lie
      outside the threshold. As the algorithm progresses, the population
      smoothly approaches a Monte Carlo approximation of the \gls{ABC} target
      distribution, which is assumed to resemble the posterior.
    }
    \label{fig:abcsmc}
\end{figure}

\software{Netabc} is written in the \software{C} programming language. The
\software{igraph} library~\autocite{csardi2006igraph} is used to generate and
store contact networks and phylogenies. Judy arrays~\autocite{baskins2004judy}
are used for hash tables and dynamic programming matrices. The
\gls{GSL}~\autocite{gough2009gnu} is used to generate random draws from
probability distributions, and to perform the bisection step in the adaptive
\gls{ABC}-\gls{SMC} algorithm. Parallelization is implemented with \gls{POSIX}
threads~\autocite{barney2009posix}. In addition to the \software{netabc} binary
to perform kernel-assisted \gls{ABC}, we provide three additional stand-alone
utilities: \software{treekernel}, to calculate the tree kernel;
\software{nettree}, to simulate a transmission tree over a contact network; and
\software{treestat}, to compute various summary statistics of phylogenies. The
programs are freely available at \url{https://github.com/rmcclosk/netabc}.

To check that our implementation of Gillespie simulation was correct, we
reproduced Figure 1A of \textcite{leventhal2012inferring} (our
\cref{fig:leventhal}), which plots the imbalance of transmission trees
simulated over four network models at various levels of pathogen
transmissibility. Our implementation of adaptive \gls{ABC}-\gls{SMC} was tested
by applying it to the same mixture of Gaussians used by
\textcite{del2012adaptive} to demonstrate their method (originally used
by~\textcite{sisson2007sequential}). We were able to obtain a close
approximation to the function (see \cref{fig:smctest}), and attained the
stopping condition used by the authors in a comparable number of steps. To
check that the algorithm would converge to a bimodal distribution, we also
applied it to a mixture of two Gaussians with means $\pm$4 and variances 1. The
algorithm was able to recover both peaks (\cref{fig:smctest2}).

\subsection{Simulation of transmission trees over contact networks}
\label{subsec:nettree}

The simulation of epidemics, and the corresponding transmission trees, over
contact networks is performed in \software{netabc} using the Gillespie
simulation algorithm~\autocite{gillespie1976general}. This method has been
independently implemented and applied by several
authors~\autocite[\textit{e.g.}][]{o2011contact, robinson2013dynamics,
leventhal2012inferring, groendyke2011bayesian, villandre2016assessment}.
\textcite{groendyke2011bayesian} published their implementation as an
\software{R} package, but since the \gls{SMC} algorithm is quite
computationally intensive, we chose to implement our own version in
\software{C} {\color{blue}\uline{as part of \software{netabc}}}.

Let $G = (V, E)$ be a directed contact network. We assume the individual nodes
and edges of $G$ follow the dynamics of the \gls{SIR}
model~\autocite{kermack1927contribution}. Each directed edge $e = (u, v)$ in
the network is associated with a transmission rate $\beta_e$, which indicates
that, once $u$ becomes infected, the waiting time until $u$ infects $v$ is
distributed as $\Exponential(\beta_e)$. Note that $v$ may become infected
before this time has elapsed, if $v$ has other incoming edges. $v$ also has a
removal rate $\gamma_v$, so that the waiting time until removal of $v$ from the
population is $\Exponential(\gamma_v)$. Removal may correspond to death or
recovery with immunity, or a combination of both, but in our implementation
recovered nodes never re-enter the susceptible population. We define a
\defn{discordant edge} as an edge $(u, v)$ where $u$ is infected and $v$ has
never been infected.

To describe the algorithm, we introduce some notation and variables. Let
$\inc(v)$ be the set of incoming edges to $v$, and $\out(v)$ be the set of
outgoing edges from $v$. Let $I$ be the set of infected nodes in the network,
$R$ be the set of removed nodes, and $S$ be the remaining susceptible nodes,
and $D$ be the set of discordant edges in the network. Let $\beta$ be the total
transmission rate over all discordant edges, and $\gamma$ be the total removal
rate of all infected nodes,
\[
  \beta = \sum_{e \in D} \beta_e, \quad
  \gamma = \sum_{v \in I} \gamma_v.
\]
The variables $S$, $I$, $R$, $D$, $\beta$, and $\gamma$ are all updated as the
simulation progresses. When a node $v$ becomes infected, it is deleted from $S$
and added to $I$. Any formerly discordant edges in $\inc(v)$ are deleted from
$D$, and edges in $\out(v)$ to nodes in $S$ are added to $D$. If $v$ is later
removed, it is deleted from $I$ and added to $R$, and any discordant edges in
$\out(v)$ are deleted from $D$. At the time of either infection or removal, the
variables $\beta$ and $\gamma$ are updated to reflect the changes in the
network. Since these updates are straightforward, we do not write them
explicitly in the algorithm.

\newcommand{\tip}{\mathit{tip}}

The Gillespie simulation algorithm is given as Algorithm~\ref{alg:nettree}. The
transmission tree $T$ is simulated along with the epidemic. We keep a map
called $\tip$, which maps infected nodes in $I$ to the tips of $T$. The
simulation continues until either there are no discordant edges left in the
network, or we reach a user-defined cutoff of time ($t_{\max}$) or number of
infections ($I_{\max}$). We use the notation $\Uniform(0, 1)$ to indicate a
number drawn from a uniform distribution on $(0, 1)$, and likewise for
$\Exponential(\lambda)$. The combined number of internal nodes and tips in $T$
is denoted $|T|$.

\begin{algorithm}
  \label{alg:nettree}
  \caption{Simulation of an epidemic and transmission tree over a contact network}
  \begin{algorithmic}
    \State infect a node $v$ at random, updating $S$, $I$, $D$, $\beta$ and $\gamma$
    \State $T \gets$ a single node with label $1$
    \State $\tip[v] \gets 1$
    \State $t \gets 0$
    \While{$D \neq \emptyset$ and $|I| + |R| < I_{\max}$ and $t < t_{\max}$}
      \State $s \gets \min(t_{\max} - t, \Exponential(\beta + \gamma))$
      \For{$v \in \tip$}
        \State{extend the branch length of $\tip[v]$ by $s$}
      \EndFor
      \State $t \gets t + s$
      \If{$t < t_{\max}$}
        \If{$\Uniform(0, \beta + \gamma) < \beta$}
          \State choose an edge $e = (u, v)$ from $D$ with probability $\beta_e / \beta$
                 and infect $v$
          \State $\tip[v] \gets |T|+1$
          \Comment{add new tips to tree and tip array}
          \State $\tip[u] \gets |T|+2$
          \Comment{corresponding to $u$ and $v$}
          \State add tips with labels $(|T|+1)$ and $(|T|+2)$ to $T$
          \State connect the new nodes to $\tip[v]$ in $T$, with branch lengths $0$
        \Else
          \State choose a node $v$ from $I$ with probability $\gamma_v / \gamma$
                 and remove $v$
          \State delete $v$ from $\tip$
        \EndIf
        \State update $S$, $I$, $R$, $D$, $\beta$, and $\gamma$
      \EndIf
    \EndWhile
  \end{algorithmic}
\end{algorithm}

\subsection{Phylogenetic kernel}

The tree kernel developed by \textcite{poon2013mapping} provides a
comprehensive similarity score between two phylogenetic trees, via the
dot-product of the two trees' feature vectors in the infinite-dimensional space
of all possible subset trees with branch lengths (see \cref{subsec:treeshape}).
The kernel was implemented using the fast algorithm developed by
\textcite{moschitti2006making}. First, the production rule of each node, which
is the total number of children and the number of leaf children, is recorded.
The nodes of both trees are ordered by production rule, and a list of pairs of
nodes sharing the same production rule is created. These are the nodes for
which the value of the tree kernel must be computed - all other pairs have a
value of zero. The pairs to be compared are then re-ordered so that the child
nodes are always evaluated before their parents. Due to its recursive
definition, ordering the pairs in this way allows the tree kernel to be
computed by dynamic programming. The complexity of this implementation is
$O(|T_1||T_2|)$, where $|T|$ counts the number of nodes in the tree $T$.

The tree kernel cannot be used directly as a distance measure for \gls{ABC},
since it is maximized, not minimized, when the two trees being compared are the
same. Therefore, we defined the distance between two trees as
\[
  \rho(T_1, T_2) = 1 - \frac{K(T_1, T_2)}{\sqrt{K(T_1, T_1) K(T_2, T_2)}},
\]
which is a number between 0 and 1 minimized when $T_1 = T_2$. This is similar
to the normalization used by \textcite{collins2002new, poon2013mapping}.

\subsection{Adaptive sequential Monte Carlo for Approximate Bayesian computation}
\label{subsec:adaptsmc}

We implemented the adaptive \gls{SMC} algorithm for \gls{ABC} developed by
\textcite{del2012adaptive}. This algorithm is similar to the reference
\gls{ABC}-\gls{SMC} algorithm described in \cref{subsec:abcalg}, except that
the sequence of tolerances $\varepsilon_i$ is automatically determined rather
than specified in advance. The tolerances are chosen such that the \gls{ESS} of
the particle population, which indicates the quality of the Monte Carlo
approximation (see \cref{subsec:sis}), decays at a controlled rate. A
sudden precipitous drop in \gls{ESS} would indicate that only a small number of
particles had non-zero importance weights, which would result in a very poor
Monte Carlo approximation to the target distribution. This situation is
referred to as the ``collapse'' of the approximation, and is mitigated by the
adaptive approach. A single parameter $\alpha$ (not to be confused with the
\gls{BA} model parameter) controls the decay rate, with $\varepsilon_i$ being
chosen to satisfy
\[
  \ESS(w_i) = \alpha \ESS(w_{i-1}).
\]
Here, $w_i$ is the vector of weights at the $i$th step. Note that, since $w_i$
depends on $\varepsilon_i$, this equation solves for the updated weights and
the updated tolerance simultaneously. As pointed out by
\textcite{del2012adaptive}, the equation has no analytic solution, but can be
solved numerically by bisection. The forward kernels $K_i$ are taken to be
\gls{MCMC} kernels with stationary distributions $\pi_{\varepsilon_i}$ and
proposal distributions
\[
  q_i(\theta, \theta') \prod_{k=1}^M \Pr(z_i^{(k)'} \mid \theta'),
\]
where $\theta$ is the vector of model parameters and $z_k$ are $M$ datasets
simulated according to $\theta'$. In our implementation, $q$ is either a
Gaussian proposal for continuous parameters, or a Poisson proposal for discrete
parameters. For the Poisson proposals, the number of steps to move the particle
is drawn from a Poisson distribution, and the direction in which to move the
particle is chosen uniformly at random. For both proposals, the variance was
set equal to twice the empirical variance of the particles,
following~\autocite{beaumont2009adaptive, del2012adaptive}. The backwards
kernels are
\[
  L_{i-1}(x', x) = \frac{\pi_n(x)K(x, x')}{\pi_n(x')}.
\]
When substituted into \cref{eq:smcwt}, the forward kernels $K(x, x')$ and
densities $\pi_n(x') = \pi_{\varepsilon_n}(x')$ cancel out, and we are left
with the weight update 
\begin{align*}
  w_i(x) 
    &\propto w_{i-1}(x) \frac{\pi_n(x \mid y)}{\pi_{i-1}(x \mid y)} \\
    &= w_{i-1}(x) \frac{\pi(x) \pi_i(y \mid x)}{\pi(x) \pi_{i-1}(y \mid x)} \\
    &= w_{i-1}(x) \frac{\sum_{k=i}^M \I_{A_{\varepsilon_i, y}}(z_k)}
            {\sum_{k=i}^M \I_{A_{\varepsilon_{i-1}, y}}(z_k)}.
\end{align*}
In other words, when the distance threshold $\varepsilon_{i-1}$ is contracted
to $\varepsilon_i$, the particles' weights are multiplied by the proportion of
simulated datasets which are still inside the new threshold. The algorithm may
be stopped when one of two termination conditions is reached. The user may
specify a final tolerance $\varepsilon$, or a final acceptance rate of the
\gls{MCMC} kernel. The latter condition stops the algorithm when the particles
are not moving around very much, implying little change in the estimated
target.

\section{Analysis of \acrlong{BA} model with synthetic data}
\label{sec:ba}

\subsection{Methods}
\glsreset{BA}

{\color{blue}\uline{Using synthetic data,}} we investigated four parameters
related to the \gls{BA} contact network model, denoted \gls{N}, \gls{m},
\gls{alpha}, \gls{I} (see \cref{subsec:pa}). The first three of these are
parameters of the model itself, while \gls{I} is related to the simulation of
transmission trees over the network. However, we will refer to all four as
\gls{BA} parameters. \gls{N} denotes the total number of nodes in the network,
or equivalently, susceptible individuals in the population. When a node is
added to the network, \gls{m} new undirected edges are added incident to it,
and are attached to existing nodes of degree $k$ with probability proportional
to $k^\alpha + 1$ (\cref{subsec:pa}). To simulate transmission trees over a
\gls{BA} network, we allowed an epidemic to spread until \gls{I} nodes were
infected, and sampled a transmission tree at that time. The \gls{alpha}
parameter is unitless, while \gls{m} has units of edges or connections per
vertex, and \gls{N} and \gls{I} both have units of nodes or individuals.

{\color{blue}\uline{Our investigation comprised three sets of computational
experiments aimed at determining which of these parameters could be estimated
from phylogenetic data, and with what degree of accuracy. First, we performed
an exploratory analysis using a tree kernel-based classifier to determine which
of the four \gls{BA} parameters might be identifiable from tree shapes when all
others were held fixed. This analysis was also used to validate our choice of
the tree kernel as a distance function for \gls{ABC}, by evaluating two further
classifiers based on other tree summary statistics from the literature. Second,
we used grid search to investigate the accuracy and precision of marginal
parameter estimates, again holding all but one parameters of the model fixed.
Finally, we applied the full \gls{ABC}-\gls{SMC} algorithm to jointly estimate
the four parameters of the \gls{BA} model simultaneously.}}

We assumed that all contacts had symmetric transmission risk, which was
implemented by replacing each undirected edge in the network with two directed
edges (one in each direction). Nodes in our networks followed simple \gls{SI}
dynamics, meaning that they became infected at a rate proportional to their
number of infected neighbours, and never recovered. We did not consider the
time scale of the transmission trees in these simulations, only their shape.
Therefore, the transmission rate along each edge in the network was set to 1,
the removal rate of each node was set to 0, and all transmission trees' branch
lengths were scaled by their mean. \textit{igraph} library's implementation of
the BA model~\autocite{csardi2006igraph} was used to generate the graphs. The
analyses were run on Westgrid (\url{https://www.westgrid.ca/}) and a local
computer cluster. With the exception of our own programs, all analyses were
done in \software{R}, and all packages listed below are \software{R} packages.
{\color{blue}\uline{Code to run all simulation-based experiments is freely
available at \url{https://github.com/rmcclosk/thesis}.}}

\subsubsection*{Classifiers for BA model parameters based on tree shape}
\label{subsec:kernel}

\glsreset{nltt}

{\color{blue}\uline{
Our first computational experiment was designed as an exploratory analysis of
the four \gls{BA} model parameters defined above: \gls{alpha}, \gls{I},
\gls{m}, and \gls{N}. The objective of this experiment was to determine whether
any of the four parameters might be identifiable from the shape of the
transmission tree. TODO }

\uline{In addition, a secondary objective of this section was to validate the
use of the tree kernel as a distance measure for \gls{ABC} in our context. As
discussed in} \cref{sec:abc}, \uline{the choice of distance function is
extremely important for the accuracy of the \gls{ABC} approximation to the
posterior. Therefore, we evaluated two additional tree statistics in the same
manner as we evaluated the tree kernel (that is, by constructing and testing a
classifier). First, we considered Sackin's index~\autocite{shao1990tree}, which
measures the degree of imbalance or asymmetry in a phylogeny} (see
\cref{subsec:treeshape}). \uline{Sackin's index is widely used for
characterizing phylogenies~\autocite{frost2013modelling} and has been
demonstrated to vary between transmission trees simulated under different
contact network types~\autocite{leventhal2012inferring}. Sackin's index does
not take branch lengths into account, considering only the tree's topology. The
other statistic we considered was the
\gls{nltt}~\autocite{janzen2015approximate}, which compares two trees based on
normalized distributions of their branching times} (see \cref{sec:treeshape}).
\uline{In contrast with Sackin's index, the \gls{nltt} does not explicity
consider the trees' topologies, but it does use their normalized branch
lengths. While the \gls{nltt} is a newly developed statistic not yet in
widespread use, the unnormalized \gls{ltt}~\autocite{nee1992tempo} was the
basis of seminal early work extracting epidemiological information from
phylodynamics~\autocite{holmes1995revealing}.}

\uline{We expect the tree kernel to classify the \gls{BA} parameters more
accurately than either Sackin's index or the \gls{nltt}, since the tree kernel
takes both topology and branch lengths into account. }}

This experiment involved a large number of variables which were varied
combinatorially. For ease of exposition, we will describe a single experiment
first, then enumerate the values of all variables for which the experiment was
repeated. The parameters of the tree kernel, $\lambda$ and $\sigma$
(\cref{subsec:treeshape}) will be referred to as \defn{meta-parameters} to
distinguish them from the parameters of the \gls{BA} model. 

The attachment power parameter \gls{alpha} was varied among three values: 0.5,
1.0, and 1.5. For each value, the \software{sample\_pa} function in the
\software{igraph} package was used to simulate 100 networks, with the other
parameters set to \gls{N} = 5000 and \gls{m} = 2. This step yielded a total of
300 networks. An epidemic was simulated on each network using our
\software{nettree} binary until \gls{I} = 1000 nodes were infected, at which
point 500 of them were sampled to form a transmission tree. A total of 300
transmission trees were thus obtained, comprised of 100 trees for each of the
three values of \gls{alpha}. The trees were ``ladderized'' so that the subtree
descending from the left child of each node was not smaller than that
descending from the right child. Summary statistics, such as Sackin's index and
the ratio of internal to terminal branch lengths, were computed for each
simulated tree using our \software{treestat} binary. The trees were visualized
using the \software{ape} package~\autocite{paradis2004ape}. Our
\software{treekernel} binary was used to calculate the value of the kernel for
each pair of trees, with the meta-parameters set to $\lambda = 0.3$ and $\sigma
= 4$. These values were stored in a symmetric 300 $\times$ 300 kernel matrix.
Similarly, we computed the \gls{nltt} statistic between each pair of trees
using our \software{treestat} binary, and stored them in a second $300 \times
300$ matrix.

To investigate the identifiability of \gls{alpha} from tree shape, we
constructed classifiers for \gls{alpha} based on the three tree shape
statistics discussed above. First, we used the \software{kernlab}
package~\autocite{zeileis2004kernlab} to create a \gls{kSVR} classifier using
the computed kernel matrix. Second, we used the \software{e1071}
package~\autocite{meyer2015e1071} to create an ordinary \gls{SVR} classifier
using the pairwise \gls{nltt} matrix. Finally, we performed an ordinary linear
regression of \gls{alpha} against Sackin's index. Each of these classifiers was
evaluated with 1000 two-fold cross-validations. We also performed a \gls{kPCA}
projection of the kernel matrix, and used it to visualize the separation of the
different \gls{alpha} values in the tree kernel's feature space. A schematic of
this experiment is presented in \cref{fig:kernelexpt}.

Similar experiments were performed with the values shown in
\cref{tab:kernelexpt}. The other three \gls{BA} parameters, namely \gls{N},
\gls{m}, and \gls{I}, were each varied while holding the others fixed. The
experiments for \gls{alpha}, \gls{m}, and \gls{N} were repeated with three
different values of \gls{I}. All experiments were repeated with trees having
three different numbers of tips. Kernel matrices were computed for all pairs of
the meta-parameters \gls{lambda} = \sett{0.2, 0.3, 0.4} and \gls{sigma} =
\sett{\nicefrac18, \nicefrac14, \nicefrac12, 1, 2, 4, 8}.

\begin{landscape}
\begin{table}[ht]
  \centering
  \documentclass{article}

\usepackage{tikz}
\usepackage{array}
\usepackage{nicefrac}
\usepackage{fullpage}
\usepackage{mathptmx}
\usepackage[normalem]{ulem}

\usetikzlibrary{positioning}
\usetikzlibrary{arrows}
\usetikzlibrary{calc}

\begin{document}
\immediate\write18{./kernel-expt-extra.R}

\pagestyle{empty}

\newcommand{\boxw}{7cm}
\newcommand{\boxh}{2cm}

\begin{tikzpicture}
  \tikzstyle{box}=[minimum width=\boxw, minimum height=\boxh, draw]
  \tikzstyle{next}=[->, >=stealth, thick]

  \node (rect1) [box] { };
  \node at (rect1.north) [anchor=north] {\uline{3 parameter values}};
  \node (b) [above=0.5cm of rect1.south, blue] {\LARGE $\theta_1$};
  \node (r) [left=1.5cm of b, red] {\LARGE $\theta_2$};
  \node (g) [right=1.5cm of b, green] {\LARGE $\theta_3$};

  \node (rect2) [below=of rect1, box] { };
  \node at (rect2.north) [anchor=north] {\uline{300 networks}};
  \node (b) [above=4pt of rect2.south] {
    \begin{tabular}{m{0.5cm} m{1cm} m{0.5cm} m{1cm} m{0.5cm} m{1cm}}
      \includegraphics[width=1cm]{tinynet1} & $\times$ 100 &
      \includegraphics[width=1cm]{tinynet2} & $\times$ 100 &
      \includegraphics[width=1cm]{tinynet3} & $\times$ 100
    \end{tabular}
  };

  \node (rect3) [below=of rect2, box] { };
  \node at (rect3.north) [anchor=north] {\uline{300 epidemics}};
  \node (b) [above=4pt of rect3.south] {
    \begin{tabular}{m{0.5cm} m{1cm} m{0.5cm} m{1cm} m{0.5cm} m{1cm}}
      \includegraphics[width=1cm]{tinyepi1} & $\times$ 100 &
      \includegraphics[width=1cm]{tinyepi2} & $\times$ 100 &
      \includegraphics[width=1cm]{tinyepi3} & $\times$ 100
    \end{tabular}
  };

  \node (rect4) [below=of rect3, box] { };
  \node at (rect4.north) [anchor=north] {\uline{300 transmission trees}};
  \node (b) [above=4pt of rect4.south] {
    \begin{tabular}{m{0.5cm} m{1cm} m{0.5cm} m{1cm} m{0.5cm} m{1cm}}
      \includegraphics[width=1cm]{tinytree1} & $\times$ 100 &
      \includegraphics[width=1cm]{tinytree2} & $\times$ 100 &
      \includegraphics[width=1cm]{tinytree3} & $\times$ 100
    \end{tabular}
  };

  \node (rect5) [right=of rect2, box] { };
  \node at (rect5.north) [anchor=north, text width=\boxw, align=center] {
    \uline{3 tree shape statistics} \\ \vspace{4pt}
    300 $\times$ 300 kernel matrix \\
    300 $\times$ 300 nLTT matrix \\
    300 Sackin's index values
  };

  \node (rect6) [right=of rect1, box] { };
  \node at (rect6.north) [anchor=north, text width=\boxw, align=center] {
    \uline{3 cross-validations} \\ \hfill \\
    \includegraphics[height=1cm]{tinyksvr}
    $\qquad$
    \includegraphics[height=1cm]{tinysvr}
    $\qquad$
    \includegraphics[height=1cm]{tinyreg}
  };

  \node (rect7) [right=of rect3, box] { };
  \node at (rect7.north) [anchor=north, text width=\boxw, align=center] {
    \uline{kernel-PCA projection} \\ \hfill \\
    \includegraphics[height=1cm]{tinypca}
  };

  \coordinate [left=0.5 of rect5.west] (r5w);
  \coordinate [below=0.5cm of r] (r1ssw);
  \coordinate [below=of r1ssw] (r2nnw);
  \coordinate [below=0.5cm of g] (r1sse);
  \coordinate [below=of r1sse] (r2nne);
  \coordinate [below=3.5cm of r] (r2ssw);
  \coordinate [below=3.5cm of g] (r2sse);
  \coordinate [below=4.5cm of r] (r3nnw);
  \coordinate [below=4.5cm of g] (r3nne);
  \coordinate [below=6.5cm of r] (r3ssw);
  \coordinate [below=6.5cm of g] (r3sse);
  \coordinate [below=7.5cm of r] (r4nnw);
  \coordinate [below=7.5cm of g] (r4nne);
  \coordinate [below=0.5cm of rect4.north east] (r4ene);
  \coordinate [above=0.5cm of rect4.south east] (r4ese);

  \draw[next] (r1ssw) -- (r2nnw);
  \draw[next] (r1ssw) -- ($(r2nnw) + (-0.5cm, 0cm)$);
  \draw[next] (r1ssw) -- ($(r2nnw) + (0.5cm, 0cm)$);
  \draw[next] (r1sse) -- (r2nne);
  \draw[next] (r1sse) -- ($(r2nne) + (-0.5cm, 0cm)$);
  \draw[next] (r1sse) -- ($(r2nne) + (0.5cm, 0cm)$);
  \draw[next] (rect1.south) -- (rect2.north);
  \draw[next] (rect1.south) -- ($(rect2.north) + (-0.5cm, 0cm)$);
  \draw[next] (rect1.south) -- ($(rect2.north) + (0.5cm, 0cm)$);
  \draw [next] (rect2) -- (rect3);
  \draw [next] (r2ssw) -- (r3nnw);
  \draw [next] (r2sse) -- (r3nne);
  \draw [next] (rect3) -- (rect4);
  \draw [next] (r3ssw) -- (r4nnw);
  \draw [next] (r3sse) -- (r4nne);
  \draw [next] (rect4.east) -| (r5w) -- (rect5);
  \draw [thick] (r4ene) -- ++ (0.5cm, -0.5cm);
  \draw [thick] (r4ese) -- ++ (0.5cm, 0.5cm);
  \draw [next] (rect5) -- (rect6);
  \draw [next] (rect5) -- (rect7);
\end{tikzpicture}

\end{document}

  \caption[Variables used in tree kernel simulation experiments]
  {
    Values of parameters and other variables used in tree kernel simulation
    experiments. Each row corresponds to one of the \gls{BA} model parameters.
    One kernel matrix was created for every combination of values except the
    one indicated in the ``varied parameter'' column, which was varied when
    producing simulated trees.
  }
  \label{tab:kernelexpt}
\end{table}

\begin{table}[ht]
  \centering
  \begin{tabular}{cccccccc}
  parameter & grid values & test values & $N$ & $\alpha$ & $m$ & $I$ & tips \\
  \hline
  $N$ & 1050, 1125, \ldots, 15000 & 1000, 3000, \ldots, 15000 & - & 1.0 & 2 & 1000 & 100, 500, 1000 \\
  $\alpha$ & 0, 0.01, \ldots, 2 & 0, 0.25, \ldots 2 & 5000 & - & 2 & 1000 & 100, 500, 1000 \\
  $m$ & 1, 2, \ldots, 6 & 1, 2, \ldots 6 & 5000 & 1.0 & - & 1000 & 100, 500, 1000 \\
  $I$ & 500, 525, \ldots, 5000 & 500, 1000, 1500, 2000 & 5000 & 1.0 & 2 & - & 100, 500 \\
  \hline
\end{tabular}

  \caption[Variables used in grid search experiments]
  {
    Variables and \gls{BA} parameter values used for grid search experiments. 
    Trees were simulated under the test values, and compared to a grid of trees
    simulated under the grid values. Kernel scores were used to calculate point
    estimates and credible intervals for the test values.
  }
  \label{tab:gridexpt}
\end{table}
\end{landscape}

\begin{figure}[ht]
  \centering
  \includegraphics{kernel-expt.pdf}
  \caption[Schematic of experiments investigating impact of BA model parameters
           on tree shape.]{
    Schematic of experiments designed to investigate the impact of variations
    in BA model parameters on transmission tree shapes. The parameters of the
    BA model were varied one at a time {\color{blue}\uline{while holding all
    others fixed}}. Transmission trees were simulated under three different
    values of each parameter, then compared pairwise using the tree kernel.
    Classifiers were constructed for each parameter, and their accuracy was
    evaluated by cross-validation. Kernel-PCA projections were used to visually
    examine the separation of the trees in the feature space defined by the
    tree kernel.
  }
  \label{fig:kernelexpt}
\end{figure}

\subsubsection*{Grid search}

{\color{blue}\uline{The previous experiment was an exploratory analysis
intended to determine which of the \gls{BA} parameters were identifiable, and
whether the tree kernel could potentially be used to distinguish different
parameter values when all others were held fixed. In this experiment, which was
still of an exploratory nature, we continued to consider one parameter at a
time while fixing the other three. However, rather than checking for
identifiability, we were now interested in quantifying the accuracy and
precision of kernel score-based estimates. This was done by examining the
distribution of kernel scores on a grid of parameter values, when trees
simulated according to those values were compared with a single simulated test
tree. This experiment did not involve the full \gls{ABC}-\gls{SMC} algorithm,
nor did we continue to use Sackin's index or the \gls{nltt}.}}

As in the previous section, we will begin by describing a single experiment,
and then list the variables for which similar experiments were performed. We
varied \gls{alpha} along a narrowly spaced grid of values: 0, 0.01, \ldots, 2.
For each value, fifteen networks were generated with \software{igraph}, and
transmission trees were simulated over each using \software{nettree}. These
trees will be referred to as ``grid trees''. Next, one further test tree was
simulated with the test value \gls{alpha} = 0. Both the grid trees and the test
tree had 500 tips, and were simulated with the other \gls{BA} parameters set to
{\color{blue}\uline{the known values}} $N$ = 5000, $m$ = 2, and $I$ = 1000. The
test tree was compared to each of the grid trees using the tree kernel, with
the meta-parameters set to $\lambda = 0.3$ and $\sigma = 4$, using the
\software{treekernel} binary. The median kernel score was calculated for each
grid value, and the scores were normalized such that the area under the curve
was equal to 1. {\color{red}\sout{The grid value with the highest median kernel
score was taken as the point estimate for the test value}}{\color{blue}\uline{
A point estimate for the test value was obtained by calculating the weighted
mean of the grid values with the kernel scores as weights}}, and a 95\%
credible interval was obtained using the \software{hpd} function in the
\software{TeachingDemos} package~\autocite{snow2013teachingdemos}.

Each experiment of the type just described was repeated ten times with the same
test value. Similar experiments were performed for each of the four \gls{BA}
parameters, with several test values and trees of varying sizes. The variables
are listed in \cref{tab:gridexpt}. A graphical schematic of the grid search
experiments is shown in \cref{fig:gridexpt}. {\color{blue}\uline{We emphasize
that these experiments were marginal in nature; that is, the each parameter was
varied and estimated individually while holding the others at known, fixed
values.}}

%Spearman's correlation was used to test whether the number of tips in the tree
%was correlated with the acccuracy of the point estimates. One-way \gls{ANOVA}
%was used to test whether the accuracy of the point estimates were significantly
%higher for any parameter values. If so, the distribution of errors was examined
%to choose a suitable post hoc test. If there appeared to be a correlation
%between the parameter value and the accuracy of the estimate, Spearman's
%correlation was calculated. If there were one or more particular values which
%appeared to have lower error, we used the Wilcoxon rank-sum test.

\begin{figure}[ht]
  \centering
  \includegraphics[width=\textwidth]{gridsearch-expt}
  \caption[Schematic of grid search experiment.]{
    Graphical schematic of grid search experiments used to investigate \gls{BA}
    model parameters. Trees were simulated along a narrowly spaced grid of
    values of one parameter (``grid trees'') {\color{blue}\uline{with all other
    parameters fixed to known values}}. Separate trees were simulated for a
    small subset of the grid values (``test trees''), {\color{blue}\uline{also
    holding the other parameters fixed}}. Each test tree was
    compared to every grid tree using the tree kernel, and the resulting kernel
    scores were normalized to resemble a probability density from which the
    mode and 95\% highest density interval were calculated.
  }
  \label{fig:gridexpt}
\end{figure}

\subsubsection*{Approximate Bayesian computation}

{\color{blue}\uline{
Our final simulation-based experiment was designed to test the full
\gls{ABC}-\gls{SMC} algorithm by jointly estimating the four parameters of the
\gls{BA} model. In contrast to the previous two experiments, we did not inform
the algorithm of any of the true parameter values.
}}

We simulated three trees each under a variety of parameter values, and ran the
\software{netabc} program to estimate posterior distributions for the
parameters. The parameter values and priors used are listed in
\cref{tab:abcexpt}. The tree kernel meta-parameters were set to $\lambda = 0.3$
and $\sigma = 4$. The \gls{SMC} algorithm was run with 1000 particles, five
sampled datasets per particle, and the $\alpha$ parameter (not to be confused
with the \gls{BA} preferential attachment parameter, see
\cref{subsec:adaptsmc}) set to 0.95. The algorithm was stopped when the
acceptance rate of the \gls{MH} kernel dropped below 1.5\%, the same criterion
used by \citeauthor{del2012adaptive}. Approximate marginal posterior densities
for each parameter were calculated using the \software{density} function in
\software{R} applied to the final weighted population of particles.
{\color{red}\sout{Credible intervals were obtained for each parameter using the
\software{HPDinterval} function in the \software{coda}
package~\autocite{plummer2006coda}.}} {\color{blue}\uline{Posterior means
obtained for each parameter using the \software{wtd.mean} function in the
\software{Hmisc} package~\autocite{harrell2016hmisc}. Credible intervals were
obtained using the \software{hpd} function in the \software{TeachingDemos}
package~\autocite{snow2013teachingdemos}.}}

\begin{table}[ht]
  \centering
  \begin{tabular}{ccc}
  parameter or variable & test values & prior \\
  \hline
  $N$ & 5000 & Uniform(500, 15000) \\
  $\alpha$ & 0, 0.5, 1, 1.5 & Uniform(0, 2) \\
  $m$ & 2, 3, 4 & DiscreteUniform(1, 5) \\
  $I$ & 1000, 2000 & Uniform(500, 5000) \\
  tips & 500 & - \\
  \hline
\end{tabular}

  \caption[Variables used in grid search experiments]
  {
    Variables and \gls{BA} parameter values used for \gls{ABC} validation
    experiments. Trees were simulated under the test values, and
    kernel-assisted \gls{ABC} was used to re-estimate posterior distributions for the
    \gls{BA} parameters without training.
  }
  \label{tab:abcexpt}
\end{table}

{\color{blue}\uline{
To evaluate the effects of the true parameter values on the accuracy of the
posterior mean estimates, we analysed the \gls{alpha} and \gls{I} parameters
individually using \glspl{GLM}. The response variable was the error of the
point estimate, and the predictor variables were the true values of
\gls{alpha}, \gls{I}, and \gls{m}. We did not test for differences across true
values of \gls{N}, because \gls{N} was not varied in these simulations. The
distribution family and link function for the \glspl{GLM} were chosen as
Gaussian and inverse, respectively, by examination of residual plots and
\gls{AIC}. The $p$-values of the estimated \glspl{GLM} coefficients were
corrected using Holm-Bonferroni correction~\autocite{holm1979simple} with $n =
6$ (two \glspl{GLM} with three predictors each). Because there was clearly
little to no identifiability of the \gls{N} and \gls{m} parameters with
\gls{ABC} (see Results in next section), we did not construct \glspl{GLM} for
those parameters. }}

%\subsubsection*{Characterization of power-law exponent in Barab\'asi-Albert networks}
%
%Most studies of social network or transmission network
%parameters~\autocite[e.g.][]{liljeros2001web, jones2003assessment,
%schneeberger2004scale, brown2011transmission} report the coefficient
%\gls{gamma} of the power law degree distribution. To make our results
%comparable to previous work, we used simulated networks to investigate the
%relationship between the \gls{BA} model parameters and \gls{gamma}. A network
%was simulated for each combination of parameters listed in
%\cref{tab:gammaexpt}. A power law distribution was fitted to the degree
%distribution of each simulated network using the \software{fit\_power\_law}
%function in \software{igraph} with the `R.mle' implementation. We fitted a
%\gls{GLM} with Gamma-distributed errors and a log link function to the observed
%distribution of \gls{gamma} values, with \gls{alpha}, \gls{m}, \gls{N}, and all
%possible interaction terms as predictors.
%
%\begin{table}
%  \centering
%  \begin{tabular}{cc}
  parameter & values \\
  \hline
  $N$ & 500, 600, \ldots, 15000 \\
  $\alpha$ & 0, 0.01, \ldots, 2 \\
  $m$ & 1, 2, \ldots, 8 \\
  \hline
\end{tabular}

%  \caption[\gls{BA} parameters used as input \gls{GLM} predicting $\gamma$]
%  {
%    \gls{BA} model parameters used as input to \gls{GLM} predicting power law
%    exponent $\gamma$. One network was simulated with each combination of
%    parameters, and $\gamma$ was calculated for each network. A \gls{GLM} with
%    Gamma-distributed errors and a log link function was fit to the $\gamma$
%    values with all parameters and interaction terms as predictors.
%  }
%  \label{tab:gammaexpt}
%\end{table}

Two further experiments were performed to address potential sources of error.
To evaluate the effect of model misspecification in the case of heterogeneity
among nodes, we generated a network where half the nodes were attached with
power $\alpha$ = 0.5, and the other half with power $\alpha$ = 1.5. The other
parameters for this network were $N$ = 5000, $I$ = 1000, and $m$ = 2. To
investigate the effects of potential sampling
bias~\autocite{karcher2016quantifying}, we simulated a transmission tree where
the tips were sampled in a peer-driven fashion, rather than at random. That is,
the probability to sample a node was twice as high if any of that node's
network peers had already been sampled. The parameters of this network were $N$
= 5000, $I$ = 2000, $m$ = 2, and $\alpha$ = 0.5.

\subsection{Results}

\subsubsection*{Classifiers for BA model parameters based on tree shape}



Trees simulated under different values of \gls{alpha} were visibly quite
distinct (\cref{fig:alphatrees}). In particular, higher values of \gls{alpha}
produce networks with a small number of highly connected nodes which, once
infected, are likely to transmit to many other nodes. This results in a more
unbalanced, ladder-like structure in the phylogeny, compared to networks with
lower \gls{alpha} values. None of the other three parameters produced trees
which were as easily distinguished from each other
(\cref{fig:Itrees,fig:mtrees,fig:Ntrees,fig:Itrees}).  Sackin's index, which
measures tree imbalance, was significantly correlated with all four parameters
    (for $\alpha$, $I$, $m$, and $N$ respectively: Spearman's rho =
     0.85,
     \ensuremath{-0.12},
     \ensuremath{-0.13},
     0.09;
     $p$-values
     ${<}10^{-5}$,
     $0.003$,
     ${<}10^{-5}$,
     ${<}10^{-5}$).
The ratio of internal to terminal branch lengths was negatively correlated with
\gls{alpha} and \gls{I}, and positively correlated with \gls{m} and \gls{N}
  (Spearman's rho
    \ensuremath{-0.84},
    \ensuremath{-0.69},
    0.1,
    0.18;
  all $p < 10^{-5}$).

\begin{figure}[ht]
  \centering
  \includegraphics[width=\textwidth]{kernel-alpha-tree.pdf}
  \caption[Simulated transmission trees under three different values of BA parameter $\alpha$]{
    Simulated transmission trees under three different values of BA parameter
    $\alpha$. Epidemics were simulated on \gls{BA} networks of 5000 nodes, with
    \gls{alpha} equal to 0.5, 1.0, or 1.5, until 1000 individuals were
    infected. Transmission trees were created by sampling 500 infected nodes.
    Higher \gls{alpha} values produced networks with a small number of
    highly-connected nodes, resulting in highly unbalanced, ladder-like trees.
  }
  \label{fig:alphatrees}
\end{figure}

\Cref{fig:kpca} shows \gls{kPCA} projections of the simulated trees onto the
first two principal components of the kernel matrix. The figure shows only the
simulations with 500-tip trees and 1000 infected nodes. The three \gls{alpha}
and \gls{I} values considered are well separated from each other in feature
space. On the other hand, the three \gls{N} values overlap significantly, and
the three \gls{m} values are virtually indistinguishable. Similar observations
can be made for other values of \gls{I} and the number of tips
(\cref{fig:alphakpca,fig:Nkpca,fig:Ikpca,fig:mkpca}). The values of \gls{I} and
\gls{N} separated more clearly with larger numbers of tips, and in the case of
\gls{N}, larger epidemic sizes.

\begin{figure}[ht]
  \centering
  \includegraphics{kernel-kpca.pdf}
  \caption[Kernel-PCA projections of simulated trees under varying BA
           parameter values.]{
    Each parameter of the \gls{BA} model was individually varied to produce 300
    simulated trees. Kernel matrices were formed from all pairwise kernel
    scores among each set of 300 trees. The trees were projected onto the first
    two principal components of the kernel matrix calculated using \gls{kPCA}.
    All trees had 500 tips. The parameters not being varied were set to
    \gls{alpha} = 1, \gls{I} = 1000, \gls{m} = 2, and \gls{N} = 5000. The tree
    kernel meta-parameters were $\lambda = 0.3$ and $\sigma = 4$.
  }
  \label{fig:kpca}
\end{figure}



Accuracy of the \gls{kSVR} classifiers varied based on the parameter being
tested (\cref{fig:rsquared}, left). Classifiers based on two other tree
statistics, the \gls{nltt} and Sackin's index, generally exhibited worse
performance than the tree kernel, although the magnitude of the disparity
varied between the parameters (\cref{fig:rsquared}, centre and right). The
results were largely robust to variations in the tree kernel meta-parameters
$\lambda$ and $\sigma$, although accuracy varied between different epidemic and
sampling scenarios
(\cref{fig:alphacrossv,fig:mcrossv,fig:Icrossv,fig:Ncrossv}).

When classifying $\alpha$, the \gls{kSVR} classifier had an average $R^2$ of 
    0.92,
compared to 
    0.56
for the \gls{nltt}-based SVR, and
    0.75
for the linear regression against Sackin's index. There was little variation
about the mean for different tree and epidemic sizes. No classifier could
accurately identify the $m$ parameter in any epidemic scenario, with average
$R^2$ values of 
  0.12 for \gls{kSVR},
  0.01 for the \gls{nltt}, and
  0.06
for Sackin's index. Again, there was little variation in accuracy between
epidemic scenarios, although the accuracy of the \gls{kSVR} was slightly higher
on 1000-tip trees 
    (average $R^2$ 
     0.01,
     0.11,
     0.32
     for 100, 500, and 1000 tips respectively).

The accuracy of classifiers $I$ varied significantly with the number of tips in
the tree. For 100-tip trees, the average $R^2$ values were
  0.7,
  0.55, and
  0.02
for the tree kernel, \gls{nltt}, and Sackin's index respectively. For 500-tip
trees, the values increased to
  0.93,
  0.83, and
  0.07.
Finally, the performance of classifiers for $N$ depended heavily on the
epidemic scenario. The $R^2$ of the \gls{kSVR} classifier ranged from
  0.08
for the smallest epidemic and smallest sample size, to
  0.82
for the largest. Likewise, $R^2$ for the \gls{nltt}-based SVR ranged from 
  0.01
to
  0.54.
Sackin's index did not accurately classify $N$ in any scenario, with an average
$R^2$ of
  0.03
and little variation between scenarios.

\begin{figure}[ht]
  \centering
  \includegraphics[width=\textwidth]{kernel-rsquared.pdf}
  \caption[Cross-validation accuracy of kernel-SVR, nLTT-based SVR, and
  Sackin's index regression classifiers for BA model parameters.]{
      Cross-validation accuracy of kernel-SVR classifier (left), SVR classifier
      using \gls{nltt} (centre), and linear regression using Sackin's index
      (right) for \gls{BA} model parameters. Kernel meta-parameters were set to
      $\lambda = 0.3$ and $\sigma = 4$. Each point was calculated based on 300
      simulated transmission trees over networks with three different values of
      the parameter being tested. Vertical lines are empirical 95\% confidence
      intervals based on 1000 two-fold cross-validations. {\color{blue}\uline{
      The classifiers for \gls{I} were not evaluated with 1000-tip trees,
      because one of the tested \gls{I} values was 500, and it is not possible
      to sample a tree of size 1000 from 500 infected individuals.}}
  }
  \label{fig:rsquared}
\end{figure}

\subsubsection*{Marginal parameter estimates with grid search}













