% for intro about power law
Strictly speaking, when $\alpha \neq 1$, the degree distribution no longer
obeys a power law, but we do not believe this affects the applicability of the
model to real world networks. When $\alpha < 1$, the distribution follows a
stretched exponential~\autocite{krapivsky2000connectivity}.
\textcite{laherrere1998stretched} argue that stretched exponentials offer a
better explanation for diverse natural phenomena than power laws, and also
point out that the two distributions are often similar except in the far right
tail. When $\alpha > 1$, the degree distribution exhibits behaviour referred to
as \defn{gelation}, namely a small number of hub nodes, possibly only one,
which collectively are connected to every other node in the graph. Despite
these deviations from the power law distribution, the idea that the strength of
preferential attachment may be more or less than exactly linear, and that it
may vary between geographic and demographic contexts, seems a natural one.
In fact, \textcite{de2007preferential} were able to estimate the \acrlong{PA}
power from partner count data collected from the same individuals for
consecutive time intervals, and found a value less than one in all cases.  It
is also worth noting that, in addition to the \gls{BA} model, other
investigations of the interaction between contact networks and transmission
trees have studied the \gls{ER} and \gls{WS} models, whose degree distributions 
do not generally follow a power law under any parameter settings.

% intro or objective - it's hard to get real contact networks
True contact networks have been estimated in several communities experiencing 
sexually transmitted infection outbreaks (\autocite{clemencon2015statistical},
also see \autocite{morris2011hiv, rothenberg2007large} and references therein).
Though valuable, this kind of investigation is a substantial undertaking,
especially in the time-sensitive situation of an emerging epidemic. The
in-depth surveying, contact tracing, and data analysis required are simply
infeasible in most contexts; even if the resources exist, there are often more
pressing concerns.

% methods: fitting gamma to real data
As discussed in \cref{subsec:pa}, the degree distributions generated by values
of \gls{alpha} not equal to one do not follow a power law, but rather a
stretched exponential (for \gls{alpha} $<$ 1) or a gelation-type distribution
(for \gls{alpha} $>$ 1). However, for many 

% results: fitting gamma to real data

% discussion about PA power estimation paper
For all datasets we examined, the posterior mean estimates for \gls{alpha} were
sub-linear, ranging from X to X. The sub-linearity is consistent with the
results of \textcite{de2007preferential}, who developed a statistical inference
method to estimate the parameters of a more sophisticated \acrlong{PA} model
incorporating heterogeneous node behaviour. They found \gls{alpha} values
ranging from 0.26 to 0.62, depending on the gender and time period considered. 
Our estimates of \gls{alpha} for the \textcite{niculescu2015recent} was above
this range under both priors, as were the estimates for the
\textcite{wang2015targeting} data and the BC data when \gls{m} = 1 was
disallowed by the prior. The dataset investigated by
\textcite{de2007preferential} was derived from a survey of a random sample of
the Norwegian population, whereas our investigation focused on datasets from
known phylogenetic or geographic clusters of \gls{HIV} infected persons. It is
therefore unsurprising that we detected stronger \acrlong{PA} dynamics in some
cases. For instance, random sampling is much less likely to discover the
high-degree nodes characterizing the tail of the degree distribution, simply
because those individuals are rare in the general population. In addition, it
is plausible that \gls{HIV}-positive individuals are more likely to be highly
connected in their sexual networks, as the odds of acquiring \gls{HIV} increase
with the number of unprotected sexual contacts.

Both \textcite{de2007preferential} and \textcite{novitsky2014impact} studied
populations whose primary risk factor for \gls{HIV} infection was heterosexual
contact. \citeauthor{de2007preferential} explicitly excluded reported
homosexual contacts; \citeauthor{novitsky2014impact} did not, but noted that
heterosexual contact is the primary mode of transmission in Botswana where the
study was done. In the first of the two papers describing the Botswana
study~\autocite{novitsky2013phylogenetic}, the authors noted that their sample
was gender-biased, being composed of approximately 75\% women. Our estimate of
\gls{alpha} for these data was X or X, depending on the prior on \gls{m};
\citeauthor{de2007preferential} estimated 0.54, 0.57, and 0.29 for 3-year,
5-year, and lifetime partnership networks respectively for the female portion
of their sample.

For both choices of prior on \gls{m}, the datasets derived from \gls{IDU}
populations had a higher estimated \acrlong{PA} power than the other datasets
(\cref{fig:abchpd,fig:abchpdm2}). This finding is in line with
\textcite{dombrowski2013topological}, who reanalyzed a network of \glspl{IDU}
in Brooklyn, USA, collected between 1991 and
1993~\autocite{friedman2006social}. They found that the the \gls{IDU} network 
resembled a \gls{BA} network much more closely than other social and sexual 
networks, and offered sociological explanations for the apparent \acrlong{PA}
dynamics in this population. Importantly, from a public health perspective,
the authors asserted that the removal of \emph{random} individuals from
\gls{IDU} networks may have the paradoxical effect of increasing the network's
epidemic susceptibility. When low-degree nodes are removed, as would occur
during a police crackdown, their network neighbours may turn to well-known
community members for advice or supplies, thus increasing the connectivity of
these high-degree nodes.

One somewhat surprising result was the difference between parameter estimates 
for the \textcite{li2015hiv} and \textcite{wang2015targeting} datasets. Both
groups studied cohorts of acutely infected \gls{MSM} in major Chinese cities
(Shanghai and Beijing respectively); yet, the \citeauthor{li2015hiv} data was 
estimated to have a lower \acrlong{PA} power and larger infected population
than the \textcite{wang2015targeting} data. 

% for discussion
In order to compare our results to existing literature on networks and
distributions of partner counts, we have reported estimated values for the
power law exponent \gls{gamma} of the real data sets we evaluated. However, the
posterior means for \gls{alpha} for all six datasets were less than one; the
degree distributions in this parameter range are stretched exponential, not
power law~\autocite{krapivsky2000connectivity}. As we show in
\cref{fig:powerlaw}, the power law fit does capture the slope of the degree
distribution fairly well, but the results should still be interpreted
cautiously. \textcite{krapivsky2000connectivity} showed that the power law
distribution can be maintained, with \gls{gamma} tuned to any desired value, by
a straightforward modification of the \gls{BA} model. The authors define the
``connection kernel'' $A_k$ the probability of a new connection to a node of
degree $k$, up to a normalizing constant. In the \gls{BA} model as we have
presented it here, $A_k = k^{\alpha} + 1$. Taking $A_k$ as any asymptotically
linear function will result in a power law distribution, with the exponent
\gls{gamma} determined by the properties of $A_k$. Implementing such a model
would be straightforward and seems a natural next step toward improving the
realism of the \gls{BA} model.

\textcite{rothenberg2007large} estimated the power law exponents for 15 major
network studies of sexual and injection drug use networks carried out between
1981 and 2000. The estimated \gls{gamma} values for ``all contacts'' (that is,
including both interviewed and non-interviewed individuals) were 
between 2.08 and 2.87 for all but one of the networks. 

% misc
romano2010social (Social Networks Shape the Transmission Dynamics of
Hepatitis C Virus) found power law exponent 2.15 with xmin = 10

hughes2009molecular (Molecular Phylodynamics of the Heterosexual HIV Epidemic
in the United Kingdom) found a power law was a good fit to UK sexual networks
with slope = 2.1

see morris2011hiv (http://www.icpsr.umich.edu/icpsrweb/NAHDAP/studies/22140) and
rothenberg2007large (Large-Network Concepts and Small-Network Characteristics:
Fixed and Variable Factors) for empirical HIV networks

clemencon2015statistical estimated xmin = 7, gamma = 3.02 for MSM (see supplements)

otoh schneeberger has gamma estimates for MSM of 1.75, 1.57, 1.43

zarabi2013combining may have some data for mixed risk group studies, useful for cuevas
