Our objective is to develop a method to estimate structural parameters of the
contact network underlying an observed transmission tree.

\subsection{Prior work}

The present study is closely related to three groups of work with distinct
objectives, which will be reviewed in detail below. 
%, by
%comparing trees simulated from different network models. These have
%demonstrated that network structure does profoundly influence tree shape, but
%did not attempt to quantitatively infer the parameter from observed trees. The
%third group consists of studies which infer a structural network parameter from
%observed infection and recovery times, using Bayesian methods. These are most
%similar in aims to our work, but they rely on epidemiological observations
%rather than evolutionary information from phylogenies.

The first and largest group of related studies are phylodynamic investigations
of epidemiological parameters such as transmission rate, recovery rate, and
basic reproductive number~\autocite{pybus2009evolutionary, volz2013viral}. Like
our work, these studies make inferences about epidemiological processes from
the genetic diversity of virus populations, which is usually represented in the
form of a phylogeny. The majority of these employ a Bayesian \gls{MCMC}
approach to infer parameters of an epidemiological model whose likelihood can
be calculated, most often some variation of the
birth-death~\autocite{kendall1948generalized} or
coalescent~\autocite{kingman1982coalescent} models. Stadler
\etal~\autocite{stadler2011estimating} develop a formula for the likelihood of
a phylogeny with heterochronous tips under the birth-death model, which has
been used to estimate the basic reproductive number of several viral
epidemics~\autocite{stadler2011estimating}. However, the birth-death model is
cannot tell us anything about population structure, as it assumes that every
individual becomes infected at the same rate. Volz~\autocite{volz2012complex}
writes down the likelihood of a heterochronous phylogeny under a coalescent
model with arbitrarily complex population dynamics. This opens the door to more
complex inferences about population structure, as the population can be
partitioned into compartments with different transmission and recovery rates,
but still assumes that each compartment is homogeneously mixed. In other words,
the coalescent model can tell us about the \emph{global} structure of a
population, such as whether there exists a high-risk subgroup, but not about
the \emph{local} structure, such as the average number of contacts each
individual has. 

The use of models which assume a \defn{panmictic} (that is, homogeneously
mixed) population in phylodynamics has become widespread, so it is natural that
some researchers have investigated the effects of this assumption. In
particular, since phylodynamic methods use phylogenies as their data source,
several studies have examined the shapes of phylogenies arising from
non-panmictic populations.

Finally, a third group of studies has used Bayesian methods to infer a
structural network parameter from observed infection and recovery times. This
third group is most similar in aims to our own work; rather than using
epidemiological observations, we employ viral phylogenies. These were discussed
in \cref{subsec:netoverview}.
