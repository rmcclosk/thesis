\label{sec:obj}
\glsreset{ABC}
\glsreset{BA}
\glsreset{SMC}

The spread of a disease is most often modelled by assuming either a
homogeneously mixed population~\autocite{hamer1906milroy,
kermack1927contribution}, or a population divided into a small number of
homogeneously mixed groups~\autocite{rushton1955deterministic}. This
assumption, also called \defn{mass
action}~\autocite{heesterbeek2000mathematical}, or \defn{panmixia}, implies
that any two individuals in the same compartment are equally likely to come
into contact making transmission possible at some predefined rate. Although
this provides a reasonable approximation in many
cases~\autocite{anderson1992infectious}, the error introduced by assuming a
panmictic population can be substantial when significant contact heterogeneity
exists in the underlying
population~\autocite{bansal2007individual,barthelemy2005dynamical,keeling2005networks}.
Contact network models provide an alternative to compartmental models which do
not require the assumption of panmixia. In addition to more accurate
predictions, the parameters of the networks themselves may be of interest from
a public health perspective. For example, certain vaccination strategies may be
more or less effective in curtailing an epidemic depending on the underlying
network's degree distribution~\autocite{peng2013vaccination, ma2013importance}.
Phylodynamic methods, \add{which link viruses' evolutionary and epidemiological
dynamics}, have been used to fit many different types of models to phylogenetic
data~\autocite{pybus2009evolutionary,volz2013viral}. However, these models
generally assume a panmictic population. The primary objective of this work is
\emph{to develop a method to fit contact network models, \add{and thereby relax
the assumption of homogeneous mixing,} in a phylodynamic framework.}

\newcommand{\G}{\mathcal{G}}
\newcommand{\Nu}{\mathcal{N}}

\add{In this work, we take a Bayesian approach: our goal is to
estimate the posterior distribution on model parameters given our data,}
\[
  \madd{\post(\theta \mid T) = \frac{\lik(T \mid \theta) \prior(\theta)}
    {\int_\Theta \lik(T \mid \theta) \prior(\theta) \d \theta}},
\]
\add{where $\lik(T \mid \theta)$ is the likelihood of the parameters given $T$,
$\prior(\theta)$ is the prior on $\theta$, and $\Theta$ is the space of
possible model parameters. The denominator on the right-hand side is the
marginal probability of $T$ which acts as a normalizing constant on the
posterior (see \cref{chp:prelim} for a review of mathematical modelling and
Bayesian inference, including definitions of these concepts). As we shall show
(\cref{subsec:just}), estimating this distribution presents computational
challenges beyond those usually encountered in Bayesian inference. Both the
likelihood $\lik(T \mid \theta)$ and the normalizing constant seem to be
intractable, which rules out the use of most common \acrlong{ML} and Bayesian
methods.}

\del{Calculating the likelihood of the parameters of a contact network model
seems likely to be an intractable problem. We have not proven this is the case,
but some intuition can be provided by examining the process involved in the
likelihood calculation. Consider a contact network model with parameters
$\theta$ and an estimated transmission tree $T$ with $n$ tips. In general, we
do not know the labels of the internal nodes of $T$, only the labels of its
tips. To fit this model using likelihood-based methods, we must calculate the
likelihood of $\theta$, that is, $\Pr(T \mid \theta)$. Let $\G$ be the set of
all possible contact networks, and $\Nu$ be the set of all possible labellings
of the internal nodes of $T$. We can write the likelihood as}
\begin{align}
\begin{split}
  \label{eq:netlik}
  \mdel{\Pr(T \mid \theta)}
    &\mdel{=\sum_{\nu \in \Nu} \Pr(T, \nu \mid \theta)} \\
    &\mdel{=\sum_{G \in \G} \sum_{\nu \in \Nu} \Pr(T, \nu \mid G, \theta) \Pr(G \mid \theta)} \\
    &\mdel{=\sum_{G \in \G} \sum_{\nu \in \Nu} \Pr(T, \nu \mid G) \Pr(G \mid \theta),}
\end{split}
\end{align}
\del{the last equality following from the fact that $T$ and $\nu$ depend only
on $G$, not on $\theta$. Although $\Pr(T, \nu \mid G)$ and $\Pr(G \mid \theta)$
may individually be straightforward to calculate, the number of possible
directed graphs on $N$ nodes is $2^{N(N-1)}$~\autocite{harary2014graphical},
larger if the nodes and edges in the graph may have different labels or
attributes. Hence, the number of terms in the sum is at least exponential in
$n$, as there must be at least $n$ nodes in the network. In addition,
\cref{eq:netlik} assumes that $T$ is complete, meaning that all infected
individuals were sampled. This is rarely the case in practice - most often, we
only have access to a subset of the infected individuals. In this case, the
likelihood calculation becomes even more complex, because we must also sum over
all possible complete trees.}

\del{Depending on the network model studied, it is possible that
\cref{eq:netlik} could be simplified into a tractable expression.} An
alternative to likelihood-based methods, which could be applied to any network
model, is provided by \gls{ABC}~\autocite{rubin1984bayesianly,
tavare1997inferring, fu1997estimating, beaumont2002approximate}. All of the
ingredients required to apply \gls{ABC} to this problem are readily available.
Simulating networks is straightforward under a variety of models. Epidemics on
those networks, and the corresponding transmission trees, can also be easily
simulated. As mentioned above, contact networks can profoundly affect
transmission tree shape. Those shapes can be compared using a highly
informative similarity measure called the ``tree
kernel''~\autocite{poon2013mapping}; \add{similar kernel functions have been
demonstrated to work well as distance functions in
\gls{ABC}~\autocite{park2015k2}}. \Gls{ABC} can be implemented with several
algorithms, but \gls{SMC} has advantages over others,
\add{including improved accuracy in low-density regions and
parallelizability}~\autocite{mckinley2009inference}. A recently-developed
adaptive algorithm requiring minimal tuning on the part of the user makes
\gls{SMC} an even more attractive approach~\autocite{del2012adaptive}. In
summary, our method to infer contact network parameters will combine the
following: stochastic simulation of epidemics on networks, the tree kernel, and
adaptive \gls{ABC}-\gls{SMC}. \add{Our method will expand on the framework
developed by~\autocite{poon2015phylodynamic}, who combined \gls{ABC} with the
tree kernel to infer parameters of population genetic models from viral
phylogenies using \gls{MCMC}.}

Empirical studies of sexual contact networks have found that these networks
tend to be scale-free~\autocite{colgate1989risk, liljeros2001web,
schneeberger2004scale,clemenccon2015statistical}, meaning that their degree
distributions follow a power law (although there has been some disagreement,
see \autocite{handcock2004likelihood, bansal2007individual}). Preferential
attachment has been postulated as a mechanism by which scale-free networks
could be generated~\autocite{barabasi1999emergence}. The \gls{BA}
model~\autocite{barabasi1999emergence} is one of the simplest preferential
attachment models, which makes it a natural choice to explore with our method.
The second aim of this work is \emph{to use simulations to investigate the
parameters of the \acrlong{BA} model, including whether they have a detectable
impact on tree shape, and whether they can be accurately recovered using
\gls{ABC}.}

Due to its high global prevalence and fast mutation rate, \gls{HIV} is one of
the most commonly-studied viruses in a phylodynamic context. Consequently, a
large volume of \gls{HIV} sequence data is publicly available, more than for
any other pathogen, and including sequences sampled from diverse geographic and
demographic settings. At the time of this writing, there were $635,400$ HIV
sequences publicly available in GenBank, annotated with 172 distinct countries
of origin. Since \gls{HIV} is almost always spread through either sexual
contact or sharing of injection drug supplies, the contact networks underlying
\gls{HIV} epidemics are driven by social dynamics and are therefore likely to
be highly structured~\autocite{clemenccon2015statistical}. Moreover, since no
cure yet exists, efforts to curtail the progression of an epidemic have relied
on preventing further transmissions through measures such as \gls{tasp} and
education leading to behaviour change. The effectiveness of this type of
intervention can vary significantly based on the underlying structure of the
network and the particular nodes to whom the intervention is
targeted~\autocite{little2014using,wang2015targeting}. Due to this combination
of data availability and potential public health impact, \gls{HIV} is an
obvious context in which our method could be applied. Therefore, the third and
final aim of this work is \emph{to apply \gls{ABC} to fit the \acrlong{BA}
model to existing \gls{HIV} outbreaks}.

To summarize, this work has three objectives. First, we will develop a method
which uses \gls{ABC} to infer parameters of contact network models from
observed transmission trees. Second, we will use simulations to characterize
the parameters of the \gls{BA} network model in terms of their effect on tree
shape and how accurately they can be recovered with \gls{ABC}. Finally, we will
apply the method to fit the \gls{BA} model to several real-world \gls{HIV}
datasets.

\add{The remainder of this background chapter is organized in four sections.
The first section introduces phylogenies and transmission trees, which are the
input data from which our method aims to make statistical inferences. This
section also introduces phylodynamics, a family of methods that, like ours, aim
to infer epidemiological parameters from evolutionary data. The second section
focuses on contact networks and network models, whose parameters we are
attempting to infer. The relationship between contact networks and transmission
trees is also discussed. The third and fourth sections introduce \gls{SMC}
and \gls{ABC} respectively, which are the two algorithmic components of the
method we will implement. In particular, \gls{ABC} refers to the general
approach of using simulations to replace likelihood calculations in a Bayesian
setting, while \gls{SMC} is a particular algorithm which can be used to
implement \gls{ABC}.}

\glsreset{ABC}
\glsreset{BA}
\glsreset{HIV}
\glsreset{ML}
\glsreset{SMC}
