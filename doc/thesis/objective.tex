Our objective is to develop a method to estimate structural parameters of the
contact network underlying an observed transmission tree.

\subsection{Prior work}

The present study is closely related to three groups of work within
computational epidemiology, which will be discussed in detail below. The first
and largest of these are phylodynamic investigations of epidemiological
parameters, such as transmission rate, recovery rate, and basic reproductive
number. Like our work, these studies make inferences from evolutionary
information in the form of a viral phylogeny, but their aims do not include
uncovering population structure and they typically assume a homogeneously mixed
population. A second, more recent group of studies has evaluated the effect of
network structure on transmission tree shape, by comparing trees simulated from
different network models. These have demonstrated that network structure does
profoundly influence tree shape, but did not attempt to quantitatively infer
the parameter from observed trees. The third group consists of studies which
infer a structural network parameter from namely observed infection and
recovery times using Bayesian methods. These are most similar in aims to our
work, but they rely on epidemiological observations rather than evolutionary
information from phylogenies.

In recent years, a large number of studies have been carried out using
phylodynamic methods to infer epidemiological quantities of interest. The
majority of these employ a Bayesian \gls{MCMC} approach to infer parameters of
an epidemiological model whose likelihood can be calculated, most often some
variation of the birth-death~\autocite{kendall1948generalized} or
coalescent~\autocite{kingman1982coalescent} models. Stadler
\etal~\autocite{stadler2011estimating} develop a formula for the likelihood of
a phylogeny with heterochronous tips under the birth-death model, which has
been used to estimate the basic reproductive number of several viral
epidemics~\autocite{stadler2011estimating}. However, the birth-death model is
cannot tell us anything about population structure, as it assumes that every
individual becomes infected at the same rate. Volz~\autocite{volz2012complex}
writes down the likelihood of a heterochronous phylogeny under a coalescent
model with arbitrarily complex population dynamics. This opens the door to more
complex inferences about population structure, as the population can be
partitioned into compartments with different transmission and recovery rates,
but still assumes that each compartment is homogeneously mixed. In other words, 
the coalescent model can tell us about the \emph{global} structure of a
population, such as whether there exists a high-risk subgroup, but not about
the \emph{local} structure, such as the average number of contacts each
individual has.

The use of models which assume a \defn{panmictic} (that is, homogeneously
mixed) population in phylodynamics has become widespread, so it is natural that
some researchers have investigated the effects of this assumption. In
particular, since phylodynamic methods use phylogenies as their data source,
several studies have examined the shapes of phylogenies arising from
non-panmictic populations. \citeauthor{o2010contact}~(\citeyear{o2010contact})
simulated epidemics over networks with four types of degree distribution. They
then estimated the Bayesian skyride~\autocite{minin2008smooth} population size
trajectory in two ways: from the phylogeny, using \gls{MCMC}; and from the
incidence and prevalence trajectories, using the method developed by Volz
\etal~\autocite{volz2009phylodynamics}. They found that the concordance between
the two skyrides, as well as the relationship between the skyride and
prevalence curve, was qualitatively different for each degree distribution.
\citeauthor{leventhal2012inferring} simulated transmission trees over \gls{ER},
\gls{WS}, \gls{BA}, and full networks with fixed number of nodes and mean degree.
They calculated Sackin's index of the simulated trees while varying the
epidemic, network, and sampling parameters, and found that the relationship
between these parameters and Sackin's index varies considerably among the
different network models.

The third group of studies was pioneered by Britton and
O'Niell~\autocite{britton2002bayesian}, who developed a Bayesian method to
infer the $p$ parameter of an \gls{ER} network, along with the $\beta$ and
$\gamma$ parameters of \gls{SI} model. Their method is able to use either
infection and removal times, or removal times only. However, it is designed for
only a small number of observations (15 and 42 cases in their applications),
and their estimates of $p$ for real outbreaks were mostly uninformative (95\%
confidence intervals [0.11-0.96] and 0.055-0.96]; $p$ is bounded in [0, 1]).
