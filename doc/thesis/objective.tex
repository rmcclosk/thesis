\glsreset{ABC}

The spread of a disease is most often modelled by assuming either a
homogeneously mixed population~\autocite{hamer1906milroy,
kermack1927contribution}, or a population divided into a small number of
homogeneously mixed groups~\autocite{rushton1955deterministic}. This
assumption, also called the ``law of mass action'' or \defn{panmixia}, implies
that any two individuals in the same compartment are equally likely to come
into contact causing transmission. Although this provides a reasonable
approximation in many cases~\autocite{anderson1992infectious}, the error
introduced by assuming a panmictic population can be substantial when
significant contact heterogeneity exists in the underlying population. Contact
network models provide an alternative to compartmental models which do not
require the assumption of panmixia. In addition to more accurate predictions,
the parameters of the networks themselves may be of interest from a public
health perspective. For example, certain vaccination strategies may be more or
less effective in curtailing an epidemic depending on the underlying network's
degree distribution~\autocite{peng2013vaccination}. Phylodynamic methods have
been used to fit many different types of model to phylogenetic
data~\autocite{pybus2009evolutionary,volz2013viral}, but as far as we know, no
methods have yet been developed to fit contact network models. The primary
objective of this work is to develop such a method.

\newcommand{\G}{\mathcal{G}}
\newcommand{\Nu}{\mathcal{N}}

Calculating the likelihood of the parameters of a contact network models seems
likely to be an intractable problem, which would imply that these models are
amenable to neither \gls{ML} nor Bayesian inference. We have not proven this is
the case, but some intuition can be provided by examining the process involved
in the likelihood calculation. Consider a contact network model with parameters
$\theta$, and an estimated transmission tree $T$ with $n$ tips. In general, we
do not know the labels of the internal nodes of $T$, only the labels of its
tips. To fit this model using likelihood-based methods, we must calculate the
likelihood of $\theta$, that is, $\Pr(T \mid \theta)$. Let $\G$ be the set of
all possible contact networks, and $\Nu$ be the set of all possible labellings
of the internal nodes of $T$. We can write the likelihood as
\begin{align}
\begin{split}
  \label{eq:netlik}
  \Pr(T \mid \theta)
    &= \sum_{\nu \in \Nu} \Pr(T, \nu \mid \theta) \\
    &= \sum_{G \in \G} \sum_{\nu \in \Nu} \Pr(T, \nu \mid G, \theta) \Pr(G \mid \theta) \\
    &= \sum_{G \in \G} \sum_{\nu \in \Nu} \Pr(T, \nu \mid G) \Pr(G \mid \theta),
\end{split}
\end{align}
the last equality following from the fact that $T$ and $\nu$ depend only on
$G$, not on $\theta$. Although $\Pr(T, \nu \mid G)$ and $\Pr(G \mid \theta)$
may individually be straightforward to calculate, the number of possible
directed graphs on $N$ nodes is $2^{N(N-1)}$, larger if the nodes and edges in
the graph may have different labels or attributes. Hence, the number of terms
in the sum is at least exponential in $n$, as there must be at least $n$ nodes
in the network. In addition, \cref{eq:netlik} assumes that $T$ is complete,
meaning that all infected individuals were sampled. This is rarely the case in
practice - most often, we only have access to a subset of the infected
individuals. In this case, the likelihood calculation becomes even more
complex, because we must also sum over all possible complete trees.

Depending on the network model studied, it is possible that \cref{eq:netlik}
could be simplified into a tractable expression. However, a simpler alternative
to likelihood-based methods, which would apply to any network model, is
provided by \gls{ABC}. All of the ingredients required to apply \gls{ABC} to
this problem are readily available. Simulating networks is straightforward
under a variety of models. Epidemics on those networks, and the corresponding
transmission trees, can also be easily simulated. As mentioned above, contact
networks can profoundly affect transmission tree shape, and those shapes can be
compared using a highly informative similarity measure called the ``tree
kernel''~\autocite{poon2013mapping}. \Gls{ABC} can be implemented with
\gls{SMC}, which has several advantages over other
algorithms~\autocite{mckinley2009inference}. A recently-developed adaptive
algorithm requiring minimal tuning on the part of the user makes \gls{SMC} an
even more attractive approach~\autocite{del2012adaptive}. In summary, our
method to infer contact network parameters will combine the following:
stochastic simulation of epidemics on networks, the tree kernel, and adaptive
\gls{ABC}-\gls{SMC}. Since our distance measure is a kernel function, our
method is a type of kernel-\gls{ABC}. For ease of exposition, we will often use
the term ``kernel-\gls{ABC}'' to refer to our method specifically.

Empirical studies of sexual contact networks have found that these networks
tend to be scale-free~\autocite{colgate1989risk, liljeros2001web,
schneeberger2004scale}, meaning that their degree distributions follow a power
law (although there has been some disagreement, see
\autocite{handcock2004likelihood, bansal2007individual}). Preferential
attachment has been postulated as a mechanism by which scale-free networks
could be generated~\autocite{barabasi1999emergence}. This makes the \gls{BA}
model, one of the simplest preferential attachment models, a natural choice to
explore with our method. The second aim of this work is to use simulations to
investigate the parameters of the \gls{BA} model, including whether they have a
detectable impact on tree shape, and whether they can be accurately recovered
using kernel-\gls{ABC}.

Due to its high global prevalence and fast mutation rate, \gls{HIV} is one of
the most commonly-studied viruses in a phylodynamic context. Consequently, a
large volume of \gls{HIV} sequence data is publicly available, more than for
any other pathogen, and including sequences sampled from diverse geographic and
demographic contexts. At the time of this writing, there were $635400$ HIV
sequences publicly available in GenBank, annotated with 172 distinct countries
of origin. Since \gls{HIV} is almost always spread through either sexual
contact or sharing of injection drug supplies, the contact networks underlying
\gls{HIV} epidemics are driven by social dynamics and are therefore likely to
be highly nonrandom. Moreover, since no cure yet exists, efforts to curtail the
progression of an epidemic have relied on preventing further transmissions
through measures such as \gls{tasp} and education leading to behaviour change.
The effectiveness of this type of intervention can vary significantly based on
the underlying structure of the network and the particular nodes to whom the
intervention is targeted. Due to this combination of data availability and
potential public health impact, \gls{HIV} is an obvious context in which our
method could be applied. Therefore, the third and final aim of this work is to
apply kernel-\gls{ABC} to fit the \gls{BA} model to existing \gls{HIV}
outbreaks.

To summarize, this work has three objectives. First, we will develop a method
which uses kernel-\gls{ABC} to infer parameters of contact network models from
observed transmission trees. Second, we will use simulations to characterize
the parameters of the \gls{BA} network model in terms of their effect on tree
shape and how accurately they can be recovered with kernel-\gls{ABC}. Finally,
we will apply the method fit the \gls{BA} model to several real-world \gls{HIV}
datasets.
