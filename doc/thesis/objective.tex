\itshape
This section is intended to be a synthesis of all the background material
together into some research objectives.
\normalfont

The spread of a disease through a susceptible population is most often modelled
by assuming either a homogeneously mixed population, or a population divided
into a small number of homogeneously mixed compartments. This assumption, also
called the ``law of mass action'', implies that any two individuals in the same
compartment are equally likely to come into contact causing transmission.
Although this is clearly unrealistic for most populations, the predictions they
make have proven to be fairly accurate in practice. However, when there is
substantial contact heterogeneity, such as when considering a sexually
transmitted infection, the error introduced by assuming a panmictic population
may be substantial.

Contact network models provide an alternative to compartmental models which do
not require the assumption of panmixia. In addition to more accurate
prevalence predictions, the parameters of the networks themselves may be of
interest from a public health perspective. For example, certain vaccination
strategies may be more or less effective in curtailing an epidemic depending on
the underlying network's degree distribution.

Phylodynamic methods have been used to investigate a great diversity of
epidemiological parameters. However, as far as we are aware, no phylodynamic
methods have been developed to fit contact network models. The primary
objective of this work is to develop such a method.

Network structure has a substantial impact on the speed and pattern of epidemic
spread. Indeed, a small number of methods have been developed which exploit
this fact to perform statistical inference. Contact network parameters, such as
edge density and degree distribution, have been estimated from epidemiological
data, such as incidence reports and individual recovery times.

Contact networks' parameters can have a strong effect on the shape of
transmission trees, and therefore on viral phylogenies, resulting from
epidemics on those networks. Yet, this relationship has not yet been
``inverted'' to quantitatively estimate network parameters from transmission
trees.

\newcommand{\G}{\mathcal{G}}
\newcommand{\Nu}{\mathcal{N}}

Although we have not proven it analytically, calculating the likelihood of the
parameters of a contact network models seems likely to be an intractable
problem. This would imply that these models are amenable to neither \gls{ML}
nor Bayesian inference. To illustrate this, consider a contact network model
with parameters $\theta$, and an observed transmission tree $T$ with $n$ tips.
In general, we do not know the labels of the internal nodes of $T$, only the
labels of its tips. To fit this model using likelihood-based methods, we must
calculate the likelihood of $\theta$, that is, $\Pr(T \mid \theta)$. Let $\G$
be the set of all possible contact networks, and $\Nu$ be the set of all
possible labellings of the internal nodes of $T$. We can write the likelihood as
\begin{align}
\begin{split}
  \label{eq:netlik}
  \Pr(T \mid \theta)
    &= \sum_{\nu \in \Nu} \Pr(T, \nu \mid \theta) \\
    &= \sum_{G \in \G} \sum_{\nu \in \Nu} \Pr(T, \nu \mid G, \theta) \Pr(G \mid \theta) \\
    &= \sum_{G \in \G} \sum_{\nu \in \Nu} \Pr(T, \nu \mid G) \Pr(G \mid \theta),
\end{split}
\end{align}
the last equality following from the fact that $T$ and $\nu$ depend only on
$G$, not on $\theta$. Although $\Pr(T, \nu \mid G)$ and $\Pr(G \mid \theta)$
may individually be straightforward to calculate, the number of possible
directed graphs on $N$ nodes is $2^{N(N-1)}$, larger if the nodes and edges in
the graph may have different labels or attributes. Hence, the number of terms
in the sum is at least exponential in $n$, more if we do not know \textit{a
priori} how many nodes are in the network (as is likely). In addition,
\cref{eq:netlik} assumes that $T$ is complete, meaning that all infected
individuals were sampled. This is rarely the case in practice - most often, the
observed tree is a subsampled version of the true tree. In this case, the
likelihood calculation becomes even more complex, because we must also sum over
all possible complete trees.

Depending on the network model studied, it is possible that \cref{eq:netlik}
could be simplified into a tractable expression. However, a simpler alternative 
to likelihood-based methods, which would apply to any network model, is
provided by \gls{ABC}. All of the ingredients required to apply \gls{ABC} are
readily available. Simulating networks is straightforward under a variety of
models. Epidemics on those networks, and the corresponding transmission trees,
can also be easily simulated. As mentioned above, contact networks can
profoundly affect transmission tree shape, and those shapes can be compared
using a highly informative similarity measure. \Gls{SMC} has several advantages
over other algorithms for \gls{ABC}~\autocite{mckinley2009inference}, including
a recently-developed adaptive algorithm requiring minimal tuning on the part of
the user~\autocite{del2012adaptive}. In summary, our method to infer contact
network parameters will combine the following: stochastic simulation of
epidemics on networks, the tree kernel, and adaptive \gls{ABC}-\gls{SMC}.
