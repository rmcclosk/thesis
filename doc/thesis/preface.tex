The initial idea to use approximate Bayesian computation (ABC) to infer contact
network model parameters was Dr. Poon's, based on his previous work using ABC
to infer parameters of population genetic models. The tree kernel was
originally developed by Dr. Poon, but the version used here was implemented by
me to improve computational efficiency. The idea to apply sequential Monte
Carlo was mine, but Dr. Alexandre Bouchard-C\^ot\'e made me aware of the
adaptive version used in this work. Dr. Sarah Otto suggested the experiments
involving a network with a heterogeneous $\alpha$ parameter and peer-driven
sampling. Dr. Richard Liang provided guidance in the development of the
Gillespie simulation algorithm and statistical advice. The \software{netabc}
program, and all supplementary analysis programs, were written by me.

A version of chapter 2 has been submitted for publication with the title
``Reconstructing network parameters from viral phylogenies.'' An oral
presentation entitled ``Phylodynamic inference of contact network parameters
with kernel-ABC'' was given based on chapter 2 to the 23rd HIV Dynamics and
Evolution meeting on April 25, 2016, in Woods Hole, Massachusetts, USA (the
presentation was delivered remotely). A poster based on chapter 2 entitled
``Likelihood-free estimation of contact network parameters from viral
phylogenies'' is scheduled for presentation at the Intelligent Systems for
Molecular Biology meeting on July 8, 2016, in Orlando, Florida, USA.

Use of the BC data is in accordance with an ethics application that was
reviewed and approved by the UBC/Providence Health Care Research Ethics Board
(H07-02559). Rosemary M. McCloskey completed the Tri-Council Policy Statement:
Ethical Conduct for Research Involving Humans Course on Research Ethics (TCPS
2: CORE) tutorial on June 8, 2016.

\add{Source code for the \software{netabc} program is freely available at
\url{https://github.com/rmcclosk/netabc} under the GPL-3 license. Scripts to
run all computational experiments, as well as the source code for this thesis,
are available at \url{https://github.com/rmcclosk/thesis}.}
